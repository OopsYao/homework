\documentclass[cn]{homework}

\title{作业4}

\begin{document}
    \maketitle

    \problem
    \begin{proof}
        显然该模型对应的特征方程为
        \[\lambda^2-\theta_1\lambda-\theta_2\lambda=0\]
        则特征根$\lambda_{1,2}$满足
        \[\lambda_1+\lambda_2=\theta_1,
        \quad\lambda_1\lambda_2=-\theta_2\]
        注意到若特征根为虚数,则$\lambda_1,\lambda_2$相互共轭,因此
        \[(1-\lambda_1)(1-\lambda_2)=|1-\lambda_1|^2>0\]
        此处无法取得等号的原因是由于MA过程的可逆性;
        若特征根为两实数,则仍由可逆性有$1-\lambda_{1,2}>0$,
        则$(1-\lambda_1)(1-\lambda_2)>0$成立。
        因此
        \[-\lambda_1\lambda_2+\lambda_1+\lambda_2<1\]
        即
        \[\theta_1+\theta_2<1\]
        
        同理有
        \[(1-\lambda_1)(1-\lambda_2)=|1-\lambda_1|^2>0\]
        亦得到
        \[\theta_2-\theta_1<1\]

        同时由$|\lambda_{1,2}|<1$知
        \[|\theta_2|=|\lambda_1\lambda_2|<1\]
    \end{proof}

    \problem

    \problem
    \begin{subproblem}[(\alph*)]
        \item\label{pb:acf}
        \newcommand{\Cov}{\mathrm{Cov}}
        \newcommand{\Var}{\mathrm{Var}}
        由于有限项MA模型为平稳的,故
        对于任意$k\in\mathbb N$,有
        \[\begin{aligned}
            \gamma_k&=\Cov(Y_t,Y_{t-k})\\
                    &=\Cov(Z_t-\theta Z_{t-1},Z_{t-k}-\theta Z_{t-k-1})\\
                    &=\begin{cases}
                        \Var(Z_t-\theta Z_{t-1}),&k=0\\
                        \Cov(-\theta Z_{t-1},Z_{t-1}),&k=1\\
                        0,&k>1
                    \end{cases}\\
                    &=\begin{cases}
                        (1+\theta^2)\sigma^2,&k=0\\
                        -\theta\sigma^2,&k=1\\
                        0,&k>1
                    \end{cases}
        \end{aligned}\]
        因此
        \[\begin{aligned}
            \rho(k)&=\frac{\gamma_k}{\gamma_0}\\
                   &=\begin{cases}
                        1,&k=0\\
                        -\frac{\theta}{1+\theta^2},&k=1\\
                        0,&k>1
                   \end{cases}
        \end{aligned}\]

        \item
        令$\rho(1)=0.4$可解得
        \[\theta_1=-0.5,\theta_2=-2\]

        由于$\theta=\theta_1=-0.5$时,该MA模型具有可逆性,
        故更偏好$\theta=-0.5$。

        \item
        \begin{proof}
            由题意知$\sum Z_{t-n}$收敛,不妨设
            \[Z^*_n=\sum_{k=1}^nZ_{t-k},n=1,2,\ldots\]
            则
            \begin{equation}
                \label{eq:partial sum}
                Y_t=Z_t+C\cdot\lim_{n\to\infty}Z^*_n
            \end{equation}
            由中心极限定理知
            \[\lim_{n\to\infty}Z^*_n\]
            方差不存在,因此$Y_t$方差亦不存在。
            故不可能是宽平稳的。
        \end{proof}

        \item
        \begin{proof}
            将\cref{eq:partial sum}滞后一阶有
            \[\begin{aligned}
                Y_{t-1}&=Z_{t-1}+C\cdot\lim_{n\to\infty}(Z^*_n-Z_{t-1})\\
                    &=Z_{t-1}-C\cdot Z_{t-1}+C\cdot\lim_{n\to\infty}Z^*_n
            \end{aligned}\]
            故
            \[X_t=Z_t+(C-1)Z_{t-1}\]
            即$X_t$是MA(1)模型,自然具有平稳性。
        \end{proof}

        \item
        同题\ref{pb:acf},可得
        \[\rho_X(k)=\begin{cases}
            1,&k=0\\
            \frac{C-1}{1+(C-1)^2},&k=1\\
            0,&k>1
        \end{cases}\]
    \end{subproblem}

    \problem
    (a)为ARMA(1,1),且由表达式易知其为平稳且可逆的。
    (b)为MA(1),也易判断其为可逆的。
    (c)为AR(2),由于其特征根为一对共轭复数,不妨设为$\lambda,\bar\lambda$,
    则
    \[|\lambda|^2=|\bar\lambda|^2=\lambda\bar\lambda=0.75<1,\]
    因此该模型亦为平稳的。
\end{document}