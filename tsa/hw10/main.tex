\documentclass[cn]{homework}

\counterwithout{equation}{problem}

\title{作业10}

\begin{document}
    \maketitle

    \section{ADF检验与PP检验可靠性的比较}

    \subsection{ADF与PP检验简介}
    PP检验中,
    假设数据的生成过程为
    \begin{equation}
        \label{eq:real process}
        y_t=\rho y_{t-1}+u_t,u_t=\phi(B)\varepsilon_t
    \end{equation}
    这里$\varepsilon_t$独立同分布,期望为0,方差为$\sigma^2<\infty$。
    现我们希望在模型
    \[y_t=\alpha+\rho y_{t-1}+u_t\]
    中检验假设
    \[H_0:\alpha=0,\rho=1\]
    由于$u_t$并不是独立同分布的,因此t统计量
    \[t=\frac{\hat\rho-1}{\hat\sigma_{\hat\rho}}\]
    这里
    \[\begin{aligned}
    \hat\sigma_{\hat\rho}&=\sqrt{\frac{s^2}{\sum y_{t-1}^2}}\\
    s^2&=\frac{1}{T-1}\sum_{t=1}^T(y_t-\hat\rho y_{t-1})^2
    \end{aligned}\]
    并不具有极限分布(样本数量$T\to\infty$)为DF分布的性质。

    可以证明以下构造出的统计量
    具有极限分布
    \begin{equation}
        \label{eq:pp lim dist}
        \begin{aligned}
        Z_p=T(\hat\rho-1)-\frac{\hat\lambda^2
        -\hat\gamma_0^2}{2}\frac{T^2\hat\sigma_{\hat\rho}^2}{s^2}
        \Rightarrow
        \frac{\frac{1}{2}(W^2(1)-1)-W(1)\int_0^1W(r)\diff r}%
        {\int_0^1W^2(r)\diff r-\left(\int_0^1W(r)\diff r\right)^2}\\
        Z_t=t\sqrt{\frac{\hat\gamma_0}{\hat\lambda^2}}
        -\frac{\hat\lambda^2-\hat\gamma_0^2}{2\hat\gamma}
        \frac{T\hat\sigma_{\hat\rho}}{s}
        \Rightarrow
        \frac{\frac{1}{2}(W^2(1)-1)-W(1)\int_0^1W(r)\diff r}%
        {\sqrt{\int_0^1W^2(r)\diff r-\left(\int_0^1W(r)\diff r\right)^2}}
        \end{aligned}
    \end{equation}
    这里
    \[\begin{aligned}
        \hat\lambda^2&=\hat\gamma_0^2
        +2\sum_{j=1}^q\left(1-\frac{j}{q+1}\right)\hat\gamma_j\\
        \hat\gamma_j&=\frac{1}{T}\sum_{t=j+1}^T\hat u_t\hat u_{t-j},j=0,1,\ldots,q
    \end{aligned}\]
    $q$是残差序列自相关的最大阶数,$\hat u_t$为残差。

    ADF检验是假设生成过程为AR(p),
    \[\psi(B)y_t=\varepsilon_t\]
    这里$\psi(x)=1-\phi_1 x-\phi_2 x^2-\cdots-\phi_p x^p$。
    而上式可以等价转化为
    \[y_t=\rho y_{t-1}+\zeta_1\Delta y_{t-1}+\zeta_2\Delta y_{t-2}
    +\cdots+\zeta_{p-1}\Delta y_{t-p+1}+\varepsilon_t\]
    这里$\rho=\phi_1+\phi_2+\cdots+\phi_p,\zeta_j=-(\phi_{j+1}+\cdots+\phi_p)$。
    则在如下模型中检验$H_0:\rho=1$
    \[y_t=\alpha+\rho y_{t-1}+\zeta_1\Delta y_{t-1}+\zeta_2\Delta y_{t-2}
    +\cdots+\zeta_{p-1}\Delta y_{t-p+1}\]
    当假设成立时,下面的统计量与\cref{eq:pp lim dist}具有相同的极限分布
    \[\begin{aligned}
        k=\frac{T(\hat\rho-1)}{1-\hat\zeta_1-\hat\zeta_2-\cdots-\hat\zeta_{p-1}}
        \Rightarrow
        \frac{\frac{1}{2}(W^2(1)-1)-W(1)\int_0^1W(r)\diff r}%
        {\int_0^1W^2(r)\diff r-\left(\int_0^1W(r)\diff r\right)^2}\\
        t\Rightarrow
        \frac{\frac{1}{2}(W^2(1)-1)-W(1)\int_0^1W(r)\diff r}%
        {\sqrt{\int_0^1W^2(r)\diff r-\left(\int_0^1W(r)\diff r\right)^2}}
    \end{aligned}\]
    这里$\hat\zeta_j$为$\zeta_j$的最小二乘估计。

    \subsection{文献相关结论}

    文献中模拟的数据服从\cref{eq:real process},
    分析了ADF检验与PP检验的实际检验水平与势,并认为ADF中t检验的稳健性
    要好于其他检验,同时发现实际检验水平较高(低)者通常伴随着较高(低)的势。
    进而得出ADF与PP检验可靠性比较需要依赖于对两类错误权衡;另一方面由于数据的
    生成过程是未知的,因而认为传统的ADF与PP检验存在较严重的校验水平扭曲,且扭曲程度
    与截断参数的选择有关。

    根据不同的滞后截断参数,文献分别用传统的DF临界值与理论上极限分布的分位数为临界值
    进行了对比,认为应该依据样本序列,充分考虑样本可能的数据生成过程模型,以统计量的
    实际分位数为临界值,选择势较高的检验方法进行检验。

    这给我们的启示是,为了追求准确性,应尽量选取真实的分布临界值,由此得到的实验结果具有
    较强的可靠性,同时实验中要充分考虑到检验方法的势,偏好更高的势。


    \section{单位根检验中样本长度的选择}
    单位根检验中,在备择假设成立的条件下——即模型为平稳过程,统计量\sidenote{实际上
    该分布左边,极限情况下表现为DF分布,文献为了便于计算分位数使用正态分布代替。}
    $T(\hat\rho-1)\dot\sim\mathcal N\left(T(1-\rho),\sqrt{T(1-\rho^2)}\right)$,
    从而临界值取为$C$的决策有不犯第二类错误的概率(即检验功效)为
    \[P(T(\hat\rho-1)<C)=ND\left(\frac{C+T(1-\rho)}{\sqrt{T(1-\rho^2)}}\right)\]
    这里$ND$为标准正态累积分布函数。
    于是反过来,要求了一个最低检验功效时就对样本数量$T$做出了限制
    \[\begin{aligned}
        T=\frac{Z^2_p(1+\rho)-2C+\sqrt{(Z_p^2(1+\rho)-2C)^2-4C^2(1-\rho)}}{2(1-\rho)}
    \end{aligned}\]
    或者是给定
    最低检验功效和样本数量$T$时,就确定了一个$\rho$值,这是在该样本和功效要求下
    所能区分的最大$\rho$值。
    \[\rho=\frac{\left(1+\frac{C}{T}\right)\left(1+\sqrt{%
        1-\left(1+\frac{Z_p^2}{T}\right)
        \left(1-\left(\frac{Z_p/\sqrt{T}}{1+C/T}\right)^2\right)
    }\right)}{1+Z_p^2/T}\]

    因此文献能够得出,倘若$\rho$很接近1,检验功效要求的样本数量将会非常巨大——少数样本
    根本无法满足检验功效。

    因此有限的样本数量中,是无法判断假设$\rho=1$的,文献通过讨论有无截距项等情况,
    先是用正态分布的方式拟合分位点,后面用曲线拟合的方式来提高精度(因为真实分布是
    左偏的),导出了类似上面的相关的$T,\rho$的表达式,由此在给定功效要求的基础上,
    建议修改假设$H_0:\rho=1$为$H_0:\rho\geq\rho_0$。

    也就是说,在实际检验中,我们往往忽略了检验的功效,这导致小样本情况下犯第二类错误的
    的概率将会比较大,因此可以适当地减弱我们的假设为$\rho\geq\rho_0$。

\end{document}