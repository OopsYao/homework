\documentclass[nobib]{tufte-handout}
\RequirePackage{amssymb} % Maybe `unicode-math' should after this
\RequirePackage{titlesec} % Tweak section
\RequirePackage{unicode-math} % To setmathfont
\RequirePackage{xparse} % NewDocumentCommand
\RequirePackage{letltxmacro} % LetLtxMacro
\RequirePackage{ctex} % Chinese

% Restyle section
\titleformat*{\section}{\Large\bfseries}

% Font setting under xetex
% Specific package needed. `unicode-math` satisfies
\IfFontExistsTF{TeX Gyre Pagella}{\setmainfont{TeX Gyre Pagella}}{}
\IfFontExistsTF{TeX Gyre Pagella Math}{\setmathfont{TeX Gyre Pagella Math}}{}

% Reformat title, author and date
\LetLtxMacro{\OldTitle}{\title}
\RenewDocumentCommand{\title}{ o m }{%
    \IfNoValueTF{#1}{%
        \OldTitle[#2]{\textnormal{\textbf{#2}}}
    }{%
        \OldTitle[#1]{\textnormal{\textbf{#2}}}
    }
}

\LetLtxMacro{\OldAuthor}{\author}
\RenewDocumentCommand{\author}{ o m }{%
    \IfNoValueTF{#1}{%
        \OldAuthor[#2]{\small{#2}}
    }{%
        \OldAuthor[#1]{\small{#2}}
    }
}

% The following code seems to hava something wrong
% \LetLtxMacro{\OldDate}{\date}
% \RenewDocumentCommand{\date}{ o }{%
%     \IfNoValueTF{#1}{ \OldDate{} }{ \OldDate{\small{#1}} }
% }
\date{\today} % Default date
\date{}

% Try to fix some error
% It seems to be a bug under `xetex'
% See https://github.com/Tufte-LaTeX/tufte-latex/issues/64
% 
% Set up the spacing using fontspec features
\renewcommand\allcapsspacing[1]{{\addfontfeature{LetterSpace=15}#1}}
\renewcommand\smallcapsspacing[1]{{\addfontfeature{LetterSpace=10}#1}}

% Code display
\usepackage{xcolor}
\definecolor{lbcolor}{rgb}{0.98,0.98,0.98} 
\usepackage{listings}
\lstset{
    basicstyle=\ttfamily,
    breaklines=true,
    backgroundcolor=\color{lbcolor},
    postbreak=\mbox{\textcolor{red}{$\hookrightarrow$}\space}
}

% Crossref
\usepackage{cleveref}
\crefformat{equation}{式~(#2#1#3)}
\crefformat{table}{表~#2#1#3}
\crefformat{figure}{图~#2#1#3}


\def\keywords#1{\noindent\textbf{关键词:}#1}

\title{XX实证研究}
\input{.author.cn}

\begin{document}
    \maketitle    

    \section{摘要}
    \begin{abstract}
        随着国民收入的提高,人们的物质文化需求不断增长,外出旅游成为
        许多人的选择。

        \keywords{时间序列;VAR模型;脉冲响应函数;方差分解}
    \end{abstract}

    \section{引言}
    随着国民经济收入的不断增长,人们的物质文化需求逐渐提升,
    旅游作为一种文化消遣方式的重要性逐渐提升,越来越多的人们
    利用假期甚至是周末的休闲时光出门旅游,旅游业的迅速增长
    促进了我国经济的发展,而反过来人们收入水平的提高亦会
    增长人们的旅游需求。

    \section{文献综述}
    随着旅游业成为学术界研究的一个热点领域,各种各样的
    模型和理论被运用其中,比较常见的有VAR、ECM、TVP等计量模型,
    这大大促进了领域专业性的发展\cite{song2006forecasting}。
    许多学者会利用VAR以及脉冲响应和方差分解分析的方式
    进行建模,
    余丹阳\cite{余丹阳2018基于}选取了湖南省的旅游数据,
    构建VAR模型,并对其进行脉冲响应分析和方差分解分析,
    研究了湖南省旅游总收入与湖南省GDP的相互影响关系;
    王纯阳\cite{王纯阳2010基于}
    利用美国客源市场的相关数据,
    建立旅游需求与旅游价格、收入、替代价格等变量之间的
    VAR模型发现美国对中国的旅游需求
    受到自身的波动、旅游价格等影响;
    吴丽云\cite{吴丽云2010我国旅游业与经济增长的关系分析}
    采用协整性检验和Granger因果检验方法对我国1985年以来的
    国内旅游、入境旅游、出境旅游和经济增长之间的关系进行了检验,
    发现经济增长与国内旅游业的发展相辅相成,
    但出境旅游对中国经济增长有一定的抑制作用。

    总的说来

    \section{变量选取与数据来源}
    我们从中国统计年鉴\cite{中华2014中}中

    \section{模型选取与实证分析}
    \subsection{模型设定}

    \subsection{平稳性检验}

    \subsection{协整检验}

    \subsection{Granger因果检验}

    \subsection{脉冲响应函数分析}

    \subsection{方差分解}

    \section{结论}

    \bibliographystyle{unsrt}
    \bibliography{refs}
    \appendix
    \section{代码(Python)}
    \lstinputlisting[language=Python]{var.py}
\end{document}