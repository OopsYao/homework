\documentclass[cn]{homework}

\title{作业13}

\newcommand{\cov}{\mathrm{Cov}}

\begin{document}
    \maketitle

    \problem
    \begin{subproblem}[(\alph*)]
        \item
        \begin{proof}
            由于$Z_t$并不满足不同项直接不相关的要求,倘若存在可逆
            线性变换$U_t=MZ_t$,使得$\Sigma_U$为对角阵,这里要求$M$可逆。

            原模型可以表示为
            \begin{equation}
                \label{eq:X var}
                X_t=\mu+\Pi_iX_{t-1}+\Pi_2X_{t-2}+Z_t
            \end{equation}
            同左乘$M$得
            \[\begin{aligned}
                MX_t&=M\mu+M\Pi_1+M\Pi_2X_{t-2}+MZ_t\\
            \end{aligned}\]
            即关于$Y_t$的VAR(2)模型,这里$Y_t=MX_t$,
            \begin{equation}
                \label{eq:Y var}
                Y_t=M\mu+M\Pi_1M^{-1}Y_t+M\Pi_2M^{-1}Y_{t-2}+U_t
            \end{equation}
            同时由于
            \[EY_t=M\cdot EX_t,\Sigma_Y=M\Sigma_XM'\]
            从而$X_t$与$Y_t$同平稳。

            而\cref{eq:Y var}的平稳条件为
            \[B_1=\begin{pmatrix}
                M\Pi_1 M^{-1} & M\Pi_2 M^{-1} \\
                I &
            \end{pmatrix}
            =NA_1N^{-1}\]
            的特征值全在单位圆以内,这里
            \[N=\begin{pmatrix}
                M & \\
                  & M
            \end{pmatrix},
            A_1=\begin{pmatrix}
                \Pi_1 & \Pi_2 \\
                I
            \end{pmatrix}\]
            又注意到相似矩阵的特征值相同,故只需判断$A_1$的
            特征值即可。

            由于$\Sigma_{Z}=PP'$,而$\det P\neq 0$,故可以选取$M=P^{-1}$,
            使得$\Sigma_{U}=M\Sigma_{Z}M'=I$,从而
            $M$的存在性得到保证。
        
            经过计算可以发现$A_1$的
            特征方程为
            \[\lambda\left(\lambda^5-\frac{19}{10}\lambda^4
            +\frac{63}{50}\lambda^3-\frac{323}{1000}\lambda^2
            -\frac{21}{1000}\lambda+\frac{2}{125}\right)=0\]
            其有4个实特征值
            为0,-0.179,0.346,0.910,
            以及一对共轭复特征值$\lambda,\bar\lambda$,且
            $|\lambda|^2=\lambda\bar \lambda=0.284$,
            因此特征值全在单位圆以内,
            故$B_1$的特征值也如此,故$Y_t$
            是平稳的,从而$X_t$平稳。


        \end{proof}

        \item
        \cref{eq:X var}两端同取期望,由平稳性可知
        \[EX_t=\mu+\Pi_1EX_t+\Pi_2EX_t\]
        即
        \[(I-\Pi_1-\Pi_2)EX_t=\mu\]
        解得
        \[EX_t=\frac{5}{61}\begin{pmatrix}
            2\\
            26\\
            -1
        \end{pmatrix}\]

        \item
        由于\cref{eq:Y var}的VAR(1)形式为
        \[
            \begin{pmatrix}
                Y_t\\
                Y_{t-1}
            \end{pmatrix}
        =\begin{pmatrix}
            \mu \\
            0
        \end{pmatrix}
        +B_1\begin{pmatrix}
            Y_{t-1}\\
            Y_{t-2}
        \end{pmatrix}+
        \begin{pmatrix}
            U_t\\
            0
        \end{pmatrix}\]
        故两端左乘$N^{-1}$得
        \[\begin{pmatrix}
            X_t\\
            X_{t-1}
        \end{pmatrix}
        =\begin{pmatrix}
            M^{-1}\mu\\
            0
        \end{pmatrix}
        +A_1\begin{pmatrix}
            X_{t-1}\\
            X_{t-2}
        \end{pmatrix}
        +\begin{pmatrix}
            Z_t\\
            0
        \end{pmatrix}\]
        即
        \[\begin{pmatrix}
            X_t\\
            X_{t-1}
        \end{pmatrix}
        =\begin{pmatrix}
            1.1\\
            0.3\\
            0   \\
            0   \\
            0   \\
            0
        \end{pmatrix}
        +\begin{pmatrix}
            0.7 & 0.1 &     & -0.2 &     &     \\
                & 0.4 & 0.1 &      & 0.1 & 0.1 \\
            0.9 &     & 0.8 &      &     &     \\
            1   &     &     &      &     &     \\
                & 1   &     &      &     &     \\
                &     & 1   &      &     &     \\
        \end{pmatrix}
        \begin{pmatrix}
            X_{t-1}\\
            X_{t-2}
        \end{pmatrix}
        +\begin{pmatrix}
            Z_t\\
            \boldsymbol 0
        \end{pmatrix}\]
    \end{subproblem}
\end{document}