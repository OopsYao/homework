\documentclass[cn]{homework}

\title{第2周作业}

\begin{document}
    \maketitle

    \section{题1}
    \begin{proof}
    设随机变量$Y=(X-\mu)/\sigma\sim N(0,1)$,故
    \begin{align*}
        E\e^{uX}&=E\e^{u(\mu+\sigma Y)}\\
                &=\e^{u\mu}E\e^{u\sigma Y}\\
                &=\e^{u\mu}
                    \int_{-\infty}^\infty
                      \e^{u\sigma x}\frac{\e^{-\frac{x^2}{2}}}{\sqrt{2\pi}}
                    \diff x\\
                &=\e^{u\mu}\cdot\e^\frac{u^2\sigma^2}{2}
                    \int_{-\infty}^\infty
                    \frac{\e^{-\frac{(x-u\sigma)^2}{2}}}{\sqrt{2\pi}}
                    \diff x\\
                &=\e^{u\mu+\frac{u^2\sigma^2}{2}}
    \end{align*}
    \end{proof}

    \section{题2}
    \begin{proof}
        由测度变换公式
        \begin{equation}
            \label{eq:p2}
            \begin{aligned}
            \tilde E\e^{uY}&=E(\e^{uY}Z)\\
                           &=E\e^{uY-\theta X-\theta^2/2}\\
                           &=E\e^{u\theta-\theta^2/2+(u-\theta)X}
            \end{aligned}
        \end{equation}
        由于$X$服从$P$测度下的标准正态分布,故
        \[E\e^{(u-\theta)X}=\e^{(u-\theta)^2/2}\]
        因此\cref{eq:p2}变为
        \[\tilde E\e^{uY}=\e^{u\theta-\theta^2/2}
        \cdot\e^{(u-\theta)^2/2}=\e^{u^2/2}\]
        即$Y$在$\tilde P$下服从标准正态分布。
    \end{proof}

    \section{题3}
    \begin{subproblem}
        \item
        \begin{proof}
        由于
        \[\tilde P(\Omega)=\int_\Omega Z\diff P=EZ\]
        故证明$EZ=1$即可。

        由于$X$在$P$下服从参数$\lambda$的指数分布,故
        \begin{align*}
            EZ&=\int_0^\infty\frac{\bar\lambda}{\lambda}\e^{-(\bar\lambda-\lambda)x}
            \cdot\lambda\e^{-\lambda x}\diff x\\
            &=\int_0^\infty\bar\lambda\e^{-\bar\lambda x}\diff x\\
            &=1
        \end{align*}
        因此$\tilde P(\Omega)=1$。
        \end{proof}
        
        \item
        同样地,由于$X$在$P$下的分布已知,故$\tilde P$下的分布函数
        \begin{align*}
            \tilde P\{X\leq a\}&=\int_{X\leq a}Z\diff P\\
            &=\int_0^a\frac{\bar\lambda}{\lambda}\e^{-(\bar\lambda-\lambda)x}
              \cdot\lambda\e^{-\lambda x}\diff x\\
            &=\int_0^a\bar\lambda\e^{-\bar\lambda x}\diff x\\
            &=1-\e^{-\bar\lambda a}\quad\forall a\geq 0
        \end{align*}
    \end{subproblem}
    \sdfjljdldksjf

\end{document}