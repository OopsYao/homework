\documentclass[cn]{homework}

\title{第七周作业}

\begin{document}
    \maketitle

    \problem
    \begin{proof}
        由于$d$为度量,故非负性显然,而
        \[||x||=d(x,0)=0\Leftrightarrow x=0\]
        因此正定性也满足。
        同时由于$d$的相似性有
        \[||\alpha x||=d(\alpha x,0)=|\alpha|d(x)
        =|\alpha|\cdot||x||\]
        正齐次得证。
        而由$d$的平移不变性
        \[||x+y||=d(x+y,0)=d(x,-y)
        \leq d(x,0)+d(0,-y)=||x||+||y||\]
        从而三角不等式得证。
        因此$(X,||\cdot||)$为赋范线性空间。
    \end{proof}

    \problem
    \begin{subproblem}
        \item
        \begin{proof}
            非负性是显然的,由于对于任意$\alpha\in\mathbb R$,
            \[||\alpha x||_1=
            \left(\int_a^b(|\alpha x(t)|^2+|\alpha x'(t)|^2)\diff t\right)^{1/2}
            =|\alpha|\cdot||x||_1\]
            故正齐次得证。

            同时,显然$||0||_1=0$,而倘若$||x||_1=0$,
            则$x(t)$在$[a,b]$上几乎处处为0,进一步由$x(t)$的连续性,
            $x(t)=0$,因此正定性得证。

            为记号的简便性,我们记
            \[||x||_2=\sqrt{\int_a^b|x(t)|^2\diff t}\]
            于是
            \[||x||_1=\sqrt{||x||_2^2+||x'||_2^2}\]
            因此
            \begin{multline*}
            (||x||_1+||y||_1)^2=
            ||x||_2^2+||y||_2^2
            +||x'||_2^2+||y'||_2^2\\
            +2\sqrt{(||x||_2^2+||x'||_2^2)(||y||_2^2+||y'||_2^2)}
            \end{multline*}
            而由Cauchy不等式,
            \[(||x||_2^2+||x'||_2^2)(||y||_2^2+||y'||_2^2)
            \geq(||x||_2\cdot||y||_2+||x'||_2\cdot||y'||_2)^2\]
            因此
            \[(||x||_1+||y||_1)^2
            \geq(||x||_2+||y||_2)^2+(||x'||_2+||y'||_2)^2\]
            进一步由Minkovski不等式
            \[(||x||_1+||y||_1)^2\geq ||x+y||_2^2+||x'+y'||_2^2
            =||x+y||_1^2\]
            即
            \[||x+y||_1\leq||x||_1+||y||_1\]
            故三角不等式得证,因此$||\cdot||_1$为范数。
        \end{proof}

        \item
        不完备。考虑函数列
        \[x_n(t)=\begin{cases}
            0,&x\in[a,\frac{a+b}{2}]\\
            \xi_n(t),&x\in(\frac{a+b}{2},\frac{a+b}{2}+\frac{b-a}{3n})\\
            1,&x\in[\frac{a+b}{2}+\frac{b-a}{3n},b]
        \end{cases}
        \quad n=1,2,3,\ldots\]
        这里$\xi_n(t)$为次数不超过$K$的多项式函数\sidenote{事实上只要取连接两点,且保证
        整体连续可微的一致有界的函数列即可,直觉上讲这样的函数列有很多。},且满足
        \[\begin{aligned}
            \xi_n\left(\frac{a+b}{2}\right)=0,
            \xi_n\left(\frac{a+b}{2}+\frac{b-a}{3n}\right)=1\\
            \xi_n'\left(\frac{a+b}{2}\right)=0,
            \xi_n'\left(\frac{a+b}{2}+\frac{b-a}{3n}\right)=0\\
            \xi_n''\left(\frac{a+b}{2}\right)=0,
            \xi_n''\left(\frac{a+b}{2}+\frac{b-a}{3n}\right)=0\\
        \end{aligned}\]
        因此$x_n(t)\in C^1[a,b]$,且一致有界\sidenote{可以让$K$取足够大的
        值来使得$\xi_n$在$(\frac{a+b}{2},\frac{a+b}{2}+\frac{b-a}{3n})$
        上单增,因此$|\xi_n|\leq 1$,一致有界,进一步$\xi_n'$一致有界。}

        于是对于任意$m>n$,
        \[\begin{aligned}
            ||x_m-x_n||_1^2=&\int_{\frac{a+b}{2}}^{\frac{a+b}{2}+\frac{b-a}{3n}}
            |\xi_m(t)-\xi_n(t)|^2+|\xi'_m(t)-\xi'_n(t)|^2\diff t\\
            &\leq M_1\cdot\frac{b-a}{3n}\to 0
        \end{aligned}\]
        这里$M_1$为常数,由$x_n,x_n'$的一致有界性故存在。
        于是$\{x_n\}_{n=1}^\infty$为Cauchy列。倘若$(C^1[a,b],||\cdot||_1)$
        为完备的,则存在$x\in C^1[a,b]$使得$x_n\to x$。
        考虑函数
        \[y(t)=\begin{cases}
            0,&x\in[a,\frac{a+b}{2})\\
            \frac{1}{2},&x=\frac{a+b}{2}\\
            1,&x\in(\frac{a+b}{2},b]
        \end{cases}\]
        则由Minkovski不等式
        \[\begin{aligned}
        \sqrt{\int_a^b|x(t)-y(t)|^2\diff t}
        &\leq\sqrt{\int_a^b|x(t)-x_n(t)|^2\diff t}
        +\sqrt{\int_a^b|x_n(t)-y(t)|^2\diff t}\\
        &\leq ||x-x_n||_1
            +\sqrt{\int_{\frac{a+b}{2}}^{\frac{a+b}{2}+\frac{b-a}{3n}}
                \left|\xi_n(t)-\frac{1}{2}\right|^2\diff t}\\
        &\leq ||x-x_n||_1+\sqrt{M_2\cdot\frac{b-a}{3n}}
        \end{aligned}\]
        这里$M_2$同样地由于$x_n$的一致有界而存在。
        两端取$n\to\infty$可得
        \[\int_a^b|x(t)-y(t)|^2\diff t=0\]
        于是$x(t)$与$y(t)$在$(a,b)$上几乎处处相等,而两者皆在
        挖去中点的区间$[a,b]$上连续,因此相等,即$x(t)$在
        $[a,(a+b)/2)$上为0,在$((a+b)/2,b]$上为1,这和$x\in C^1[a,b]$
        是矛盾的,因此不完备。
    \end{subproblem}

    \problem
    \newcommand{\img}{\mathrm i}
    \begin{proof}
        设$x_k=(a_{k1}+b_{k1}\img,a_{k2}+b_{k2}\img,\ldots,a_{kn}+b_{kn}\img)$
        为$\mathbb C^n$中的Cauchy列,这里$a_{ki},b_{ki}\in\mathbb R,i=1,2,\ldots,n$。
        因此对于任意$\varepsilon>0$,存在$N$使得
        $l,m>N$时有
        \[||x_l-x_m||=\max_i\sqrt{(a_{li}-a_{mi})^2+(b_{li}-b_{mi})^2}
        <\varepsilon\]
        于是对于任意$i=1,2,\ldots,n$
        \[\begin{aligned}
            |a_{li}-a_{mi}|&<\varepsilon\\
            |b_{li}-b_{mi}|&<\varepsilon
        \end{aligned}\]
        故$\{a_{ki}\}^\infty_{k=1},\{b_{ki}\}^\infty_{k=1}$为$\mathbb R$
        中的Cauchy列,于是由$\mathbb R$的完备性,存在$a_i,b_i\in\mathbb R$
        使得
        \[\begin{aligned}
            \lim_{k\to\infty}a_{ki}&=a_i\\
            \lim_{k\to\infty}b_{ki}&=b_i\\
        \end{aligned}\]
        考虑$x=(a_1+b_1\img,a_2+b_2\img,\ldots,a_n+b_n\img)\in\mathbb C^n$,
        则
        \[||x_k-x||=\max_i\sqrt{(a_{ki}-a_i)^2+(b_{ki}-b_i)^2}
        \to 0\quad(k\to\infty)\]
        因此$\mathbb C^n$是完备的。
    \end{proof}

    \problem
    \begin{proof}
        非负性是显然的,而$||x||=0$等价于$x(t)$在$[0,1]$上几乎处处为0,进一步由
        $x$的连续性知$x$在$[0,1]$上处处为0,即$x=0$,正定性得证。
        对于任意$\alpha\in\mathbb R$,
        \[||\alpha x||=\left(\int_0^1|\alpha\cdot x(t)|^2\diff t\right)^{1/2}
        =\alpha\cdot||x||\]
        故具有正齐次。最后由Minkovski不等式知三角不等式成立,因此$||\cdot||$
        为范数。
        
        考虑$[0,1]$上的连续函数列$x_n$定义如下,$x_n$在$[0,1/2]$上为0,
        在$[1/2+1/(3n),1]$上为1,其余部分用直线连接,则对于任意$\varepsilon>0$,
        存在$N>1/(3\varepsilon^2)$,当$m\geq n>N$时,
        \[\begin{aligned}
            ||x_m-x_n||^2=\int_{\frac{1}{2}}^{\frac{1}{2}+\frac{1}{3n}}
            |x_m(t)-x_n(t)|^2\diff t
            \leq\frac{1}{3n}<\varepsilon^2
        \end{aligned}\]        
        因此$\{x_n\}_{n=1}^\infty$为Cauchy列。倘若$X$完备,则存在
        连续函数$x\in X$,使得$x_n\to x$。现考虑函数
        \[y(t)=\begin{cases}
            0,&t\in[0,\frac{1}{2})\\
            \frac{1}{2},&t=\frac{1}{2}\\
            1,&t\in(\frac{1}{2},1]
        \end{cases}\]
        则由Minkovski不等式
        \[\begin{aligned}
            \sqrt{\int_0^1|x(t)-y(t)|^2\diff t}
            &\leq\sqrt{\int_0^1|x(t)-x_n(t)|^2\diff t}
            +\sqrt{\int_0^1|x_n(t)-y(t)|^2\diff t}\\
            &\leq ||x-x_n||+\sqrt{\int_{\frac{1}{2}}^{\frac{1}{2}+\frac{1}{3n}}
             |x_n(t)-y(t)|\diff t}\\
            &\leq ||x-x_n||+\sqrt{\frac{1}{3n}}
        \end{aligned}\]
        取$n\to\infty$可得
        \[\int_0^1|x(t)-y(t)|^2\diff t=0\]
        于是$x(t)$与$y(t)$在$[0,1]$上几乎处处相等,进一步由两者
        在$[0,1/2)\cup (1/2,1]$上连续,可知$x$在$[0,1/2)$上为0,
        而在$(1/2,1]$上为1,这与$x$在$[0,1]$上连续矛盾,因此
        $X$在此范数下不完备。
    \end{proof}
\end{document}