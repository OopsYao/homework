\documentclass[cn]{homework}

\title{第十五周作业}

\begin{document}
    \maketitle

    \problem
    \begin{proof}
        由H\"older不等式知,
        \[\sum_{k=1}^\infty |a_k\xi_k|
        \leq\sup_{k\geq 1}|a_k|\sum_{k=1}^\infty|\xi_k|<\infty\]
        从而$T$是$l^1$上的算子,显然$T$也是线性的。
        又$\sup_{k\geq 1}|a_k|$与$\xi_k,k=1,2,\ldots$无关,
        故$T$为有界线性算子。

        % TODO ||T||
    \end{proof}

    \problem
    \begin{proof}
        必要性,对于$x(t)\equiv 1$,应有$(Tx)(t)=\alpha(t)\in C[a,b]$,即证。
        
        充分性,
        对于任意$x\in C[a,b]$,由于$\alpha(\cdot)$连续,
        应有$\alpha(t)x(t)$为关于$t$的连续函数,而$T$显然是线性的,
        从而$T$为线性算子。
        同时$\alpha(t)$在$[a,b]$上有界$M$(与$x$无关)使得
        \[\alpha(t)x(t)\leq Mx(t),t\in[a,b]\]
        从而
        \[\sup_{t\in[a,b]}\alpha(t)x(t)\leq M\sup_{t\in[a,b]}x(t)\]
        因此$T$为有界线性算子。
    \end{proof}

    \problem
    \begin{subproblem}[(\arabic*)]
        \item
        由于
        \[\begin{aligned}
            ||Tx||&=\max_{-1\leq t\leq 1}|x'(t)|\\
            &=\max\left\{\max_{-1\leq t\leq 1}|x(t)|,\max_{-1\leq t\leq 1}|x'(t)|\right\}\\
            &=||x||_1
        \end{aligned}\]
        从而$T$是有界的
        
        \item
        由微分中值定理存在依赖于$t$的$\xi$满足$-1\leq\xi\leq t$使得
        \[x(t)=x(-1)+(t+1)x'(\xi)\]
        从而
        \[\max_{-1\leq t\leq 1}|x(t)|
        =\max_{-1\leq t\leq 1}|x(-1)+(t+1)x'(\xi)|\]

        由于
        \[\begin{aligned}
            \max_{-1\leq t\leq 1}|x'(t)|
            &\leq\max_{-1\leq t\leq 1}\left|x'(t)+\frac{1}{2}x(-1)\right|
            +\frac{1}{2}|x(-1)|
        \end{aligned}\]
        % TODO More thinking
    \end{subproblem}

    \problem
    \begin{subproblem}[(\arabic*)]
        \item
        由于
        \[\begin{aligned}
            ||Tf||&=\max_{x\in[a,b]}\left|\int_a^xf(t)\diff t\right|\\
            &\leq\max_{x\in[a,b]}\int_a^x|f(t)|\diff t\\
            &=\int_a^b|f(t)|\diff t\\
            &=||f||
        \end{aligned}\]
        即对于$f\neq\boldsymbol 0$,$||Tf||/||f||$有上界1,
        而当$f\equiv 1$时,
        \[||Tf||=\max_{x\in[a,b]}|x-a|=b-a=\int_a^b\diff t=||f||\]
        从而知
        \[||T||=\sup_{f\neq\boldsymbol 0}\frac{||Tf||}{||f||}=1\]

        \item
        由于
        \[\begin{aligned}
            ||Tf||&=\int_a^b\left|\int_a^xf(t)\diff t\right|\diff x\\
            &\leq\int_a^b\left(\int_a^x|f(t)|\diff t\right)\diff x\\
            &=\int_a^b\left(\int_t^b|f(t)|\diff x\right)\diff t\\
            &=\int_a^b(b-t)|f(t)|\diff t\\
            &\leq(b-a)||f||
        \end{aligned}\]
        注意到上式中我们使用了二重积分的换序。

        % TODO Find an example s.t. ||Tf||=(b-a)||f||
    \end{subproblem}

    \problem
    \begin{proof}
        显然$f$是一个线性泛函,我们只需要注意到对于$x\neq\boldsymbol 0$,
        \[||fx||=|x(a)-x(b)|
        \leq|x(a)|+|x(b)|
        \leq 2\max_{t\in[a,b]}|x(t)|
        =2||x||\]
        故$||fx||\leq ||x||$具有上界1,
        考虑$x(t)=t-(a+b)/2$(关于$((a+b)/2,0)$对称),从而
        $||x||=(b-a)/2$,而$||fx||=b-a=2||x||$,故有
        \[||f||=\sup_{f\neq\boldsymbol 0}\frac{||fx||}{||x||}
        =2\]
        上述确界的存在也证明了$f$为有界线性泛函。
    \end{proof}

    
\end{document}