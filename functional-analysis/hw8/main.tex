\documentclass[cn]{homework}

\title{第九周作业}

\begin{document}
    \maketitle

    \problem
    由于
    \[\left(\int_0^t|f(t)|^2\right)^{\frac{1}{2}}
    \leq
    \left(\int_0^t(1+t)|f(t)|^2\right)^{\frac{1}{2}}
    \leq
    \left(\int_0^t2|f(t)|^2\right)^{\frac{1}{2}}\]
    故
    \[||f||_2\leq||f||_3\leq\sqrt{2}||f||_2\]
    即$||\cdot||_2$和$||\cdot||_3$等价。

    如果$||\cdot||_1$和$||\cdot||_2$等价,则存在$K$使得
    对于任意$f\in L^2[0,1]$有
    \[||f||_2\leq K||f||_1\]
    因此对于$f\neq \boldsymbol 0$应有
    \[\frac{||f||_2}{||f||_1}\leq K\]
    而考虑$f(t)=a^t\in L^2[0,1](a\neq 1)$,则
    \[\frac{||f||_2}{||f||_1}
    =\frac{\sqrt{(a^2-1)/\ln a}}{(a-1)/\ln a}
    =\sqrt{\frac{a^2-1}{(a-1)^2}\ln a}\to\infty\]
    因此$||\cdot||_1$和$||\cdot||_2$不可能等价。

    \problem
    \begin{proof}
        对于$Z$中任意Cauchy列$\{z_n\}$,对于任意$\varepsilon>0$,
        存在$N$使得对于任意$m,n\geq N$有
        \begin{equation}
            \label{eq:Cauchy seq inequality}
            ||z_n-z_m||=||x_{1n}-x_{1m}||+||x_{2n}-x_{2m}||
            <\varepsilon
        \end{equation}
        因此$\{x_1n\},\{x_2n\}$分别为$X_1,X_2$中的Cauchy列,
        而分别由其完备性,则存在$x_1\in X_1,x_2\in X_2$使得
        $x_{1n}\to x_1,x_{2n}\to x_2$。

        从而在\cref{eq:Cauchy seq inequality}中取$m\to\infty$,
        由范数的连续性有
        \[||x_{1n}-x_1||+||x_{2n}-x_2||\leq\varepsilon\]
        从而有对于$z=(x_1,x_2)\in Z$,
        \[||z_n-z||\leq\varepsilon\]
        即$z_n\to z$。因此$Z$是Banach空间。
    \end{proof}

    \problem
    % TODO Examples of norm space where norm can not be generated inner product
    考虑$C[0,1]$,则对于$x=f(t)\equiv 1$以及$y=g(t)=t$,有
    \[\begin{aligned}
        ||x||&=||y||=1\\
        ||x+y||&=2\\
        ||x-y||&=1\\
    \end{aligned}\]
    于是
    \[2(||x||^2+||y||^2)=4\neq ||x+y||^2+||x-y||^2=5\]
    不满足平行四边形法则,故无法由内积导出。

    再考虑$L^1[0,1]$,同样地对于$x=f(t)\equiv 1$以及$y=g(t)=2t$,有
    \[\begin{aligned}
        ||x||&=||y||=1\\
        ||x+y||&=2\\
        ||x-y||&=\frac{1}{2}
    \end{aligned}\]
    于是
    \[2(||x||^2+||y||^2)=4\neq ||x+y||^2+||x-y||^2=\frac{17}{4}\]
    不满足平行四边形法则,故无法由内积导出。

    \problem
    \begin{proof}
        由于
        \[(x_n,x)\to(x,x)=||x||^2\quad(n\to\infty)\]
        故
        \[\overline{(x_n,x)}\to\overline{(x,x)}=||x||^2\]
        从而
        \[\begin{aligned}
            ||x_n-x||^2&=(x_n-x,x_n-1)\\
            &=||x_n||^2+||x||^2-\overline{(x_n,x)}-(x_n,x)
            \to 0
            \quad(n\to\infty)
        \end{aligned}\]
        即
        \[x_n\to x\quad(n\to\infty)\]
    \end{proof}

    \problem
    \begin{proof}
        必要性,考虑数域最多为实数域\sidenote{考虑到复数域的情况,
        题目疑似有误,事实上对于$\mathbb C$作为复线性空间来说,
        其为1维的,此时$(x,y)=xy$,显然不满足共轭对称的条件。
        但若定义
        \[(x,y):=\sum_{i,j}\alpha_{ij} x_i\overline y_j\]
        这里$(\alpha_{ij})$为埃尔米特矩阵(共轭对称),则满足
        内积的条件。}的情况,则由于$(\alpha_{ij})$
        为实对称矩阵
        \[\begin{aligned}
            (y,x)&=
            \sum_{i,j}\alpha_{ij}y_ix_j\\
            &=\sum_{i,j}\alpha_{ji}y_ix_j\\
            &=(x,y)
        \end{aligned}\]
        同时
        \[\begin{aligned}
            (x+y,z)&=\sum_{i,j}\alpha_{ij}(x_i+y_i)z_j\\
            &=\sum_{i,j}\alpha_{ij}x_iz_j
              +\sum_{i,j}\alpha_{ij}y_iz_j\\
            &=(x,z)+(y,z)
        \end{aligned}\]
        以及
        \[\begin{aligned}
            (kx,y)&=\sum_{i,j}\alpha_{ij}kx_iy_j\\
            &=k(x,y)
        \end{aligned}\]
        故满足对第一个元素的线性性。同时由于$(\alpha_{ij})$的正定性,
        有
        \[(x,x)=\sum_{i,j}x_ix_j\geq 0\]
        且仅在$x_1=x_2=\cdots=x_n=0$即$x=\boldsymbol 0$时取得等号,
        故$(\cdot,\cdot)$为内积。

        反之,考虑矩阵
        \[\alpha_{ij}:=(e_i,e_j)\]
        则由内积的线性性
        \[\begin{aligned}
            (x,y)&=\left(\sum_{i=1}^nx_ie_i,\sum_{j=1}^ny_je_j\right)\\
            &=\sum_{i,j}x_iy_j(e_i,e_j)\\
            &=\sum_{i,j}\alpha_{ij}x_iy_j\\
        \end{aligned}\]
        同时由内积的正定性,对于任意$x_1,x_2,\ldots,x_n\in\mathbb R$,
        \[\begin{aligned}
            \sum_{i,j}\alpha_{ij}x_ix_j
            &=(x,x)\geq 0
        \end{aligned}\]
        这里$x=\sum_{i=1}^nx_ie_i$,显然有$(\alpha_{ij})$的正定性。
    \end{proof}
\end{document}