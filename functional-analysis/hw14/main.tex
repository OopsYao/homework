\documentclass[cn]{homework}

\title{第十六周作业}

\begin{document}
    \maketitle

    \problem
    \begin{proof}
        由于$x(t)$在$[0,1]$上连续,故一致连续,于是
        对于任意$\varepsilon>0$,存在$N$使得对于任意$n>N$有
        $\left|x(t)-x(t^{1+\frac{1}{n}})\right|<\varepsilon,t\in[0,1]$,
        从而
        \[||T_n x-x||<\varepsilon\]
        即$T_n$强收敛于$T$,这里$T(x)=x$为有界线性算子。

        % TODO Not in norm
    \end{proof}
    
    \problem
    考虑到
    \[\begin{aligned}
        ||S_nT_nx-STx||&=||S_nT_nx-S_nTx+S_nTx-STx||\\
        &\leq||S_n(T_nx-Tx)||+||S_n(Tx)-S(Tx)||\\
    \end{aligned}\]
    由于$S_n$强收敛于$S$,故右半部分$||S_n(Tx)-S(Tx)||\to 0$,
    对于左半部分,
    同样由于$S_n$的强收敛,对于任意$y\in E_1$,有$S_ny\to Sy$,
    从而$||S_ny||\to ||Sy||$,从而$||S_ny||$有界,
    于是由于$E_1$2为Banach空间,从一致有界原则知,
    $||S_n||$有界,从而结合$T_n$强收敛于$T$知
    \[||S_n(T_nx-Tx)||\leq ||S_n||\cdot||T_nx-Tx||\to 0\]
    于是
    \[||S_nT_nx-STx||\to 0\]
    即$S_nT_n$强收敛于$ST$。
\end{document}