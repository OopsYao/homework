\documentclass[cn]{homework}

\title{第十六周作业}

\begin{document}
    \maketitle

    \problem
    \begin{proof}
        由于$x(t)$在$[0,1]$上连续,故一致连续,于是
        对于任意$\varepsilon>0$,存在$N$使得对于任意$n>N$有
        $\left|x(t)-x(t^{1+\frac{1}{n}})\right|<\varepsilon,t\in[0,1]$,
        从而
        \[||T_n x-x||<\varepsilon\]
        即$T_n$强收敛于$T$,这里$T(x)=x$为有界线性算子。

        % TODO Not in norm
    \end{proof}
    
    \problem
    考虑到
    \[\begin{aligned}
        ||S_nT_nx-STx||&=||S_nT_nx-S_nTx+S_nTx-STx||\\
        &\leq||S_n(T_nx-Tx)||+||S_n(Tx)-S(Tx)||\\
    \end{aligned}\]
    由于$S_n$强收敛于$S$,故右半部分$||S_n(Tx)-S(Tx)||\to 0$,
    对于左半部分,
    同样由于$S_n$的强收敛,对于任意$y\in E_1$,有$S_ny\to Sy$,
    从而$||S_ny||\to ||Sy||$,从而$||S_ny||$有界,
    于是由于$E_1$2为Banach空间,从一致有界原则知,
    $||S_n||$有界,从而结合$T_n$强收敛于$T$知
    \[||S_n(T_nx-Tx)||\leq ||S_n||\cdot||T_nx-Tx||\to 0\]
    于是
    \[||S_nT_nx-STx||\to 0\]
    即$S_nT_n$强收敛于$ST$。

    \problem
    \begin{proof}
        必要性,对于任意$G=\mathrm{span}\{x_1,x_2,\ldots,x_n\}$中的元素
        $x=\sum_{k=1}^n\beta_kx_k$来说,
        应有
        \[|f(x)|=\left|\sum_{k=1}^n\beta_kf(x_k)\right|
        =\left|\sum_{k=1}^n\alpha_k\beta_k\right|
        \leq M||x||=M\left\|\sum_{k=1}^n\beta_kx_k\right\|\]
        即证。

        充分性,考虑泛函$f_0:G\to\mathbb K$,定义为
        \[f_0:x=\sum_{k=1}^n\beta_kx_k\mapsto\sum_{k=1}^n\alpha_k\beta_k\]
        显然$f_0$为一线性泛函,则对于任意$x=\sum_{k=1}^n\beta_kx_k\in G$,
        \[|f_0(x)|=\left|\sum_{k=1}^n\beta_kf_0(x_k)\right|
        =\left|\sum_{k=1}^n\alpha_k\beta_k\right|\]
        从而由题述不等式知,
        \[|f_0(x)|\leq M\left\|\sum_{k=1}^n\beta_kx_k\right\|=M||x||\]
        从而$f_0$为有界线性泛函。于是由泛函延拓定理知,$f_0$可以延拓为全空间上的
        有界线性泛函$f$,且满足$||f||=||f_0||\leq M$,由于$f$为$f_0$的延拓,
        自然有$f(x_k)=\alpha_k,k=1,2,\ldots,n$,充分性得证。
    \end{proof}

    \problem
    \begin{proof}
        由于$f$为非零泛函,故存在$x_0\in X$使得$f(x_0)\neq 0$,
        对于任意$x\in X$,定义
        \[y=x-\frac{f(x)}{f(x_0)}x_0\]
        则$f(y)=f(x)-f(x)=0$,从而$y\in\mathscr N$,
        于是$X=\mathscr N+\{\alpha x_0\}$,
        而对于任意$\alpha x_0$,倘若$\alpha x_0\in\mathscr N$,应有
        $f(\alpha x_0)=\alpha f(x_0)=0$,从而$\alpha=0$,即\footnote{这里不是很清楚
        直和$\oplus$在一般的向量赋范空间中的含义,我们处理为被和空间的交为零元素。}
        \[\{\alpha x_0\}\cap\mathscr N=\{\boldsymbol 0\}\]
        从而$X=\mathscr N\oplus\{\alpha x_0\}$。
    \end{proof}

    \problem
    \begin{proof}
        如若不然,即$x\neq y$,考虑线性子空间$\mathrm{span}\{x-y\}$上的泛函
        \[f_0:\alpha(x-y)\mapsto\alpha||x-y||\]
        显然$f_0$为线性泛函,且有界($||f_0||=1$),
        从而由泛函延拓定理知其可以延拓为$X$上的有界线性泛函$f\in X^*$,
        于是
        \[f(x-y)=||x-y||\]
        另一方面由题设知
        \[f(x-y)=f(x)-f(y)=0\]
        从而$||x-y||=0$,于是$x=y$,矛盾。
    \end{proof}

    \problem
    \begin{proof}
        对于任意$\{\eta_k\}\in l^\infty$,可以映射到如下的一个泛函$f\in
        (l^1)^*=\mathscr B(l^1,\mathbb K)$,
        \[f:\{\xi_k\}\mapsto\sum_{k=1}^\infty\eta_k\xi_k\]
        关于$f$是$l^1$上线性泛函是显然的,其有界性只需要注意到
        \[\sup_{\{\xi_k\}\in l^1}
        \frac{\left|\sum_{k=1}^\infty\eta_k\xi_k\right|}
        {\sum_{k=1}^\infty|\xi_k|}
        \leq\sup_{\{\xi_k\}\in l^1}
        \frac{\sup_{k\geq 1}|\eta_k|\sum_{k=1}^\infty|\xi_k|}{\sum_{k=1}^\infty|\xi_k|}
        =\sup_{k\geq 1}|\eta_k|\]
        同时可以发现,对于$e_n\in l^1$,这里$e_n$表示第$n$个元素为1,其余为0的数列,
        \[f(e_n)=\eta_n\]
        从而
        \[||f||\geq\sup_{e_n,n\geq 1}|f(e_n)|=\sup_{n\geq 1}|\eta_n|\]
        于是有$||f||=||\{\eta_k\}||$,这说明$l^\infty$到$(l^1)^*$
        的映射是等距的。

        下面我们证明这个映射是一个满射,只需要考虑对于任意$f\in(l^1)^*$,
        定义$\eta_n=f(e_n)$,则由于$f$为有界线性泛函,应有
        \[\sup_{n\geq 1}|\eta_n|=\sup_{n\geq 1}||f||=||f||<\infty\]
        即$\{\eta_k\}\in l^\infty$,
        我们还需要说明$f$可以被表示为
        \[f:\{\xi_k\}\mapsto\sum_{k=1}^\infty f(e_k)\xi_k\]
        注意到\footnote{只需要注意到,当$n\to\infty$时,
        由于$\{\xi_k\}\in l^1$,
            \[\left\|\{\xi_k\}-\sum_{k=1}^n\xi_ke_k\right\|
            =\sum_{k=n}^\infty|\xi_k|\to 0\]
        },$\{\xi_k\}=\sum_{k=1}^\infty \xi_ke_k$,
        从而由$f$是有界线性的,故连续,知
        \[f(\{\xi_k\})=f(\sum_{k=1}^\infty\xi_ke_k)=\sum_{k=1}^\infty f(e_k)\xi_k\]
        这就证明了是一个满射。

        综上,$l^\infty$与$(l^1)^*$是等距同构的,故在这个意义上$(l^1)^*=l^\infty$。
    \end{proof}
\end{document}