\documentclass[cn]{homework}

\title{第十六周作业}

\begin{document}
    \maketitle

    \problem
    \begin{proof}
        由于$x(t)$在$[0,1]$上连续,故一致连续,于是
        对于任意$\varepsilon>0$,存在$N$使得对于任意$n>N$有
        $\left|x(t)-x(t^{1+\frac{1}{n}})\right|<\varepsilon,t\in[0,1]$,
        从而
        \[||T_n x-x||<\varepsilon\]
        即$T_n$强收敛于$T$,这里$T(x)=x$为有界线性算子。

        % TODO Not in norm
    \end{proof}
    
    \problem
    考虑到
    \[\begin{aligned}
        ||S_nT_nx-STx||&=||S_nT_nx-S_nTx+S_nTx-STx||\\
        &\leq||S_n(T_nx-Tx)||+||S_n(Tx)-S(Tx)||\\
    \end{aligned}\]
    由于$S_n$强收敛于$S$,故右半部分$||S_n(Tx)-S(Tx)||\to 0$,
    对于左半部分,
    同样由于$S_n$的强收敛,对于任意$y\in E_1$,有$S_ny\to Sy$,
    从而$||S_ny||\to ||Sy||$,从而$||S_ny||$有界,
    于是由于$E_1$2为Banach空间,从一致有界原则知,
    $||S_n||$有界,从而结合$T_n$强收敛于$T$知
    \[||S_n(T_nx-Tx)||\leq ||S_n||\cdot||T_nx-Tx||\to 0\]
    于是
    \[||S_nT_nx-STx||\to 0\]
    即$S_nT_n$强收敛于$ST$。

    \problem
    \begin{proof}
        必要性,对于任意$G=\mathrm{span}\{x_1,x_2,\ldots,x_n\}$中的元素
        $x=\sum_{k=1}^n\beta_kx_k$来说,
        应有
        \[|f(x)|=\left|\sum_{k=1}^n\beta_kf(x_k)\right|
        =\left|\sum_{k=1}^n\alpha_k\beta_k\right|
        \leq M||x||=M\left\|\sum_{k=1}^n\beta_kx_k\right\|\]
        即证。

        充分性,考虑泛函$f_0:G\to\mathbb K$,定义为
        \[f_0:x=\sum_{k=1}^n\beta_kx_k\mapsto\sum_{k=1}^n\alpha_k\beta_k\]
        显然$f_0$为一线性泛函,则对于任意$x=\sum_{k=1}^n\beta_kx_k\in G$,
        \[|f_0(x)|=\left|\sum_{k=1}^n\beta_kf_0(x_k)\right|
        =\left|\sum_{k=1}^n\alpha_k\beta_k\right|\]
        从而由题述不等式知,
        \[|f_0(x)|\leq M\left\|\sum_{k=1}^n\beta_kx_k\right\|=M||x||\]
        从而$f_0$为有界线性泛函。于是由泛函延拓定理知,$f_0$可以延拓为全空间上的
        有界线性泛函$f$,且满足$||f||=||f_0||\leq M$,由于$f$为$f_0$的延拓,
        自然有$f(x_k)=\alpha_k,k=1,2,\ldots,n$,充分性得证。
    \end{proof}

    \problem
    \begin{proof}
        由于$f$为非零泛函,故存在$x_0\in X$使得$f(x_0)\neq 0$,
        对于任意$x\in X$,定义
        \[y=x-\frac{f(x)}{f(x_0)}x_0\]
        则$f(y)=f(x)-f(x)=0$,从而$y\in\mathscr N$,
        于是$X=\mathscr N+\{\alpha x_0\}$,
        而对于任意$\alpha x_0$,倘若$\alpha x_0\in\mathscr N$,应有
        $f(\alpha x_0)=\alpha f(x_0)=0$,从而$\alpha=0$,即\footnote{这里不是很清楚
        直和$\oplus$在一般的向量赋范空间中的含义,我们处理为被和空间的交为零元素。}
        % TODO Direct sum in normed vector space?
        \[\{\alpha x_0\}\cap\mathscr N=\{\boldsymbol 0\}\]
        从而$X=\mathscr N\oplus\{\alpha x_0\}$。
    \end{proof}
\end{document}