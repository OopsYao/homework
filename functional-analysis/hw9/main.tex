\documentclass[cn]{homework}

\title{第十周作业}

\begin{document}
    \maketitle

    \problem
    \label{pb:scale}
    \begin{subproblem}[(\arabic*)]
        \item
        \begin{proof}
            由于该等式成立等价于等式两端平方成立,即
            \[||x+y||^2=(||x||+||y||)^2\]
            展开即得
            \[(x,y)+(y,x)=2||x||\cdot||y||\]
            于是下面的复数模的三角不等式与Cauchy不等式均取得等号
            \[|(x,y)+(y,x)|\leq|(x,y)|+|(y,x)|\leq 2||x||\cdot||y||\]
            因此应有
            % TODO So what's next?

        \end{proof}

        \item
        \begin{proof}
            % TODO Same to p1?
        \end{proof}

        \item
        \begin{proof}
            由于$x-y=x-z+z-y$,倘若$x-z,z-y$非零,即$x\neq z,y\neq z$,
            因此存在$k>0$使得
            \[x-z=k(z-y)\]
            即
            \[z=\frac{1}{1+k}x+\frac{k}{1+k}y\]
            因此存在$\alpha=1/(1+k)\in[0,1]$使得题述等式成立。
            倘若$x-z,z-y$其一为零,不妨设$x=z$,则存在$\alpha=1$符合题述,
            因此满足条件的$\alpha$总是成立的。
        \end{proof}
    \end{subproblem}

    \problem
    \begin{proof}
        特别地,对于$\lambda=0$,可以得到$||x||=||y||$,对于$\lambda=1/2$有
        \[||x+y||=2||x||=||x||+||y||\]
        因此倘若$x,y$非零,由题\ref{pb:scale}可得$y$为$x$的正倍数,同时注意到
        二者范数相同,因此只能有$x=y$。倘若其一为零向量,则二者同具有零范数,
        从而同为零向量。

        % TODO What about normed space
    \end{proof}
\end{document}