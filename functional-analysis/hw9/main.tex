\documentclass[cn]{homework}

\title{第十周作业}

\begin{document}
    \maketitle

    \problem
    \label{pb:scale}
    \begin{subproblem}[(\arabic*)]
        \item
        \begin{proof}
            由于该等式成立等价于等式两端平方成立,即
            \[||x+y||^2=(||x||+||y||)^2\]
            展开即得
            \[(x,y)+(y,x)=2||x||\cdot||y||\]
            于是下面的复数模的三角不等式与Cauchy不等式均取得等号
            \[|(x,y)+(y,x)|\leq|(x,y)|+|(y,x)|\leq 2||x||\cdot||y||\]
            因此应有
            % TODO So what's next?

        \end{proof}

        \item
        \begin{proof}
            % TODO Same to p1?
        \end{proof}

        \item
        \begin{proof}
            由于$x-y=x-z+z-y$,倘若$x-z,z-y$非零,即$x\neq z,y\neq z$,
            因此存在$k>0$使得
            \[x-z=k(z-y)\]
            即
            \[z=\frac{1}{1+k}x+\frac{k}{1+k}y\]
            因此存在$\alpha=1/(1+k)\in[0,1]$使得题述等式成立。
            倘若$x-z,z-y$其一为零,不妨设$x=z$,则存在$\alpha=1$符合题述,
            因此满足条件的$\alpha$总是成立的。
        \end{proof}
    \end{subproblem}

    \problem
    \begin{proof}
        特别地,对于$\lambda=0$,可以得到$||x||=||y||$,对于$\lambda=1/2$有
        \[||x+y||=2||x||=||x||+||y||\]
        因此倘若$x,y$非零,由题\ref{pb:scale}可得$y$为$x$的正倍数,同时注意到
        二者范数相同,因此只能有$x=y$。倘若其一为零向量,则二者同具有零范数,
        从而同为零向量。

        % TODO What about normed space
    \end{proof}

    \problem
    % TODO Equivalence of orthogonality

    \problem
    对于实空间来说,等式等价于
    \[2(x,y)=0\]
    从而知其正交。考虑复数域上的复空间$\mathbb C$,
    则内积$(x,y)=x\bar y$,诱导的范数为复数的模,考虑到
    \newcommand{\img}{\mathrm i}
    \[|1+\img|^2=|1|^2+|\img|^2=2\]
    但$(1,\img)=-\img\neq 0$,并不正交。

    \problem
    \begin{subproblem}
        \item
        \begin{proof}
        由于$M\perp N$,则对于任意$x\in M,y\in N$,有$x\perp N,y\perp M$,
        从而$x\in N^{\perp},y\in M^{\perp}$,即$M\subset N^{\perp},N\subset M^{\perp}$。
        \end{proof}

        \item
        \begin{proof}
            \newcommand{\cl}[1]{\overline{#1}}
            对于任意$x\in(\cl M)^{\perp}$,$x\perp\cl M$,而$M\subset\cl M$,
            因此$x\perp M$,即$x\in M^\perp$,故$(\cl M)^\perp\subset M^\perp$。

            反过来对于任意$x\in M^\perp$,则$x\perp M$。考虑$\cl M$中任意一点$y$,
            则存在$\{y_n\}\subset M$使得$y_n\to y$。而由于$x\perp M$,故
            $(y_n,x)=0$,则由内积的连续性
            \[(y,x)=\lim_{n\to\infty}(y_n,x)=0\]
            即有$x\in(\cl M)^\perp$,因此$M^\perp\subset(\cl M)^\perp$。
            综上可得$M^\perp=(\cl M)^\perp$。
        \end{proof}
        
    \end{subproblem}
\end{document}