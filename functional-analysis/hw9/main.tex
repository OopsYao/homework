\documentclass[cn]{homework}

\title{第十周作业}

\newcommand{\re}{\mathrm{Re}}
\newcommand{\im}{\mathrm{Im}}
\newcommand{\img}{\mathrm i}

\begin{document}
    \maketitle

    \problem
    \label{pb:scale}
    \begin{subproblem}[(\arabic*)]
        \item
        \begin{proof}
            充分性是显然的,下证必要性。
            由于该等式成立等价于等式两端平方成立,即
            \[||x+y||^2=(||x||+||y||)^2\]
            展开即得
            \[\re(x,y)=||x||\cdot||y||\neq 0\]
            于是考虑
            \[z=y-\frac{||y||^2}{\re(x,y)}x\]
            从而
            \[\begin{aligned}
               ||z||^2&=||y||^2-2\frac{||y||^2}{\re(x,y)}\re(x,y)
               +\left(\frac{||y||^2}{\re(x,y)}\right)^2||x||^2\\
               &=||y||^2\left(\frac{||x||^2||y||^2}{\left(\re(x,y)\right)^2}-1\right)\\
               &=0
            \end{aligned}\]
            因此$z=\boldsymbol 0$,即存在$k=||y||^2/\re(x,y)>0$使得$y=kx$。必要性得证。
        \end{proof}

        \item
        \begin{proof}
            由于
            \[||x-y||=\left|\|x\|-\|y\|\right|\]
            等价于
            \[||x-y||^2=\left|\|x\|-\|y\|\right|^2\]
            即
            \[\re(x,y)=||x||\cdot||y||\]
            因此由上题即证。
        \end{proof}

        \item
        \begin{proof}
            由于$x-y=x-z+z-y$,倘若$x-z,z-y$非零,即$x\neq z,y\neq z$,
            因此存在$k>0$使得
            \[x-z=k(z-y)\]
            即
            \[z=\frac{1}{1+k}x+\frac{k}{1+k}y\]
            因此存在$\alpha=1/(1+k)\in[0,1]$使得题述等式成立。
            倘若$x-z,z-y$其一为零,不妨设$x=z$,则存在$\alpha=1$符合题述,
            因此满足条件的$\alpha$总是成立的。
        \end{proof}
    \end{subproblem}

    \problem
    \begin{proof}
        特别地,对于$\lambda=0$,可以得到$||x||=||y||$,对于$\lambda=1/2$有
        \[||x+y||=2||x||=||x||+||y||\]
        因此倘若$x,y$非零,由题\ref{pb:scale}可得$y$为$x$的正倍数,同时注意到
        二者范数相同,因此只能有$x=y$。倘若其一为零向量,则二者同具有零范数,
        从而同为零向量。

        当$X$仅为赋范空间时,考虑$X=L^\infty[0,2]$,则考虑
        \[x(t)=\begin{cases}
            2,&t\in[0,1]\\
            0,&t\in(1,2]
        \end{cases}
        \quad
        y(t)=\begin{cases}
            2,&t\in[0,1]\\
            -1,&t\in(1,2]
        \end{cases}\]
        有对于任意$\lambda\in[0,1]$
        \[\lambda x(t)+(1-\lambda)y(t)=\begin{cases}
            2,&t\in[0,1]\\
            -(1-\lambda),&t\in (1,2]
        \end{cases}\]
        于是
        \[||\lambda x+(1-\lambda)y||=2=||x||\]
        但$x\neq y$。
    \end{proof}

    \problem
    \begin{proof}
        由于$||x+\alpha y||\geq ||x||$等价于
        \[||x+\alpha y||^2\geq ||x||^2\]
        即
        \begin{equation}
            \label{eq:second}
            \re(x,\alpha y)\geq -\frac{1}{2}|\alpha|^2||y||^2
        \end{equation}
        易见(1)$\Rightarrow$(2)。反过来,考虑$\alpha=t\in\mathbb R,t\neq 0$,
        则\cref{eq:second}变为
        \[t\re(x,y)\geq -\frac{1}{2}t^2||y||^2\]
        当$t>0$时,$\re(x,y)\geq -t||y||^2/2$,由$t$的任意性得$\re(x,y)\geq 0$,
        当$t<0$同样可得$\re(x,y)\leq 0$,于是$\re(x,y)=0$。

        再考虑$\alpha=\img t\neq 0,t\in\mathbb R$,只需注意到
        \[\re(x,\img ty)
        =\frac{(x,\img ty)+(\img ty,x)}{2}
        =\frac{\img t(y,x)-\img t(x,y)}{2}
        =t\im(x,y)\]
        则同样地可以得到$\im(x,y)=0$,于是$(x,y)=0$,
        从而(2)$\Rightarrow$(1),
        即(1)$\Leftrightarrow$(2)。

        而$||x+\alpha y||=||x-\alpha y||$等价于其两端平方,即得
        \[\re(x,\alpha y)=0,\forall \alpha\in\mathbb C\]
        同样地考虑$\alpha=1$与$\alpha=\img$,分别得到
        \[\re(x,y)=0,\im(x,y)=0\]
        从而$(x,y)=0$,即有(3)$\Rightarrow$(1),反过来若(1)成立,则
        对于任意$\alpha\in\mathbb C$,$(x,\alpha y)=0$,故(3)成立。

        综上,(1)(2)(3)三者等价。
    \end{proof}

    \problem
    对于实空间来说,等式等价于
    \[2(x,y)=0\]
    从而知其正交。考虑复数域上的复空间$\mathbb C$,
    则内积$(x,y)=x\bar y$,诱导的范数为复数的模,考虑到
    \[|1+\img|^2=|1|^2+|\img|^2=2\]
    但$(1,\img)=-\img\neq 0$,并不正交。

    \problem
    \begin{subproblem}
        \item
        \begin{proof}
        由于$M\perp N$,则对于任意$x\in M,y\in N$,有$x\perp N,y\perp M$,
        从而$x\in N^{\perp},y\in M^{\perp}$,即$M\subset N^{\perp},N\subset M^{\perp}$。
        \end{proof}

        \item
        \begin{proof}
            \newcommand{\cl}[1]{\overline{#1}}
            对于任意$x\in(\cl M)^{\perp}$,$x\perp\cl M$,而$M\subset\cl M$,
            因此$x\perp M$,即$x\in M^\perp$,故$(\cl M)^\perp\subset M^\perp$。

            反过来对于任意$x\in M^\perp$,则$x\perp M$。考虑$\cl M$中任意一点$y$,
            则存在$\{y_n\}\subset M$使得$y_n\to y$。而由于$x\perp M$,故
            $(y_n,x)=0$,则由内积的连续性
            \[(y,x)=\lim_{n\to\infty}(y_n,x)=0\]
            即有$x\in(\cl M)^\perp$,因此$M^\perp\subset(\cl M)^\perp$。
            综上可得$M^\perp=(\cl M)^\perp$。
        \end{proof}
        
    \end{subproblem}
\end{document}