\documentclass[cn]{homework}

\title{第八周作业}

\begin{document}
    \maketitle

    \problem
    \begin{proof}
        先证明必要性,由于$X$完备,而$S\subset X$,则对于
        $S$中的Cauchy列$\{x_n\}$,存在$x\in X$使得$x_n\to x$。
        又由于范数的连续性,
        \[||x||=\left\|\lim_{n\to\infty}x_n\right\|
        =\lim_{n\to\infty}||x_n||=1\]
        即$x\in S$,因此$S$是完备的。

        下证充分性。对于$X$中的Cauchy列$\{x_n\}$,倘若$x_n\to 0$,
        则自然收敛,倘若$x_n\not\to 0$,则存在$d>0$,使得
        某个子列$\{x_{n_k}\}$满足$||x_{n_k}||\geq d$。
        下面我们将证明$\{x_{n_k}/||x_{n_k}||\}$是$S$中的Cauchy列。

        由于Cauchy列的子列亦为Cauchy列,因此对于任意$\varepsilon>0$,
        存在$N$使得当$i,j>N$时,
        \newcommand{\norm}[1]{\Big\| #1 \Big\|}
        \newcommand{\xni}{x_{n_i}}
        \newcommand{\xnj}{x_{n_j}}
        \newcommand{\si}{\frac{\xni}{\norm\xni}}
        \newcommand{\sj}{\frac{\xnj}{\norm\xnj}}
        \[\norm{\xni-\xnj}<\varepsilon\]
        因此由范数的三角不等式,
        \begin{equation}
            \label{eq:tri inequality on s}
            \begin{aligned}
            \left\|\si-\sj\right\|&=
            \left\|\frac{\xni-\xnj}{\norm\xni}
            +\left(\frac{1}{\norm\xni}-\frac{1}{\norm\xnj}\right)\xnj\right\|\\
            &\leq\frac{\norm{\xni-\xnj}}{\norm\xni}
              +\frac{\left|\norm\xni-\norm\xnj\right|}{\norm\xni}
            \end{aligned}
        \end{equation}
        同时应该注意到同样可以由范数的三角不等式得到
        \[\left|\norm\xni-\norm\xnj\right|
        \leq\left\|\xni-\xnj\right\|\]
        因此\cref{eq:tri inequality on s}变为
        \[\left\|\si-\sj\right\|<\frac{2\varepsilon}{\norm\xni}
        \leq\frac{2\varepsilon}{d}\]
        因此$\{x_{n_k}/||x_{n_k}||\}$为$S$上的Cauchy列以及
        $\{||x_{n_k}||\}$为$\mathbb R$上的Cauchy列,故
        由于$S,\mathbb R$是完备的,分别存在$s\in S,\alpha\in\mathbb R$
        使得
        \[\begin{aligned}
            \frac{x_{n_k}}{\norm{x_{n_k}}}&\to s\\
            \norm{x_{n_k}}&\to \alpha
        \end{aligned}\]
        因此
        \[x_{n_k}\to \alpha s\in X\]
        即Cauchy列$\{x_n\}$的一个子列收敛,因此$\{x_n\}$收敛,
        故$X$是完备的,为Banach空间。
    \end{proof}

    \problem
    \begin{proof}
        显然$||\cdot||$不满足正齐次性,因此不是范数。
        而$d(x,y)=||x-y||$显然满足非负性、对称性,且
        $d(x,y)=0$等价于$x,y$在$[0,1]$上几乎处处相等,
        于是在$L^p$中可以认为是相同的,即$x=y$。

        对于任意$x,y,z\in L^p[0,1]$,
        当$x(t)=y(t),y(t)=z(t)$不同时成立时,
        \[|x-y|+|y-z|>0\]
        于是
        \[0\leq\frac{|x-y|}{|x-y|+|y-z|},\frac{|y-z|}{|x-y|+|y-z|}\leq 1\]
        注意到
        \[\alpha\leq\alpha^p,\forall\alpha\in[0,1]\]
        因此
        \begin{multline*}
            1=
            \frac{|x-y|}{|x-y|+|y-z|}+\frac{|y-z|}{|x-y|+|y-z|}\\
            \leq
            \frac{|x-y|^p}{(|x-y|+|y-z|)^p}+\frac{|y-z|^p}{(|x-y|+|y-z|)^p}
        \end{multline*}
        即
        \[(|x-y|+|y-z|)^p\leq |x-y|^p+|y-z|^p\]
        而当$x(t)=y(t)=z(t)$时上式显然成立,于是对于任意$t\in[0,1]$
        上式都会成立,故由绝对值三角不等式
        \[|x-z|^p\leq(|x-y|+|y-z|)^p\leq|x-y|^p+|y-z|^p\]
        两端同时积分即得
        \[d(x,z)\leq d(x,y)+d(y,z)\]
        因此$d$为距离。
    \end{proof}

    \problem
    \begin{proof}
        容易验证$l^\infty$是一个线性空间。
        若$||x||=0$,则由于上确界为上界,有
        \[|\xi_n|\leq 0,\forall n=1,2,\ldots\]
        因此$x=0$,反之$||\{0\}||=0$,因此正定性得证。

        对于任意$\alpha\in\mathbb R^*$,由于
        $||x||$为$\{|\xi_n|\}$的上界,
        \[|\alpha\xi_n|\leq|\alpha|\cdot||x||\]
        因此$|\alpha|\cdot||x||$为$\{|\alpha\xi_n|\}$的上界。
        对于$\varepsilon>0$,由于$||x||$为上确界,则存在$\xi_k$
        使得
        \[|\xi_k|>||x||-\frac\varepsilon{|\alpha|}\]
        于是存在$\alpha\xi_k$使得
        \[|\alpha\xi_k|>|\alpha|\cdot||x||-\varepsilon\]
        因此$|\alpha|\cdot||x||$为上确界,即
        \[||\alpha x||=|\alpha|\cdot||x||\]
        而当$\alpha=0$时上式显然成立,因此
        正齐次性满足。
        
        对于$x=\{\xi_n\},y=\{\eta_n\}$,由于
        $||x||,||y||$分别为$\{|\xi_n\},\{\eta_n\}$的上确界,于是
        \[|\xi_n+\eta_n|\leq|\xi_n|+|\eta_n|\leq||x||+||y||\]
        即$||x||+||y||$为$\{|\xi_n+\eta_n|\}$的上界,
        而$||x+y||$为其上确界,因此
        \[||x+y||\leq||x||+||y||\]
        三角不等式得证。
        因此$l^\infty$是一个赋范空间。
    \end{proof}

    \problem
    \begin{proof}
        由于连续性,我们设
        \[x(t_0)=\max_{a\leq t\leq b}|x(t)|\]
        不失一般地,不妨假设$a<x_0<b$。
        因此对于任意$0<\varepsilon\leq||x||_\infty$,
        存在$\delta>0$,使得对于任意$t\in(x_0-\delta,x_0+\delta)$,
        有$|x(t)|>||x||_\infty-\varepsilon$,于是
        \[\int_a^b|x(t)|^p\diff t
        \geq\int_{x_0-\delta}^{x_0+\delta}|x(t)|^p\diff t
        \geq\int_{x_0-\delta}^{x_0+\delta}
        \left(||x||_\infty-\varepsilon\right)^p\diff t
        =2\delta(||x||_\infty-\varepsilon)^p\]
        即
        \[||x||_p\geq(2\delta)^\frac{1}{p}(||x||_\infty-\varepsilon)\]
        
        同时由于
        \[\int_a^b|x(t)|^p\diff t\leq\int_a^b||x||_\infty^p\diff t
        =(b-a)||x||_\infty^p\]
        即
        \[||x||_p\leq(b-a)^\frac{1}{p}||x||_\infty\]
        因此有
        \[(2\delta)^\frac{1}{p}(||x||_\infty-\varepsilon)
        \leq||x||_p\leq(b-a)^\frac{1}{p}||x||_\infty\]
        依次取$p\to\infty,\varepsilon\to0^+$即得
        \[\lim_{p\to\infty}||x||_p=||x||_\infty\]
    \end{proof}

    \problem
    \begin{subproblem}[(\arabic*)]
        \item
        \begin{proof}
            由于
            \[||x||=0\Leftrightarrow\sup_{0<t\leq 1}|x(t)|=0
            \Leftrightarrow x(t)=0,0<t\leq 1\]
            正则性得证。

            而
            \[||\alpha x||=\sup_{0<t\leq 1}|\alpha x(t)|
            =|\alpha|\sup_{0<t\leq 1}|x(t)|=|\alpha|\cdot||x||\]
            因此具有正齐次性。

            同时
            \begin{multline*}
                ||x+y||=\sup_{0<t\leq 1}|x(t)+y(t)|
                \leq\sup_{0<t\leq 1}(|x(t)|+|y(t)|)\\
                \leq\sup_{0<t\leq 1}|x(t)|+\sup_{0<t\leq 1}|y(t)|
                =||x||+||y||
            \end{multline*}
            因此三角不等式得证,故$||\cdot||$为范数。
        \end{proof}
    \end{subproblem}
\end{document}