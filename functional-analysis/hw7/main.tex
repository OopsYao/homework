\documentclass[cn]{homework}

\title{第八周作业}

\begin{document}
    \maketitle

    \problem

    \problem
    \begin{proof}
        显然$||\cdot||$不满足正齐次性,因此不是范数。
        而$d(x,y)=||x-y||$显然满足非负性、对称性,且
        $d(x,y)=0$等价于$x,y$在$[0,1]$上几乎处处相等,
        于是在$L^p$中可以认为是相同的,即$x=y$。

        对于任意$x,y,z\in L^p[0,1]$,
        当$x(t)=y(t),y(t)=z(t)$不同时成立时,
        \[|x-y|+|y-z|>0\]
        于是
        \[0\leq\frac{|x-y|}{|x-y|+|y-z|},\frac{|y-z|}{|x-y|+|y-z|}\leq 1\]
        注意到
        \[\alpha\leq\alpha^p,\forall\alpha\in[0,1]\]
        因此
        \begin{multline*}
            1=
            \frac{|x-y|}{|x-y|+|y-z|}+\frac{|y-z|}{|x-y|+|y-z|}\\
            \leq
            \frac{|x-y|^p}{(|x-y|+|y-z|)^p}+\frac{|y-z|^p}{(|x-y|+|y-z|)^p}
        \end{multline*}
        即
        \[(|x-y|+|y-z|)^p\leq |x-y|^p+|y-z|^p\]
        而当$x(t)=y(t)=z(t)$时上式显然成立,于是对于任意$t\in[0,1]$
        上式都会成立,故由绝对值三角不等式
        \[|x-z|^p\leq(|x-y|+|y-z|)^p\leq|x-y|^p+|y-z|^p\]
        两端同时积分即得
        \[d(x,z)\leq d(x,y)+d(y,z)\]
        因此$d$为距离。
    \end{proof}

    \problem
    \begin{proof}
        容易验证$l^\infty$是一个线性空间。
        若$||x||=0$,则由于上确界为上界,有
        \[|\xi_n|\leq 0,\forall n=1,2,\ldots\]
        因此$x=0$,反之$||\{0\}||=0$,因此正定性得证。

        对于任意$\alpha\in\mathbb R^*$,由于
        $||x||$为$\{|\xi_n|\}$的上界,
        \[|\alpha\xi_n|\leq|\alpha|\cdot||x||\]
        因此$|\alpha|\cdot||x||$为$\{|\alpha\xi_n|\}$的上界。
        对于$\varepsilon>0$,由于$||x||$为上确界,则存在$\xi_k$
        使得
        \[|\xi_k|>||x||-\frac\varepsilon{|\alpha|}\]
        于是存在$\alpha\xi_k$使得
        \[|\alpha\xi_k|>|\alpha|\cdot||x||-\varepsilon\]
        因此$|\alpha|\cdot||x||$为上确界,即
        \[||\alpha x||=|\alpha|\cdot||x||\]
        而当$\alpha=0$时上式显然成立,因此
        正齐次性满足。
        
        对于$x=\{\xi_n\},y=\{\eta_n\}$,由于
        $||x||,||y||$分别为$\{|\xi_n\},\{\eta_n\}$的上确界,于是
        \[|\xi_n+\eta_n|\leq|\xi_n|+|\eta_n|\leq||x||+||y||\]
        即$||x||+||y||$为$\{|\xi_n+\eta_n|\}$的上界,
        而$||x+y||$为其上确界,因此
        \[||x+y||\leq||x||+||y||\]
        三角不等式得证。
        因此$l^\infty$是一个赋范空间。
    \end{proof}
\end{document}