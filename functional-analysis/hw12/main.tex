\documentclass[cn]{homework}

\title{第十三周作业}

\begin{document}
    \maketitle
    
    \problem
    \begin{subproblem}[(\arabic*)]
        \item
        \begin{proof}
            必要性,对于任意收敛点列$\{x_n\}\subset\mathcal N(f)$,
            由于$f$的连续性
            \[f(x)=f(\lim_{n\to\infty}x_n)=\lim_{n\to\infty}f(x_n)=0\]
            从而$x\in\mathcal N(f)$。显然$\mathcal N(f)$对数乘加法封闭,
            从而为闭子空间。
            
            倘若$\mathcal N(f)=X$,则$f=0$,连续性显然。否则,存在
            $z_0\in X$使得$f(z_0)\neq 0$,
            对于任意$X$中的收敛点列$y_n\to y$,定义
            \[x_n:=y_n-\frac{f(y_n)}{f(z_0)}z_0\]
            则$f(x_n)=0$,从而$x_n\in\mathcal N(f)$,
            同时
            \[\lim_{n\to\infty} x_n=y-\frac{\displaystyle\lim_{n\to\infty}f(y_n)}{f(z_0)}z_0\]
            而由于$\mathcal N(f)$为闭空间,故
            \[f\left(y-\frac{\displaystyle\lim_{n\to\infty}f(y_n)}{f(z_0)}z_0\right)
            =f(y)-\lim_{n\to\infty}f(y_n)=0\]
            从而$f$是连续的。
        \end{proof}

        \item
        \begin{proof}
            充分性,由于$f\neq 0$,故存在$x\in X$使得$f(x)\leq 0$,
            而由稠密性知存在$\{x_n\}\subset X$使得$x_n\to x$,
            于是
            \[\lim_{n\to\infty}f(x_n)=0\neq f(x)\]
            从而$f$不连续。

            反过来\sidenote{必要性的逆否命题。},
            即证明
            倘若$\mathcal N(f)$不在$X$中稠密,
            则$f$连续。
            首先我们说明任意包含$\mathcal N(f)$的子空间要么是自己$\mathcal N(f)$,
            要么是全空间$X$。倘若子空间$A\supset\mathcal N(f)$且不等,则存在
            $a\in A$使得$f(a)\neq 0$,从而对于任意$x\in X$,考虑分解
            \[x=\left(x-\frac{f(x)}{f(a)}a\right)+\frac{f(x)}{f(a)}a\]
            注意到
            \[x-\frac{f(x)}{f(a)}a\in\mathcal N(f)\]
            从而$x\in A$,从而$A=X$,也就证明了我们前面的结论。
            同时子空间的闭包显然为子空间,于是$\mathcal N(f)$的闭包要么是自己要么是
            全空间,而由于$\mathcal N(f)$不在$X$中稠密,故有$\mathcal N(f)$的闭包是自己,
            于是$\mathcal N(f)$为闭子空间,从而由前一题的结论知$f$为连续的。
        \end{proof}
    \end{subproblem}

    \problem
    \begin{subproblem}[(\arabic*)]
        \item
        \begin{proof}
            不妨设$A=(\alpha_1,\alpha_2,\ldots,\alpha_n)^\mathrm{T}$,
            这里$\alpha_i\in\mathbb R^n,i=1,2,\ldots,n$。
            则由于$|\alpha_i^\mathrm Tx|\leq||\alpha_i||_1\cdot||x||_1$,有
            \[||Ax||_1=\sum_{i=1}^n|\alpha_i^\mathrm{T}x|
            \leq\sum_{i=1}^n||\alpha_i||_1\cdot||x||_1
            =n(A)||x||_1\]
        \end{proof}

        \item
        \begin{proof}
            不妨设$A=(a_{ij}),B=(b_{ij})$,从而
            \[n(AB)=\sum_{i,j}\left|\sum_ka_{ik}b_{kj}\right|
            \leq\sum_{i,j}\sum_k|a_{ik}b_{kj}|\]
            而
            \[n(A)n(B)=\left(\sum_{i,j}|a_{ij}|\right)
            \left(\sum_{k,l}|b_{kl}|\right)
            =\sum_{i,j,k,l}|a_{ij}b_{kl}|\]
        \end{proof}
        显然有$n(AB)\leq n(A)n(B)$。
    \end{subproblem}

\end{document}