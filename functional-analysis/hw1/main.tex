\documentclass[cn]{homework}

\title{作业1}

\begin{document}
    \maketitle
    \section{题1}
    \begin{proof}
        非负性以及对称性是显然的,所以我们只需证明三角不等式的成立,
        可以考虑函数
        \[f(t)=\frac{t}{1+t},\quad t\geq 0\]
        则$\rho(x,y)=f(d(x,y))$。
        倘若我们证明了\sidenote{也就是凹性}
        \begin{align*}
            \forall u,v\geq 0\quad
            f(u)+f(v)\geq f(u+v)
        \end{align*}
        则由于原距离的三角不等式可得
        \begin{align*}
            d(x,y)+d(y,z)\geq d(x,z)
        \end{align*}
        再由$f$的单调递增性
        \begin{align*}
            f\big(d(x,y)+d(y,z)\big)\geq f\big(d(x,z)\big)
        \end{align*}
        即得
        \begin{align*}
            f\big(d(x,y)\big)+f\big(d(y,z)\big)
            \geq f\big(d(x,y)+d(y,z)\big)
            \geq f\big(d(x,z)\big)
        \end{align*}

        而由于$f$为初等函数,二阶导数存在且连续,故由凹(凸)性的
        等价定义,由$f''(t)\leq 0$,可以得到$f$是凹的。故
        \begin{fullwidth}
        \begin{align*}
            f\left(\frac{u}{u+v}(u+v)\right)+f\left(\frac{v}{u+v}(u+v)\right)
            \leq\frac{u}{u+v}f(u+v)+\frac{v}{u+v}f(u+v)
            =f(u+v)
        \end{align*}
        \end{fullwidth}
        
        因此,$\rho=f(d)$满足距离公理,因此$(X,\rho)$亦为距离空间
    \end{proof}
    \section{题2}
    \begin{subproblem}
        \item
        \begin{proof}
        对称性是显然的,由于
        \begin{align*}
            \sup d_i=0\Leftrightarrow d_i=0,\forall i=1,2,3,\dots
        \end{align*}
        故非负性亦证得。下面考虑三角不等式,记\sidenote{对于数字下标
        记法类似:$d_{1x},d_{2x},\ldots$}
        \begin{align*}
            d_x=d(y,z),\quad d_y=d(x,z),\quad d_z=d(x,y)
        \end{align*}
        则三角不等式可以表示为
        \begin{align*}
            d_x+d_y\geq d_z
        \end{align*}

        由确界的三角不等式,
        \begin{multline*}
            \sup\left\{d_{iz}-\left(d_{ix}+d_{iy}\right)+\left(d_{ix}+d_{iy}\right)\right\}\\
            \leq\sup\left\{d_{iz}-\left(d_{ix}+d_{iy}\right)\right\}
            +\sup\left\{d_{ix}+d_{iy}\right\}
        \end{multline*}
        即
        \[
            \sup\left\{d_{iz}\right\}
            \leq\sup\left\{d_{iz}-\left(d_{ix}+d_{iy}\right)\right\}
            +\sup\left\{d_{ix}+d_{iy}\right\}
        \]
        下面证明$\sup\left\{d_{iz}-\left(d_{ix}+d_{iy}\right)\right\}\leq 0$。

        由于$d_i$为距离,故
        \[d_{iz}\leq d_{ix}+d_{iy}\]
        记上确界$\alpha=\sup\left\{d_{iz}-\left(d_{ix}+d_{iy}\right)\right\}$。
        倘若$\alpha>0$,由确界定义,对于$\varepsilon=\alpha>0$,存在
        $d_{iz}-\left(d_{ix}+d_{iy}\right)>\alpha-\varepsilon=0$,矛盾。
        故$\alpha\leq 0$。因此
        \[\sup\left\{d_{iz}\right\}\leq\sup\left\{d_{ix}+d_{iy}\right\}\]
        即
        \[d_z\leq d_x+d_y\]
        三角不等式得证。故$d$为距离。
        \end{proof}

        \item
        \begin{proof}
        首先
        \[\sum_{k=1}^\infty\frac{1}{2^k}\frac{d_k}{1+d_k}
        \leq\sum_{k=1}^\infty\frac{1}{2^k}<\infty\]
        结合距离非负知,级数绝对收敛。

        由题1知$d_k/\left(1+d_k\right)$为距离,故
        \[\frac{1}{2^k}\frac{d_{kx}}{1+d_{kx}}
        +\frac{1}{2^k}\frac{d_{ky}}{1+d_{ky}}
        \leq \frac{1}{2^k}\frac{d_{kz}}{1+d_{kz}}\]
        因此
        \[
            \sum_{k=1}^\infty\frac{1}{2^k}\frac{d_{kx}}{1+d_{kx}}
            +\sum_{k=1}^\infty\frac{1}{2^k}\frac{d_{ky}}{1+d_{ky}}
            \leq\sum_{k=1}^\infty\frac{1}{2^k}\frac{d_{kz}}{1+d_{kz}}
        \]
        三角不等式得证。

        由于
        \[\sum_{k=1}^n\frac{1}{2^k}\frac{d_k}{1+d_k}
        \leq\sum_{k=1}^\infty\frac{1}{2^k}\frac{d_k}{1+d_k}
        \quad\forall n=1,2,3,\ldots\]
        故
        \[d=0\Leftrightarrow d_n=0,n=1,2,\ldots\]
        由此非负性得证,同时对称性显然,故$d$为距离。
        \end{proof}
    \end{subproblem}
\end{document}