\documentclass[cn]{homework}

\title{第六周作业}

\begin{document}
    \maketitle

    \problem
    由于$\mathbb R$是完备的,且上面定义的度量兼容$d$,
    故$(X,d)$的完备空间是$(\mathbb R,d)$。

    \problem
    其完备空间为
    $L^1[0,1]:=\left\{f\text{可测};\int_{[0,1]}|f(\xi)|\diff\xi<\infty\right\}$
    ,即$[0,1]$上勒贝格可积的可测函数构成的集合。
    \problem
    \begin{proof}
        反证法,不妨设$T$有两相异不动点$x_1\neq x_2$,满足
        \[\begin{aligned}
            Tx_1&=x_1\\
            Tx_2&=x_2
        \end{aligned}\]
        而由$T$的另一性质,
        \[d(x_1,x_2)=d(Tx_1,Tx_2)<d(x_1,x_2)\]
        产生矛盾,故不动点唯一(已知存在不动点)。
    \end{proof}

    \problem
    \begin{proof}
        我们采用数学归纳法证明,首先对于$n=1$的情形自然成立,
        倘若$n\leq m$时成立,考虑
        \[d(T^{m+1}x,T^{m+1}y)=d(T(T^m y),T(T^m y))\]
        则由于$T,T^m$为压缩映射,存在$k_1,k_m\in[0,1)$使得,
        \[\begin{aligned}
            d(T(T^m x),T(T^m y))&\leq k_1d(T^mx,T^my)\\
            d(T^mx,T^my)&\leq k_md(x,y)
        \end{aligned}\]
        于是存在$k_{m+1}=k_1k_m\in[0,1)$使得
        \[d(T^{m+1}x,T^{m+1}y)\leq k_{m+1}d(x,y)\]
        故$n=m+1$时成立,于是由数学归纳法,对于任意$n\in\mathbb N^+$,
        $T^n$为压缩映射。
    \end{proof}

    对于其逆命题,设度量空间为区间$[0,1]$,度量为实数上的度量,
    考虑映射如下定义的映射$T$,
    \[T(x)=\begin{cases}
       0,&x\in[0,\frac{1}{2}]\\ 
       \frac{1}{2},&x\in(\frac{1}{2},1]
    \end{cases}\]
    则由于
    \[T^2(x)=0,\forall x\in[0,1]\]
    于是
    \[d(T^2x,T^2y)=0,\forall x\in[0,1]\]
    $T^2$显然满足压缩映射的定义。
    而对于$T$来说\sidenote{事实上,由$T$的不连续性可直接得出
    其不为压缩映射的结论。},考虑两点$1/2,1/2+\varepsilon(\varepsilon>0)$
    \[d\left(T\left(\frac{1}{2}\right),T\left(\frac{1}{2}+\varepsilon\right)\right)
    =d\left(0,\frac{1}{2}\right)
    =\frac{1}{2}\]
    于是任意$0<\varepsilon<\frac{1}{2}$,均能说明$T$不为压缩映射。
    因此逆命题不一定成立。

    \problem
    \begin{proof}
        考虑完备空间$C[0,1]$与其上的映射
        \[T:x(t)\mapsto\frac{1}{2}\sin x(t)-a(t)\]
        对于任意$x,y\in C[0,1]$,有
        \[\begin{aligned}
            \left|\frac{\sin x}{2}-\frac{\sin y}{2}\right|
            =\left|\sin\frac{x-y}{2}\cos\frac{x+y}{2}\right|
            \leq\frac{|x-y|}{2}
        \end{aligned}\]
        于是
        \[\begin{aligned}
        d(Tx,Ty)&=\max_{t\in[0,1]}\left|\frac{\sin x(t)}{2}-\frac{\sin y(t)}{2}\right|\\
        &\leq\frac{1}{2}\max_{t\in[0,1]}|x(t)-y(t)|\\
        &=\frac{1}{2}d(x,y)
        \end{aligned}\]
        即$T$为完备空间$C[0,1]$上的一个压缩映射,故由压缩映射原理,
        存在$x(t)\in C[0,1]$,即$[0,1]$上的连续函数,是$T$的不动点,使得
        \[x=Tx\]
        即
        \[x(t)=\frac{1}{2}\sin x(t)-a(t)\]
    \end{proof}

    \problem
    \begin{proof}
        考虑实数坐标空间$\mathbb R^n$,其上有度量为欧式距离
        \[d(x,y)=\sqrt{\sum_{i=1}^n(x_i-y_i)^2}\]
        由实数完备性知$\mathbb R^n$是完备的。
        为记号的简便性,记度量诱导的范数
        \[||x||:=d(x,0)\]
        于是由度量的三角不等式导出范数的三角不等式
        \[||x+y||=d(x+y,0)=d(x,-y)\leq d(x,0)+d(0,-y)=||x||+||y||\]

        现考虑$\mathbb R^n$上的映射
        \[T:x\mapsto b-(A-E)x\]
        并记$B=A-E=(\beta_1,\beta_2,\cdots,\beta_n)$,这里$\beta_i\in\mathbb R^n$
        为$n\times 1$的列向量。
        于是对于任意$y,z\in\mathbb R^n$,记$x=y-z=(x_1,x_2,\ldots,x_n)^\mathrm T$,
        由三角不等式
        \[d(Ty,Tz)=||Bx||=\left\|\sum_{i=1}^nx_i\beta_i\right\|
        \leq\sum_{i=1}^n|x_i|\cdot||\beta_i||\]
        又由Cauchy不等式
        \[\sum_{i=1}^n|x_i|\cdot||\beta_i||
        \leq\sqrt{\left(\sum_{i=1}^nx_i^2\right)\left(\sum_{i=1}^n||\beta_i||^2\right)}\]
        于是若记
        \[||B||_F:=\sqrt{\sum_{i=1}^n||\beta_i||^2}\]
        则我们有
        \[d(Ty,Tz)\leq ||B||_F\cdot d(y,z)\]

        又由于
        \[\sum_{i=1}^n||\beta_i||^2=\sum_{i,j}b_{ij}^2
        =\sum_{i,j}(a_{ij}-\delta_{ij})^2<1\]
        于是
        \[||B||_F<1\]
        因此$T$为一个压缩映射,而$\mathbb R^n$为完备的,因此存在
        唯一不动点$x\in\mathbb R^n$使得
        \[x=Tx\]
        即
        \[Ax=b\]
    \end{proof}

    \problem
    \begin{proof}
        记$C[0,1]$的度量为$d$,并在该空间上定义新的度量
        \[d'(x,y)=\max_{t\in[0,1]}\left|\e^{-t}\big(x(t)-y(t)\big)\right|\]
        显然$d'$是满足对称性、非负性(不可区分同一性),而
        由绝对值不等式
        \[|\e^{-t}\big(x(t)-z(t))|\leq
        |\e^{-t}\big(x(t)-y(t)\big)|+|\e^{-t}\big(y(t)-z(t)\big)|\]
        可知$d'$是满足三角不等式的,即$d'$是一个度量。

        同时又由于$t\in[0,1]$,有
        \[d'(x,y)\leq d(x,y)\]
        于是原来$d$下的Cauchy列也是$d'$下的Cauchy列\sidenote{该性质我们
        也在第四周作业的最后一题中证明过了。},而原度量$d$下空间是
        完备的,因此新的度量$d'$下空间亦是完备的。

        故考虑这个新的度量空间(记为$(X,d')$)上的映射
        \[T:x(t)\mapsto y(t)+\lambda\int_0^1\e^{t-s}x(s)\diff s\]
        于是对于任意$x_1,x_2\in X$,
        \[\begin{aligned}
            d'(Tx_1,Tx_2)&=|\lambda|\max_{t\in[0,1]}
            \left|\e^{-t}\int_0^1\e^{t-s}(x_1(s)-x_2(s))\diff s\right|\\
            &=|\lambda|\cdot
            \left|\int_0^1\e^{-s}\big(x_1(s)-x_2(s)\big)\diff s\right|\\
            &\leq|\lambda|\cdot
            \int_0^1\left|\e^{-s}\big(x_1(s)-x_2(s)\big)\right|\diff s\\
            &\leq|\lambda|\cdot\int_0^1d'(x_1,x_2)\diff s\\
            &=|\lambda|\cdot d'(x_1,x_2)
        \end{aligned}\]
        由于$|\lambda|<1$,因此$T$是一个压缩映射,故由压缩映射原理知,
        存在唯一$x\in X$使得其为不动点,即存在唯一解$x\in C[0,1]$,满足
        积分方程
        \[x(t)-\lambda\int_0^1\e^{t-s}x(s)\diff s=y(t)\]
    \end{proof}
\end{document}