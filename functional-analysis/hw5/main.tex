\documentclass[cn]{homework}

\title{第六周作业}

\begin{document}
    \maketitle

    \problem
    由于$\mathbb R$是完备的,且上面定义的度量兼容$d$,
    故$(X,d)$的完备空间是$(\mathbb R,d)$。

    \problem
    其完备空间为
    $L^1[0,1]:=\left\{f\text{可测};\int_{[0,1]}|f(\xi)|\diff\xi<\infty\right\}$
    ,即$[0,1]$上勒贝格可积的可测函数构成的集合。
    \problem
    \begin{proof}
        反证法,不妨设$T$有两相异不动点$x_1\neq x_2$,满足
        \[\begin{aligned}
            Tx_1&=x_1\\
            Tx_2&=x_2
        \end{aligned}\]
        而由$T$的另一性质,
        \[d(x_1,x_2)=d(Tx_1,Tx_2)<d(x_1,x_2)\]
        产生矛盾,故不动点唯一(已知存在不动点)。
    \end{proof}

    \problem
    \begin{proof}
        我们采用数学归纳法证明,首先对于$n=1$的情形自然成立,
        倘若$n\leq m$时成立,考虑
        \[d(T^{m+1}x,T^{m+1}y)=d(T(T^m y),T(T^m y))\]
        则由于$T,T^m$为压缩映射,存在$k_1,k_m\in[0,1)$使得,
        \[\begin{aligned}
            d(T(T^m x),T(T^m y))&\leq k_1d(T^mx,T^my)\\
            d(T^mx,T^my)&\leq k_md(x,y)
        \end{aligned}\]
        于是存在$k_{m+1}=k_1k_m\in[0,1)$使得
        \[d(T^{m+1}x,T^{m+1}y)\leq k_{m+1}d(x,y)\]
        故$n=m+1$时成立,于是由数学归纳法,对于任意$n\in\mathbb N^+$,
        $T^n$为压缩映射。
    \end{proof}

    对于其逆命题,设度量空间为区间$[0,1]$,度量为实数上的度量,
    考虑映射如下定义的映射$T$,
    \[T(x)=\begin{cases}
       0,&x\in[0,\frac{1}{2}]\\ 
       \frac{1}{2},&x\in(\frac{1}{2},1]
    \end{cases}\]
    则由于
    \[T^2(x)=0,\forall x\in[0,1]\]
    于是
    \[d(T^2x,T^2y)=0,\forall x\in[0,1]\]
    $T^2$显然满足压缩映射的定义。
    而对于$T$来说\sidenote{事实上,由$T$的不连续性可直接得出
    其不为压缩映射的结论。},考虑两点$1/2,1/2+\varepsilon(\varepsilon>0)$
    \[d\left(T\left(\frac{1}{2}\right),T\left(\frac{1}{2}+\varepsilon\right)\right)
    =d\left(0,\frac{1}{2}\right)
    =\frac{1}{2}\]
    于是任意$0<\varepsilon<\frac{1}{2}$,均能说明$T$不为压缩映射。
    因此逆命题不一定成立。

    \problem
    \begin{proof}
        考虑完备空间$C[0,1]$与其上的映射
        \[T:x(t)\mapsto\frac{1}{2}\sin x(t)-a(t)\]
        对于任意$x,y\in C[0,1]$,有
        \[\begin{aligned}
            \left|\frac{\sin x}{2}-\frac{\sin y}{2}\right|
            =\left|\sin\frac{x-y}{2}\cos\frac{x+y}{2}\right|
            \leq\frac{|x-y|}{2}
        \end{aligned}\]
        于是
        \[\begin{aligned}
        d(Tx,Ty)&=\max_{t\in[0,1]}\left|\frac{\sin x(t)}{2}-\frac{\sin y(t)}{2}\right|\\
        &\leq\frac{1}{2}\max_{t\in[0,1]}|x(t)-y(t)|\\
        &=\frac{1}{2}d(x,y)
        \end{aligned}\]
        即$T$为完备空间$C[0,1]$上的一个压缩映射,故由压缩映射原理,
        存在$x(t)\in C[0,1]$,即$[0,1]$上的连续函数,是$T$的不动点,使得
        \[x=Tx\]
        即
        \[x(t)=\frac{1}{2}\sin x(t)-a(t)\]
    \end{proof}
\end{document}