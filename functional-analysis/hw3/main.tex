\documentclass[cn]{homework}

\title{第四周作业}

\begin{document}
    \maketitle

    \problem
    \begin{proof}
        设距离空间$(X,d)$可分,则存在可数子集$A\subset X$
        在$X$中稠密,故对于任意子空间$(X',d)$,有
        \[X'\subset X\subset\bar A\]
        故$A$在子空间中亦稠密,因此子空间可分。
    \end{proof}

    倘若距离空间不可分,其子空间亦可能是可分的。最平凡的例子
    是其可数或有限子空间,显然可数或有限空间都是可分的,
    自己在自己中稠密。

    \problem
    设$f:A\to\mathbb R$为连续函数,我们先证明$f$在$A$上
    是有界的。
    \begin{proof}
        反证法。倘若$f$无界,则对于任意$n$,存在$x_n\in A$ s.t.
        $|f(x_n)|>n$。又$A$为自列紧集,故无穷序列$\{x_n\}$存在
        收敛子列$\left\{x_{n_k}\right\}$满足
        \[\lim_{k\to\infty}x_{n_k}=x\in A\]

        考虑序列$\left\{|f(x_{n_k})|\right\}$,由于$|f|$连续,则
        \[\lim_{k\to\infty}|f(x_{n_k})|=|f(x)|\in\mathbb R\]
        这显然与
        \[|f(n_k)|>n_k\geq k\]
        所导致的不收敛相矛盾。因此$f$在$A$上是有界的。
    \end{proof}

    由实数基本定理,$f(A)$存在上下确界,
    下证确界值是可以取得的。
    \begin{proof}
        不妨设其上确界
        为$\alpha\in\mathbb R$,则由定义,对于$\varepsilon=1/n$,
        存在$x'_n\in A$满足
        \[f(x'_n)>\alpha-\frac{1}{n}\]
        又由$A$的自列紧性,存在子列$x'_{n_k}\to x'\in A$,因此
        由极限保号性
        \[\lim_{k\to\infty}f(x'_{n_k})\geq \alpha\]
        而上确界亦是上界,因此
        \[\lim_{k\to\infty}f(x'_{n_k})=\alpha\]
        再由连续性
        \[f(x')=\alpha\]
        即$f$在$x'$处取得上确界。同理可证得下确界。
    \end{proof}
\end{document}