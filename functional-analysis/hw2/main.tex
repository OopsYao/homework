\documentclass[cn]{homework}

\title{第三周作业}

\begin{document}
    \maketitle

    \problem
    \begin{proof}
        非负性与对称性是显然的。由绝对值不等式,
        \[|x_i-y_i|+|y_i-z_i|\leq|x_i-y_i+y_i-z_i|=|x_i-z_i|\]
        因此
        \[\sum_{i=1}^n\lambda_i|x_i-y_i|
        +\sum_{i=1}^n\lambda_i|y_i-z_i|
        \leq\sum_{i=1}^n\lambda_i|x_i-z_i|\]
        即
        \[d(x,y)+d(y,z)\leq d(x,z)\]
        三角不等式得证。

        下证距离收敛和坐标收敛的等价性。
        由于
        \[\forall i=1,2,\ldots,n\quad |x_{ik}-x|\to 0\Rightarrow
        \sum_{i=1}^n\lambda_i|x_{ik}-x_i|\to 0\]
        故依坐标收敛蕴含依距离收敛。

        反过来,由于
        \[\lambda_i|x_{ik}-x_i|\leq\sum_{i=1}^n\lambda_i|x_{ik}-x_i|
        =d(x_k,x)\]
        故
        \[d(x_k,x)\to 0\Rightarrow \lambda_i|x_{ik}-x_i|\to 0
        \Rightarrow |x_{ik}-x_i|\to 0\]
        即以距离收敛蕴含依坐标收敛,等价性得证。
    \end{proof}    
    
    \problem
    \begin{proof}
        \newcommand{\infy}{\inf_{y\in A}}
        \newcommand{\supy}{\sup_{y\in A}}
        由三角不等式
        \[d(x_n,y)\leq d(x_n,x)+d(x,y)\]
        则由下确界定义知
        \[\infy d(x_n,y)\leq d(x_n,x)+d(x,y)\]
        注意到不等式右端$y$的任意性,则
        \[\infy d(x_n,y)\leq d(x_n,x)+\infy d(x,y)\]
        由对称性可得另一不等式
        \[\infy d(x,y)\leq d(x,x_n)+\infy d(x_n,y)\]
        故
        \[\left|\infy d(x_n,y)-\infy d(x,y)\right|
        \leq d(x_n,x)\]

        而$d$是距离,则
        \[x_n\to x\Rightarrow d(x_n,x)\to 0
        \Rightarrow |f(x_n)-f(x)|\to 0\]
        即$f$是连续的。
    \end{proof}

    \problem
    \begin{proof}
        设有距离空间$(X,d_X)$与$(Y,d_Y)$,$f:X\to Y$是其间的
        等距映射。则由等距映射定义,
        $\forall\varepsilon>0$,$\exists\delta=\varepsilon$,
        使得$\forall x'\in X$满足$d_X(x',x)<\delta$有$d_Y(f(x'),f(x))
        =d_X(x',x)<\delta=\varepsilon$。即$f$是连续映射。

        又$\forall x_1,x_2\in X$若满足
        \[f(x_1)=f(x_2)\]
        则由距离公理
        \[d_Y(f(x_1),f(x_2))=0\]
        再由等距映射定义
        \[d_X(x_1,x_2)=0\]
        再由距离公理
        \[x_1=x_2\]
        因此,$f$是单射。
    \end{proof}
\end{document}