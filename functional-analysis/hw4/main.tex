\documentclass[cn]{homework}

\title{第五周作业}

\begin{document}
    \maketitle

    \problem
    \begin{proof}
        对于点列$\{x_n=n\}_{n=1}^\infty$来说,由于
        存在$N\geq 2/\varepsilon$使得
        \[d(x_n,x_m)=\left|\frac{1}{n}-\frac{1}{m}\right|
        \leq\frac{2}{\min\{m,n\}}
        \leq\frac{2}{N}
        \leq\varepsilon\]
        因此$\{x_n\}_{n=1}^\infty$为Cauchy列,
        而对于任意$k\in X$,总有$\varepsilon_k=1/k$,使得
        \[d(x_n,k)=\left|\frac{1}{n}-\frac{1}{k}\right|
        \to\left|\frac{1}{k}\right|\geq\varepsilon_k
        \quad(n\to\infty)\]
        即序列不会收敛于$X$中的任意元素,因此$X$不是完备的。
    \end{proof}

    \problem
    \begin{subproblem}
        \item 
        \begin{proof}
            \newcommand{\yz}{|y(t)-z(t)|}
            \newcommand{\yzp}{|y'(t)-z'(t)|}
            \newcommand{\xz}{|x(t)-z(t)|}
            \newcommand{\xzp}{|x'(t)-z'(t)|}
            \newcommand{\xy}{|x(t)-y(t)|}
            \newcommand{\xyp}{|x'(t)-y'(t)|}
            由绝对值的三角不等式,对于任意$x,y,z\in D$,
            $t\in[0,1]$,
            \begin{align*}
                \xz &\leq\xy+\yz\\
                \xzp &\leq\xyp+\yzp\\
            \end{align*}
            因此,
            \begin{align*}
                \sup_{0\leq t\leq 1}\xz
                &\leq\sup_{0\leq t\leq 1}\xy
                +\sup_{0\leq t\leq 1}\xz\\
                \sup_{0\leq t\leq 1}\xzp
                &\leq\sup_{0\leq t\leq 1}\xyp
                +\sup_{0\leq t\leq 1}\xzp\\
            \end{align*}
            故
            \[d(x,z)\leq d(x,y)+d(y,z)\]
            三角不等式得证,而对称性与非负性是显然的,对于不可区分同一性,
            倘若
            \[d(x,y)=0\]
            则
            \[\sup_{0\leq t\leq 1}\xy=0\]
            因此对于任意$t\in[0,1]$,有
            \[x(t)=y(t)\]
            即$x=y$。
            综上,$d$为距离,$D$为距离空间。
        \end{proof}

        \item
        \newcommand{\xnx}{|x_n(t)-x(t)|}
        \newcommand{\xnxp}{|x_n'(t)-x'(t)|}
        不妨设$x_n\to x(n\to\infty)$,按距离收敛意味着
        \[\sup_{0\leq t\leq 1}\xnx+\sup_{0\leq t\leq 1}\xnxp\to 0\]
        由两者非负性,等价于
        \begin{align*}
            \sup_{0\leq t\leq 1}\xnx&\to 0\\
            \sup_{0\leq t\leq 1}\xnxp&\to 0\\
        \end{align*}
        亦等价于,对于任意$t\in[0,1]$,
        \begin{align*}
            \xnx&\to 0\\
            \xnxp&\to 0\\
        \end{align*}
        即$x_n(t),x_n'(t)$(在函数列的层面上)一致收敛于
        $x(t),x'(t)$。

        \item
        \begin{proof}
            \newcommand{\xmn}{|x_m(t)-x_n(t)|}
            \newcommand{\xmnp}{|x'_m(t)-x_n'(t)|}
            不妨设$\{x_n\}_{n=1}^\infty$为$D$中的Cauchy列,
            则对于任意$\varepsilon>0$,存在$N$,对于任意$m,n>N$有,
            \[d(x_m,x_n)<\varepsilon\]
            故
            \begin{align*}
                \sup_{0\leq t\leq 1}\xmn&<\varepsilon\\
                \sup_{0\leq t\leq 1}\xmnp&<\varepsilon\\
            \end{align*}
            注意到$D\subset C[0,1]$,
            因此$\{x_n\}_{n=1}^\infty$与$\{x'_n\}_{n=1}^\infty$
            为$C[0,1]$中的Cauchy列,
            利用$C[0,1]$的完备性知,存在$x,y\in C[0,1]$有
            \begin{align*}
                \sup_{0\leq t\leq 1}\xnx&\to 0\\
                \sup_{0\leq t\leq 1}|x'_n(t)-y(t)|&\to 0
            \end{align*}
            因此$x_n(t)$与$x'_n(t)$分别一致收敛于$x(t)$与$y(t)$。

            下面只要证明$x\in D$,则有$d(x_n,x)\to 0$,完备性可得证。
            由于
            \[\lim_{n\to\infty}\int_0^tx'_n(\xi)\diff\xi
            =\lim_{n\to\infty}x_n(t)-x_n(0)=x(t)-x(0)\]
            而另一方面由一致连续性
            \[\lim_{n\to\infty}\int_0^tx'_n(\xi)\diff\xi
            =\int_0^t\lim_{n\to\infty}x'_n(\xi)\diff\xi
            =\int_0^ty(\xi)\diff\xi\]
            因此
            \[x(t)=x(0)+\int_0^ty(\xi)\diff\xi\]
            故由含参变量积分的可微性知
            \[x'(t)=y(t)\in C[0,1]\]
            因此$x\in D$,而
            \[d(x_n,x)=\sup_{0\leq t\leq 1}\xnx
            +\sup_{0\leq t\leq 1}|x_n'(t)-x(t)|
            \to 0\]
            故$D$是完备的。
        \end{proof}
    \end{subproblem}

    \problem
    \begin{proof}
        考虑$C[0,1]$,度量与题中定义相同(显然该度量在新的空间中
        也是度量),故新的度量空间(记为$M$)包含原来的(记为$P$)。
        考虑
        \[p_n(x)=\sum_{k=0}^n\frac{x^k}{k!}\]
        则对于任意$m>n$,注意到$x\in[0,1]$,
        \[d(p_m,p_n)=\int_0^1\left|\sum_{k=n+1}^m\frac{x^k}{k!}\right|\diff x
        =\sum_{k=n+1}^m\frac{1}{(k+1)!}\]
        考虑到
        \[\sum_{k=0}\frac{1}{k!}=\e<\infty\]
        因此
        \[\sum_{k=n+1}^m\frac{1}{(k+1)!}\to 0\quad(n\to\infty,m>n)\]
        于是$\{p_n\}$是$P$中的Cauchy列。

        而对于$\e^x\in M$,由于闭区间上(一致)连续性,
        \[\begin{aligned}
            \lim_{n\to\infty}d(p_n,\e^x)
            &=\lim_{n\to\infty}\int_0^1\left|\e^x-\sum_{k=0}^n\frac{x^k}{k!}\right|\diff x\\
            &=\int_0^1\left|\e^x-\lim_{n\to\infty}\sum_{k=0}^n\frac{x^k}{k!}\right|\diff x\\
            &=0
        \end{aligned}\]
        由度量空间极限的唯一性,$\e^x\in M$是该Cauchy列的唯一极限,
        而$\e^x\not\in P$,故$P$不是完备的。
    \end{proof}
\end{document}