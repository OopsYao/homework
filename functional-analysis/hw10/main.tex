\documentclass[cn]{homework}

\title{第十一周作业}

\begin{document}
    \maketitle

    \problem
    \begin{proof}
        显然$M$对数乘与加法封闭,从而$M$是一个子空间。考虑$M$中任意
        收敛点列$x_n=\{x_n^{(k)}\}_{k=1}^\infty,n=1,2,\ldots$,
        不妨设其收敛到$x=\{x^{(k)}\}_{k=1}^\infty\in l^2$。则由于
        对于任意$k$,
        \[\left(x^{(2k)}\right)^2=\left(x_n^{(2k)}-x^{(2k)}\right)^2
        \leq\sum_{k=1}^\infty \left(x_n^{(k)}-x^{(k)}\right)^2
        =||x_n-x||^2\]
        取$n\to\infty$即得$x^{(2k)}=0$,从而知$x\in M$,故$M$为闭子空间。
    \end{proof}

    对于任意$y\in M^\perp$,考虑如下构造的$x$,满足
    \[x=\{x_n|x_{2n-1}=y_{2n-1},x_{2n}=0,n=1,2,\ldots\}\]
    从而知道$x\in M$,于是应有$x\perp y$,即
    \[(x,y)=\sum_{n=1}^\infty x_ny_n=\sum_{n=1}^\infty y_{2n-1}^2=0\]
    因此$y_{2n-1}=0,n=1,2,\ldots$,反过来对于满足此条件的$y$显然有
    $y\perp M$,于是
    \[M^\perp=\{y|y=\{y_n\}\in l^2,y_{2n-1}=0,n=1,2,\ldots\}\]

    \problem
    % TODO Go to check whether the Hilbert space make sense
    \begin{proof}
        考虑到$X,Y$为子空间,则$\boldsymbol 0\in X,Y$,
        从而有$X,Y\subset X+Y$,因此对于任意$z_0\perp X+Y,z_0\in H$,
        有$z_0\perp X,z_0\perp Y$,
        即$(X+Y)^\perp\subset X^\perp\cap Y^\perp$。

        反过来,对于任意$z_0\in H$满足$z_0\perp X,z_0\perp Y$,
        则对于任意$z=x+y\in (X+Y)$,
        \[(z_0,z)=(z_0,x)+(z_0,y)=0\]
        即$z_0\perp (X+Y)$。从而$X^\perp\cap Y^\perp\subset (X+Y)^\perp$,
        综上有$(X+Y)^\perp=X^\perp\cap Y^\perp$。
    \end{proof}

    \problem
    \begin{proof}
        对于充分性,考虑
        $L$中的收敛点列$z_n=x_n+y_n\to z$,这里$x_n\in M,y_n\in N$,
        由勾股定理,
        \[||z_n-z_m||^2=||x_n-x_m+y_n-y_m||^2=||x_n-x_m||^2+||y_n-y_m||^2\]
        从而$\{z_n\}$为Cauchy列蕴含$\{x_n\},\{y_n\}$为Cauchy列,
        则由$H$为Hilbert空间以及$M,N$的闭性,分别存在$x\in M,y\in N$使得
        $x_n\to x,y_n\to y$,从而$z=x+y\in L$,即$L$为闭子空间。

        % TODO Prove => use orthogonality
        
    \end{proof}

    \problem
    \newcommand{\cg}{\overline}
    对于$M_1$来说,考虑任意$g\in M_1^\perp$,则
    对于
    \[f(x)=\begin{cases}
        0,&x\leq 0\\
        |x|\cdot g(x),&x>0
    \end{cases}\]
    由于$f\in M_1$,于是应有
    \[(f,g)=\int_0^1|x|\cdot|g(x)|^2\diff x=0\]
    从而知$|x|\cdot g(x)=0$在$[0,1]$上几乎处处为$0$,
    进一步由连续性知$g$在$[0,1]$上恒为$0$。
    反过来对于任意满足此条件的$g(x)$,
    也容易验证$g\perp M_1$。从而
    \[M_1^\perp=\{g\in X|g(x)=0,\forall x\geq 0\}\]

    对于任意$g\in M_2^\perp$,则由于
    \[f(x)=|x|\cdot g(x)\in M_2\]
    于是
    \[(f,g)=\int_{-1}^1|x|\cdot|g(x)|^2\diff x=0\]
    从而$|x|\cdot |g(x)|^2=0$在$[-1,1]$上几乎处处为$0$,
    进一步由
    $g$的连续性知$g(x)=0,\forall x\in[-1,1]$。于是反过来
    对于任意$f\in M_2$有$(f,g)=0$。从而
    \[M_2^\perp=\{g\in X|g(x)=0,\forall x\in[-1,1]\}\]

\end{document}