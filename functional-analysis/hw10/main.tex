\documentclass[cn]{homework}

\title{第十一周作业}

\begin{document}
    \maketitle

    \problem
    % TODO M is closed

    \problem
    % TODO Go to check whether the Hilbert space make sense
    \begin{proof}
        考虑到$X,Y$为子空间,则$\boldsymbol 0\in X,Y$,
        从而有$X,Y\subset X+Y$,因此对于任意$z_0\perp X+Y,z_0\in H$,
        有$z_0\perp X,z_0\perp Y$,
        即$(X+Y)^\perp\subset X^\perp\cap Y^\perp$。

        反过来,对于任意$z_0\in H$满足$z_0\perp X,z_0\perp Y$,
        则对于任意$z=x+y\in (X+Y)$,
        \[(z_0,z)=(z_0,x)+(z_0,y)=0\]
        即$z_0\perp (X+Y)$。从而$X^\perp\cap Y^\perp\subset (X+Y)^\perp$,
        综上有$(X+Y)^\perp=X^\perp\cap Y^\perp$。
    \end{proof}

    \problem
    % TODO L closed <=> M, N closed

    \problem
    \newcommand{\cg}{\overline}
    对于$M_1$来说,考虑任意$g\in M_1^\perp$,则
    对于
    \[f(x)=\begin{cases}
        0,&x\leq 0\\
        |x|\cdot g(x),&x>0
    \end{cases}\]
    由于$f\in M_1$,于是应有
    \[(f,g)=\int_0^1|x|\cdot|g(x)|^2\diff x=0\]
    从而知$|x|\cdot g(x)=0$在$[0,1]$上几乎处处为$0$,
    进一步由连续性知$g$在$[0,1]$上恒为$0$。
    反过来对于任意满足此条件的$g(x)$,
    也容易验证$g\perp M_1$。从而
    \[M_1^\perp=\{g\in X|g(x)=0,\forall x\geq 0\}\]

    对于任意$g\in M_2^\perp$,则由于
    \[f(x)=|x|\cdot g(x)\in M_2\]
    于是
    \[(f,g)=\int_{-1}^1|x|\cdot|g(x)|^2\diff x=0\]
    从而$|x|\cdot |g(x)|^2=0$在$[-1,1]$上几乎处处为$0$,
    进一步由
    $g$的连续性知$g(x)=0,\forall x\in[-1,1]$。于是反过来
    对于任意$f\in M_2$有$(f,g)=0$。从而
    \[M_2^\perp=\{g\in X|g(x)=0,\forall x\in[-1,1]\}\]

\end{document}