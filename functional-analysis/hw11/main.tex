\documentclass[cn]{homework}

\title{第十二周作业}

\newcommand{\img}{\mathrm i}

\begin{document}
    \maketitle

    \problem
    \begin{subproblem}[(\arabic*)]
        \item
        \emph{下面只证明了$b-a=1$的情况,$b-a<1$不知道怎么证。}
        \begin{proof}
            由Stone-Weierstrass定理知
            $\{\e^{2\pi\img nx}\}$是$L^2[-1/2,1/2]$的正交基,
            对于$b-a=1$的情况,
            考虑从$L^2[-1/2,1/2]$到$L^2[a,a+1]$的平移变换,
            \[T:f(x)\mapsto f(x-a-\frac{1}{2})\]
            则容易验证,在两个空间中
            \[(Tf,Tg)=(f,g)\]
            这里内积分别为相应空间的内积。
            从而$L^2[-1/2,1/2]$的正交基在$L^2[a,a+1]$中的象为象空间
            的基,即
            \[T\e^{2\pi\img nx}=\e^{2\pi\img n(x-a-1/2)}
            =\e^{-2n(a+1/2)\pi\img}\e^{2\pi\img nx},n\in\mathbb Z\]
            为$L^2[a,a+1]$的正交基,注意到我们将系数$\e^{-2n(a+1/2)\pi\img}$
            去除后可知,$\{\e^{2\pi\img nx}\}$亦为正交基,从而知其完全,
            即$S^\perp=\{\boldsymbol 0\}$。
        \end{proof}

        \item
        \emph{下面只验证了$b-a=2$的情况。}
        \begin{proof}
            考虑$\e^{\pi\img x}\in L^2[-1,1]$,则对于任意$n\in\mathbb Z$,
            \[\begin{aligned}
                (\e^{\pi\img x},\e^{2\pi\img nx})
                &=\int_{-1}^1\e^{\pi\img(2n-1)x}\diff x\\
                &=\int_{-1}^1\cos(2n-1)\pi x\diff x+\img\int_{-1}^1\sin(2n-1)\pi x\diff x\\
                &=0
            \end{aligned}\]
            于是$\e^{\pi\img x}\in S^\perp$,但$\e^{\pi\img x}\neq\boldsymbol 0$。
        \end{proof}
    \end{subproblem}

    \problem
    \begin{proof}
        倘若存在$\alpha_1^*,\alpha_2^*$使得,$x^*=\alpha_1^*x_1+\alpha_2^*x_2-x_3$
        与$x_1,x_2$正交,则
        \[\begin{aligned}
            (f_(\alpha_1,\alpha_2))^2
            &=||(\alpha_1-\alpha_1^*)x_1+(\alpha_2-\alpha_2^*)x_2+x^*||^2\\
            &=||(\alpha_1-\alpha_1^*)x_1+(\alpha_2-\alpha_2^*)x_2||^2
            +||x^*||^2
        \end{aligned}\]
        从而$f$在$\alpha_i=\alpha_i^*,i=1,2$时取得最小值。

        由于$x_1,x_2$为单位正交列,从而$x_3-(x_3,x_1)x_1-(x_3,x_2)x_2$正交于
        $x_1,x_2$,即$\alpha_i^*=(x_3,x_i),i=1,2$符合要求,故证得。
    \end{proof}

    \problem
    \begin{proof}
        不失一般的,我们直接设指标集$I$不可数,下面我们
        证明对于任意$\delta>0$,只有有限个Fourier系数的模大于$\delta$。

        反证法,倘若不然,则存在$\delta>0$,可以构造出一个Fourier系数列使得数列
        中所有项的模大于$\delta$,即$|(x,e_{\alpha_k})|>\delta,k=1,2,\ldots$,
        因此
        \[\sum_{k=1}^\infty|(x,e_{\alpha_k})|^2>\infty\]
        这与Bessel不等式矛盾,从而上述
        结论不成立。

        同时由于
        \[\{\alpha\in I|(x,e_\alpha)\neq 0\}
        =\bigcup_{n=1}^\infty\left\{\alpha\in I\left||(x,e_\alpha)|>\frac{1}{n}\right.\right\}\]
        即不为0的指标集是有限集的可数并,从而也是可数的。
    \end{proof}

    \problem
    \begin{proof}
        首先$M^\perp$为闭子空间,事实上对于任意收敛点列$y_n\in M^\perp,y_n\to y$,
        以及$x\in M$,由内积的连续性
        \[(y,x)=\left(\lim_{n\to\infty}y_n,x\right)
        =\lim_{n\to\infty}(y_n,x)=0\]
        从而可知$M^\perp$为闭子空间(子空间是显然的)。

        故对于任意$z\in H$,作正交分解$z=x+y$,其中$x\in M,y\in M^\perp$,
        则由于$M,M^\perp$为闭集,故分别有$x_n\in M,y_n\in M^\perp$使得
        $x_n\to x,y_n\to y$,不妨设
        \[x_n=\sum_{k=1}^{K_n}\alpha_ke_k,
        y_n=\sum_{k=1}^{K_n'}\beta_ke_k'\]
        则令
        \[z_n=x_n+y_n\in\mathrm{span}\{e_k,e_k'\}\]
        同时有$z_n\to x+y=z$,因此标准正交系$\{e_n\},\{e_n'\}$为$H$的标准正交基。
    \end{proof}

    \problem
    \begin{proof}
        由于
        \[\left(x-\sum_{k=1}^n(x,e_k)e_k,e_i\right)=0,i=1,2,\ldots n\]
        因此
        \[\left\|x-\sum_{k=1}^n(x,e_k)e_k\right\|^2
        =\left(x-\sum_{k=1}^n(x,e_k)e_k,x\right)
        =||x||^2-\sum_{k=1}^n|(x,e_k)|^2\geq 0\]
        从而
        \[\sum_{k=1}^n|(x,e_k)|^2\leq ||x||^2\]
        同理有
        \[\sum_{k=1}^n|(y,e_k)|^2\leq ||y||^2\]
        于是由Cauchy不等式
        \[\left(\sum_{k=1}^n|(x,e_k)|\cdot|(y,e_k)|\right)^2
        \leq
        \left(\sum_{k=1}^n|(x,e_k)|^2\right)
        \left(\sum_{k=1}^n|(y,e_k)|^2\right)
        \leq ||x||^2||y||^2\]
        即
        \[\sum_{k=1}^n|(x,e_k)(y,e_k)|\leq
        ||x||\cdot||y||\]
        从而
        \[\sum_{k=1}^\infty|(x,e_k)(y,e_k)|\leq ||x||\cdot||y||\]


    \end{proof}

    \problem
    \begin{proof}
        不妨设$\{e_k\}$是完备的,而对于任意$f\perp\{e_k'\}$,
        \[|(f,e_k)|=|(f,e_k-e_k')|\leq||f||\cdot||e_k-e_k'||\]
        于是
        \[\sum_{k=1}^\infty |(f,e_k)|^2
        \leq||f||^2\sum_{k=1}^\infty||e_k-e_k'||^2\]
        而上式左端由$\{e_k\}$的完备性知
        \[||f||^2=\sum_{k=1}^\infty|(f,e_k)|^2\]
        于是由$\sum_{k=1}^\infty||e_k-e_k'||^2<1$可知$||f||^2=0$,
        从而$f=\boldsymbol 0$,即$\{e_k'\}$是完全的,因此也是完备的。
    \end{proof}
\end{document}