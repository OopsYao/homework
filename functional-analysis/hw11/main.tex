\documentclass[cn]{homework}

\title{第十二周作业}

\begin{document}
    \maketitle

    \problem
    % TODO P1

    \problem
    % TODO P2

    \problem
    \begin{proof}
        不失一般的,我们直接设指标集$I$不可数,下面我们
        证明对于任意$\delta>0$,只有有限个Fourier系数的模大于$\delta$。

        反证法,倘若不然,则存在$\delta>0$,可以构造出一个Fourier系数列使得数列
        中所有项的模大于$\delta$,即$|(x,e_{\alpha_k})|>\delta,k=1,2,\ldots$,
        因此
        \[\sum_{k=1}^\infty|(x,e_{\alpha_k})|^2>\infty\]
        这与Bessel不等式矛盾,从而上述
        结论不成立。

        同时由于
        \[\{\alpha\in I|(x,e_\alpha)\neq 0\}
        =\bigcup_{n=1}^\infty\left\{\alpha\in I\left||(x,e_\alpha)|>\frac{1}{n}\right.\right\}\]
        即不为0的指标集是有限集的可数并,从而也是可数的。
    \end{proof}

    \problem
    \begin{proof}
        首先$M^\perp$为闭子空间,事实上对于任意收敛点列$y_n\in M^\perp,y_n\to y$,
        以及$x\in M$,由内积的连续性
        \[(y,x)=\left(\lim_{n\to\infty}y_n,x\right)
        =\lim_{n\to\infty}(y_n,x)=0\]
        从而可知$M^\perp$为闭子空间(子空间是显然的)。

        故对于任意$z\in H$,作正交分解$z=x+y$,其中$x\in M,y\in M^\perp$,
        则由于$M,M^\perp$为闭集,故分别有$x_n\in M,y_n\in M^\perp$使得
        $x_n\to x,y_n\to y$,不妨设
        \[x_n=\sum_{k=1}^{K_n}\alpha_ke_k,
        y_n=\sum_{k=1}^{K_n'}\beta_ke_k'\]
        则令
        \[z_n=x_n+y_n\in\mathrm{span}\{e_k,e_k'\}\]
        同时有$z_n\to x+y=z$,因此标准正交系$\{e_n\},\{e_n'\}$为$H$的标准正交基。
    \end{proof}

    \problem
    \begin{proof}
        由于
        \[\left(x-\sum_{k=1}^n(x,e_k)e_k,e_i\right)=0,i=1,2,\ldots n\]
        因此
        \[\left\|x-\sum_{k=1}^n(x,e_k)e_k\right\|^2
        =\left(x-\sum_{k=1}^n(x,e_k)e_k,x\right)
        =||x||^2-\sum_{k=1}^n|(x,e_k)|^2\geq 0\]
        从而
        \[\sum_{k=1}^n|(x,e_k)|^2\leq ||x||^2\]
        同理有
        \[\sum_{k=1}^n|(y,e_k)|^2\leq ||y||^2\]
        于是由Cauchy不等式
        \[\left(\sum_{k=1}^n|(x,e_k)|\cdot|(y,e_k)|\right)^2
        \leq
        \left(\sum_{k=1}^n|(x,e_k)|^2\right)
        \left(\sum_{k=1}^n|(y,e_k)|^2\right)
        \leq ||x||^2||y||^2\]
        即
        \[\sum_{k=1}^n|(x,e_k)(y,e_k)|\leq
        ||x||\cdot||y||\]
        从而
        \[\sum_{k=1}^\infty|(x,e_k)(y,e_k)|\leq ||x||\cdot||y||\]


    \end{proof}
\end{document}