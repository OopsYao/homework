\documentclass{homework}

\title{Homework 6}

\newcommand{\var}{\mathrm{Var}}

\begin{document}
    \maketitle    

    \problem
    \begin{proof}
        We argue by contradiction\sidenote{The opposite of statement
        that a.s. does not have a bounded variation should be that
        the probability of having a bounded variation is not zero.
        But here we assumes the probability is 1, i.e., a.s.
        having a bounded variation, which is not the same thing.
        And I believe the latter is the convention when we talk about
        the opposite of a.s. not.}. If Wiener process
        $W_t$ a.s. has a bounded variation, i.e.,
        for a.s. $\omega\in\Omega$,
        \[\sum_{i=0}^{n-1}
        |W_{t_{i+1}}(\omega)-W_{t_i}(\omega)|<C(\omega)\]
        it follows that the partial sum of quadratic variation
        satisfies
        \[I_N=\sum_{i=0}^{n-1}
        |W_{t_{i+1}}(\omega)-W_{t_i}(\omega)|^2\leq
        \tau\sum_{i=0}^{n-1}|W_{t_{i+1}}(\omega)-W_{t_i}(\omega)|
        <\tau C(\omega)\]
        where $\tau=\max_i|t_{i+1}-t_i|$ is the mesh of the
        partition.
        Then we send $\tau\to 0^+$ and obtain
        \[\text{a.s.}\quad I_N\to 0\]
        i.e., $[W_t,W_t]=0$ in a.s. sense, which implies
        convergence in probability
        \[\text{s.t.}\quad I_N\to 0\]
        
        And we know that
        \[\text{m.s.}\quad I_N\to T\]
        as the quadratic variation is $T$ on $[0,T]$,
        which yields convergence in probability
        \[\text{s.t.}\quad I_N\to T\]
        as well. And this leads a contradiction because of the
        uniqueness of limit.
    \end{proof}

    \problem
    % TODO What do you mean by NOT turn in

    \problem
    \begin{subproblem}[(\arabic*).]
        \item
        It should be understood as for any arbitrary $T>0$,
        \[\int_0^T\diff(W_t^3)=3\left(\int_0^TW_t^2\diff W_t
        +\int_0^TW_t\diff t\right)\] 
        or for any partition of $[0,T]$,
        \begin{multline*}
            \text{m.s.-}\lim_{\tau\to 0^+}
        \sum_{i=0}^{N-1}
        W_{t_{i+1}}^3-W^3_{t_i}
        =\text{m.s.-}\lim_{\tau\to 0^+}
        \sum_{i=0}^{N-1}
        3W_{t_i}^2(W_{t_{i+1}}-W_{t_i})\\
        +\text{m.s.-}\lim_{\tau\to0^+}
        \sum_{i=0}^{N-1}3W_{t^*}(t_{i+1}-t_i)
        \end{multline*}
        where $t^*_i$ of the Riemann partial sum is arbitrary
        in $[t_i,t_{i+1}]$.

        \item
        \begin{proof}
            We know that
            \[\begin{aligned}
                \diff (W_t^2)&=2W_t\diff W_t+(\diff W_t)^2\\
                &=2W_t\diff W_t+\diff t
            \end{aligned}\]
            hence
            \[\begin{aligned}
                \diff(W_t^3)&=\diff(W_t^2\cdot W_t)\\
                &=W_t^2\diff W_t+W_t\diff(W_t^2)+\diff(W_t^2)\cdot\diff W_t\\
                &=W_t^2\diff W_t
                +2W_t^2\diff W_t+W_t(\diff W_t)^2
                +2W_t(\diff W_t)^2+(\diff W_t)^3\\
                &=3W_t^2\diff W_t+3W_t(\diff W_t)^2+(\diff W_t)^3
            \end{aligned}\]
            Since
            \[\diff(W^2_t)=\diff t\]
            then we have that
            \[\begin{aligned}
                \diff(W_t^3)=3W_t^2\diff W_t+3W_t\diff t+\diff t\cdot\diff W_t
            \end{aligned}\]
            And in our previous homework we know that
            \[\diff t\cdot\diff W_t=0\]
            Therefore,
            \[\diff(W_t^3)=3W_t^2\diff W_t+3W_t\diff t\]
            \end{proof}
    \end{subproblem}
\end{document}