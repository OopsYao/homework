\documentclass{homework}

\title{Homework 8}

\begin{document}
    \maketitle

    \problem
    \begin{subproblem}[(\alph*)]
        \item
        We know from heat equation method that
        our goal is to find the solution $f(t,x)$ to
        \[\left\{\begin{aligned}
            &\frac{\partial f}{\partial t}+\frac{1}{2}\frac{\partial^2 f}{\partial x^2}=0,t\geq 0\\
            &\frac{\partial f}{\partial x}=\e^{-\frac{\lambda^2t}{2}\pm\lambda x}\\
        \end{aligned}\right.\]
        Take the case $+\lambda$ for example, the second equation gives
        us
        \[f(t,x)=\frac{1}{\lambda}\e^{-\frac{\lambda^2t}{2}+\lambda x}+C(t)\]
        where $C(t)$ is an aribitary differentiable function of $t$, if $\lambda\neq 0$, otherwise
        \[f(t,x)=x+C(t)\]
        Then we further have that if $\lambda\neq 0$,
        \[\begin{aligned}
            \frac{\partial f}{\partial t}
            &=-\frac{\lambda}{2}\e^{-\frac{\lambda^2t}{2}+\lambda x}+C'(t)\\
            \frac{\partial^2 f}{\partial x^2}
            &=\lambda\e^{-\frac{\lambda^2t}{2}+\lambda x}
        \end{aligned}\]
        and if $\lambda=0$,
        \[\begin{aligned}
            \frac{\partial f}{\partial t}
            &=C'(t)\\
            \frac{\partial^2 f}{\partial x^2}
            &=0
        \end{aligned}\]
        Hence the first equation yields
        \[\frac{\partial f}{\partial t}+\frac{1}{2}\frac{\partial^2 f}{\partial x^2}
        =C'(t)=0\]
        which gives us $C(t)=C_0$, an aribitary constant.
        Thus we obtain
        \[f(t,x)=\begin{cases}
            \frac{1}{\lambda}\e^{-\frac{\lambda^2t}{2}+\lambda x}+C_0,&\lambda\neq 0\\
            x+C_0,&\lambda=0
        \end{cases}\]
        Via heat equation method, it follows that
        \[\int_0^T\e^{-\frac{\lambda^2t}{2}+\lambda W_t}\diff W_t
        =\left.f(t,W_t)\right|_0^T
        =\begin{cases}
        \frac{1}{\lambda}\left(\e^{-\frac{\lambda^2t}{2}+\lambda W_t}-1\right),&\lambda\neq 0\\
        W_T,&\lambda=0
        \end{cases}\]
        Then we have from symmetry that
        \[\int_0^T\e^{-\frac{\lambda^2t}{2}\pm\lambda W_t}\diff W_t
        =\begin{cases}
        \pm\frac{1}{\lambda}\left(\e^{-\frac{\lambda^2t}{2}\pm\lambda W_t}-1\right),&\lambda\neq 0\\
        W_T,&\lambda=0
        \end{cases}\]
        
        \item
        If $\lambda=0$, we obtain the integral directly
        \newcommand{\thisf}[1]{\e^{\frac{\lambda^2t}2}\cos(\lambda #1)}
        \[\int_0^T\thisf{W_t}\diff W_t=W_T\]
        For $\lambda\neq 0$,
        the corresponded heat equation is
        \[\left\{\begin{aligned}
            &\frac{\partial f}{\partial t}+\frac{1}{2}\frac{\partial^2 f}{\partial x^2}=0,t\geq 0\\
            &\frac{\partial f}{\partial x}=\thisf{x}
        \end{aligned}\right.\]
        and the second equation yields
        \[f(t,x)=\frac{1}{\lambda}\e^{\frac{\lambda^2t}{2}}\sin(\lambda x)+C(t)\]
        where $C(t)$ is an aribitary differentiable function. Then we have that
        \[\begin{aligned}
            \frac{\partial f}{\partial t}
            &=\frac{\lambda}{2}\e^{\frac{\lambda^2t}{2}}\sin(\lambda x)+C'(t)\\
            \frac{\partial^2 f}{\partial x^2}
            &=-\lambda\e^{\frac{\lambda^2t}{2}}\sin(\lambda x)\\
        \end{aligned}\]
        It follows that
        \[\frac{\partial f}{\partial t}+\frac{1}{2}\frac{\partial^2 f}{\partial x^2}
        =C'(t)=0\]
        which gives us
        \[f(t,x)=\frac{1}{\lambda}\e^{\frac{\lambda^2t}{2}}\sin(\lambda x)+C_0\]
        where $C_0$ is an aribitary constant.
        Therefore
        \[\int_0^T\thisf{W_t}\diff W_t=f(t,W_t)|_0^T
        =\frac{1}{\lambda}\e^{\frac{\lambda^2T}{2}}\sin(\lambda W_T)\]

        \item
        If $\lambda=0$, we obtain the integral directly
        \renewcommand{\thisf}[1]{\e^{\frac{\lambda^2t}2}\sin(\lambda #1)}
        \[\int_0^T\thisf{W_t}\diff W_t=0\]
        Otherwise the coresponded heat equation reads
        \[\left\{\begin{aligned}
            &\frac{\partial f}{\partial t}+\frac{1}{2}\frac{\partial^2 f}{\partial x^2}=0,t\geq 0\\
            &\frac{\partial f}{\partial x}=\thisf{x}
        \end{aligned}\right.\]
        where the second equation gives us
        \[f(t,x)=-\frac{1}{\lambda}\e^{\frac{\lambda^2t}{2}}\cos(\lambda x)+C(t)\]
        where $C(t)$ is an aribitary differentiable function. Hence
        \[\begin{aligned}
            \frac{\partial f}{\partial t}&=-\frac{\lambda}{2}
            \e^{\frac{\lambda^2t}{2}}\cos(\lambda x)+C'(t)\\
            \frac{\partial^2f}{\partial x^2}&=\lambda\e^{\frac{\lambda^2t}{2}}\cos(\lambda x)
        \end{aligned}\]
        It follows that
        \[\frac{\partial f}{\partial t}+\frac{1}{2}\frac{\partial^2 f}{\partial x^2}
        =C'(t)=0\]
        thus
        \[f(t,x)=-\frac{1}{\lambda}\e^{\frac{\lambda^2t}{2}}\cos(\lambda x)+C_0\]
        where $C_0$ is an aribitary constant. Therefore,
        \[\int_0^T\thisf{W_t}\diff W_t=f(t,W_t)|_0^T
        =\frac{1}{\lambda}\left(1-\e^{\frac{\lambda^2T}{2}}\cos(\lambda W_T)\right)\]
    \end{subproblem}

    \problem
    \begin{subproblem}[(\alph*)]
        \item
        We have from It\^o's lemma that
        \[W_t\diff W_t=\diff\left(\frac{W_t^2}{2}\right)-\frac{\diff t}{2}\]
        It follows that
        \[\diff X_t=\diff t+W_t\diff W_t=\frac{\diff t}{2}+\frac{\diff (W_t^2)}{2}\]
        hence
        \[\begin{aligned}
            X_t&=X_0+\int_0^t\diff X_s\\
            &=1+\frac{1}{2}\left(\int_0^t\diff s+\int_0^t\diff(W_s^2)\right)\\
            &=1+\frac{t+W_t^2}{2}
        \end{aligned}\]

        \item
        We have from It\^o's lemma that
        \[W_t^2\diff W_t=\diff\left(\frac{W_t^3}{3}\right)-W_t\diff t\]
        hence
        \[\diff X_t=(W_t-1)\diff t+W_t^2\diff W_t=-\diff t+\frac{\diff(W_t^3)}{3}\]
        which yields
        \[\begin{aligned}
            X_t&=X_0+\int_0^t\diff X_s\\
            &=-\int_0^t\diff s+\frac{1}{3}\int_0^t(\diff W_s^3)\\
            &=\frac{W_t^3}{3}-t
        \end{aligned}\]

        \item
        By It\^o's lemma (or product rule),
        \[t\diff W_t=d(tW_t)-W_t\diff t\]
        then we have that
        \[\diff X_t=3t^2\diff t+\diff(tW_t)\]
        hence
        \[X_t=X_0+\int_0^t3s^2\diff s+\int_0^t\diff (sW_s)=t(t^2+W_t)\]
    \end{subproblem}

    \problem
    We have from integration by parts that
    \[\begin{aligned}
        \int_a^b\frac{\partial f(t,W_t)}{\partial W_t}\diff W_t
        &=f(t,W_t)|_a^b
        -\int_a^b\left(\frac{\partial f(t,W_t)}{\partial t}
        +\frac{1}{2}\frac{\partial^2 f(t,W_t)}{\partial^2 W_t}\right)
        \diff t\\
        &=f(b,W_b)-f(a,W_a)-\int_a^bG(t)\diff t
    \end{aligned}\]
    as $f(t,x)$ satisfies
    \[\frac{\partial f}{\partial t}+\frac{1}{2}\frac{\partial^2 f}{\partial x^2}=G(t)\]

    \problem
    The corresponded heat equation is
    \[\left\{\begin{aligned}
        &\frac{\partial\varphi}{\partial t}
        +\frac{1}{2}\frac{\partial^2\varphi}{\partial x^2}=0\\
        &\frac{\partial\varphi}{\partial x}=x^2+g(t)
    \end{aligned}\right.\]
    The second equation gives us
    \[\varphi(t)=\frac{x^3}{3}+xg(t)+c(t)\]
    where $c(t)$ is a function of $t$. Substituting this form into
    the heat equation yields $c'(t)=-x-xg'(t)$. Hence we can choose
    $g(t)=-t$ to make $c'(t)=0$ being valid, thus $c(t)=c_0$ being an aribitary
    constant. Therefore,
    \[\int_0^TW_t^2+g(t)\diff W_t=\left.\frac{W_t^3}{3}-tW_t\right|_0^T
    =\frac{W_T^3}{3}-TW_T\]

    As for $\int_0^Tg(t)\diff W_t$, we can obtain by product rule that
    \[\int_0^Tg(t)\diff W_t=W_tg(t)|_0^T-\int_0^TW_t\diff g(t)
    =-TW_T+\int_0^TW_t\diff t\]
    and we further conclude that
    \[\int_0^TW_t^2\diff W_t
    =\int_0^TW_t^2+g(t)\diff W_t-\int_0^Tg(t)\diff W_t
    =\frac{W_T^3}{3}-\int_0^TW_t\diff t\]

    \problem
    Denote $X_t:=W_t\e^{kW_t}$, then we have from It\^o's lemma that
    \[\begin{aligned}
        \diff X_t&=(1+kW_t)\e^{kW_t}\diff W_t
        +\frac{1}{2}(2k+k^2W_t)\e^{kW_t}\diff t\\
        &=(1+kW_t)\e^{kW_t}\diff W_t+\left(k\e^{kW_t}+\frac{k^2}{2}X_t\right)\diff t
    \end{aligned}\]
    hence
    \[\begin{aligned}
        E(X_t-X_0)&=E\left(\int_0^t\left(k\e^{kW_s}+\frac{k^2}{2}X_s\right)\diff s\right)\\
        &=\int_0^t\left(kE(\e^{kW_s})+\frac{k^2}{2}E(X_s)\right)\diff s
    \end{aligned}\]
    i.e.,
    \[\frac{\diff E(X_t)}{\diff t}=\frac{k^2}{2}E(X_t)+kE(\e^{kW_t})
    =\frac{k^2}{2}E(X_t)+k\e^{\frac{k^2t}{2}}\]
    as
    \[E(\e^{kW_t})=\e^{\frac{k^2t}{2}}\]
    The ODE above gives us
    \begin{equation}
        \label{eq:EXt}
        E(W_t\e^{kW_t})=E(X_t)=kt\e^{\frac{k^2t}{2}}
    \end{equation}
    since $E(X_0)=0$.

    Denote $Y_t:=W_t^2\e^{W_t}$, then we can derive the corresponded SDE as
    \[\begin{aligned}
        \diff Y_t&=W_t(2+W_t)\e^{W_t}\diff W_t+\frac{1}{2}(2+4W_t+W_t^2)\e^{W_t}\diff t\\
        &=W_t(2+W_t)\diff W_t+\left((1+2W_t)\e^{W_t}+\frac{X_t}{2}\right)\diff t
    \end{aligned}\]
    Similarly we can obtain the ODE of $E(Y_t)$ after dropping the It\^o integral term,
    \[\frac{\diff E(Y_t)}{\diff t}=\frac{1}{2}E(X_t)+E((1+2W_t)\e^{W_t})
    =\frac{1}{2}E(X_t)+(1+2t)\e^{\frac{t}{2}}\]
    since \cref{eq:EXt} gives us
    \[E((1+2W_t)\e^{W_t})=E(\e^{W_t})+2E(W_t\e^{W_t})=(1+2t)\e^{\frac{t}{2}}\]
    With initial condition $E(Y_0)=0$, the ODE is solved as
    \[E(W_t^2\e^{W_t})=E(Y_t)=t(1+t)\e^{\frac{t}{2}}\]

    \problem
    % TODO EW^{2k}

    \problem
    \newcommand{\X}{\boldsymbol Z}
    Denote $X_t:=\sin(t+\sigma W_t),Y_t:=\cos(t+\sigma W_t)$,
    then we have from It\^o's lemma that
    \[\begin{aligned}
        \diff X_t&=
        \begin{aligned}[t]
        &\left(\cos(t+\sigma W_t)-\frac{\sigma^2}{2}\sin(t+\sigma W_t)\right)\diff t\\
        &+\sigma\cos(t+\sigma W_t)\diff W_t\\
        \end{aligned}\\
        \diff Y_t&=
        \begin{aligned}[t]
        &-\left(\sin(t+\sigma W_t)+\frac{\sigma^2}{2}\cos(t+\sigma W_t)\right)\diff t\\
        &-\sigma\sin(t+\sigma W_t)\diff W_t
        \end{aligned}
    \end{aligned}\]
    which derives the ODE of expectation accordingly
    \[\begin{aligned}
        \frac{\diff X_t}{\diff t}&=Y_t-\frac{\sigma^2}{2}X_t\\
        \frac{\diff Y_t}{\diff t}&=-X_t-\frac{\sigma^2}{2}Y_t
    \end{aligned}\]
    If we further denote
    \[\X:=\begin{pmatrix}
        X_t\\
        Y_t
    \end{pmatrix},
    A:=\begin{pmatrix}
        -\frac{\sigma^2}{2} & 1\\
        -1 & -\frac{\sigma^2}{2}
    \end{pmatrix}\]
    then the ODEs become
    \[\frac{\diff\X}{\diff t}=A\X\]
    with initial condition $\X_0=(0,1)^\mathrm T$.
    Here the symbol $\diff\X/\diff t$ represents $(\diff X_t/\diff t,\diff Y_t/\diff t)^\mathrm T$.

    \newcommand{\img}{\mathrm i}
    After some basic calculation one can see that the matrix $A$ has eigenvalues
    $\lambda_1=-\sigma^2/2+\img,\lambda_2=-\sigma^2/2-\img$, with eigenvectors $(-\img,1)^\mathrm T,(\img,1)^\mathrm T$
    accordingly. It follows that the general solution to the ODEs is
    \[\X=C_1\begin{pmatrix}
        -\img\\
        1
    \end{pmatrix}\e^{\img t}
    +C_2\begin{pmatrix}
        \img\\
        1
    \end{pmatrix}\e^{-\img t}\]
    where $C_1,C_2$ are constants to be determined.
    Substituting initial condition into the entity above gives us
    \[C_1=C_2=\frac{1}{2}\]
    Therefore in sight of Euler's formula,
    \[\X=\begin{pmatrix}
        \frac{\img}{2}(\e^{-\img t}-\e^{\img t})\\
        \frac{1}{2}(\e^{-\img t}+\e^{\img t})
    \end{pmatrix}
    =\begin{pmatrix}
        \sin t\\
        \cos t
    \end{pmatrix}\]
    i.e., $E(\sin(t+\sigma W_t))=\sin t,E(\cos(t+\sigma W_t))=\cos t$.
\end{document}