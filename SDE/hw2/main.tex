\documentclass{homework}

\title{Homework 2}

\DeclareMathOperator{\Cov}{Cov}
\DeclareMathOperator{\Var}{Var}
\DeclareMathOperator{\Corr}{Corr}

\begin{document}
    \maketitle    

    \section{Problem 1}
    Without loss of generality, assume that $t\geq s$.
    Then by the law of distributivity,
    \begin{align*}
        \Cov(B_t,B_s)&=\Cov(B_t-B_s+B_s,B_s)\\
                     &=\Cov(B_t-B_s,B_s)+\Cov(B_s,B_s)
    \end{align*} 
    Since Brownian motion has independent increments,
    we have that
    \[\Cov(B_t-B_s,B_s)=0\]
    Therefore
    \[\Cov(B_t,B_s)=\Var(B_s)=s\]
    i.e.,
    \[\Cov(B_t,B_s)=\min\{t,s\}\]
    in general.

    Hence
    \[\Corr(B_t,B_s)=\frac{\Cov(B_t,B_s)}{\sqrt{\Var(B_t)\Var(B_s)}}
    =\frac{\min\{t,s\}}{\sqrt{ts}}\]

    \section{Problem 2}
    Note that $E\left(B_t^2\right)=t$, so by definition of
    covariance,
    \begin{equation}
        \label{eq:cov def}
        E\left((B_t^2-t)(B_s^2-s)\right)=\Cov(B_t^2,B_s^2)
    \end{equation}
    Also note that
    \begin{equation}
        \label{eq:cov formula}
        \Cov(B_t^2,B_s^2)=E(B_t^2B_s^2)-E(B_t^2)\cdot E(B_s^2)
    \end{equation}

    So the key to these problems is calculating $E(B_t^2B_s^2)$,
    which can be obtained by law of total expectation. Therefore,
    we can start from $E(B_t^2B_s^2|\mathcal F_s)$, and without
    loss of generality, we assume that $s\leq t$.

    We have that
    \begin{align*}
        E\left.\left(B_t^2B_s^2\right|\mathcal F_s\right)
        &=B_s^2E\left.\left(B_t^2\right|\mathcal F_s\right)\\
    \end{align*}
    and
    \begin{align*}
        E\left.\left(B_t^2\right|\mathcal F_s\right)
        &=E\left.\left((B_t-B_s+B_s)^2\right|\mathcal F_s\right)\\
        &=E\left((B_t-B_s)^2|\mathcal F_s\right)
            +2E\big(B_s(B_t-B_s)|\mathcal F_s\big)
            +E\left.\left(B_s^2\right|\mathcal F_s\right)
    \end{align*}
    Since Brownian motion has independent increments, we have
    that
    \begin{align*}
    E\left.\left((B_t-B_s)^2\right|\mathcal F_s\right)
    &=E\left((B_t-B_s)^2\right)=t-s\\
    E\left.\left(B_s(B_t-B_s)\right|\mathcal F_s\right)&=B_sE(B_t-B_s)=0\\
    E\left.\left(B_s^2\right|\mathcal F_s\right)&=B_s^2
    \end{align*}
    Hence
    \begin{align*}
        E\left.\left(B_t^2B_s^2\right|\mathcal F_s\right)
        &=B_s^2\cdot E\left.\left(B_t^2\right|\mathcal F_s\right)\\
        &=B_s^2\left(t-s+B_s^2\right)
    \end{align*} 

    Therefore
    \begin{align*}
    E\left(B_t^2B_s^2\right)&=E\left(E\left.\left(B_t^2B_s^2\right|\mathcal F_s\right)\right)\\
    &=E\left(B_s^2\left(t-s+B_s^2\right)\right)\\
    &=(t-s)E\left(B_s^2\right)+E\left(B_s^4\right)
    \end{align*}


    As for $E\left(B_s^4\right)$, since that $\left(B_s/\sqrt{s}\right)^2
    \sim\chi^2(1)$, we obtain,
    \[\Var(B_s^2)=2s^2\]
    thus
    \begin{align*}
        E\left(B_s^4\right)&=\Var\left(B_s^2\right)+\left(E\left(B_s^2\right)\right)^2=3s^2
    \end{align*}
    It follows that
    \begin{align*}
    E\left(B_t^2B_s^2\right)=s(t+2s)
    \end{align*}
    so in general by symmetry,
    \begin{equation}
        \label{eq:EBs2Bt2}
        E\left(B_t^2B_s^2\right)=ts+2\min\{t,s\}
    \end{equation}

    \begin{subproblem}
        \item\label{pb:2.1}
        From \cref{eq:cov def,eq:cov formula} we know that
        \[E\left((B_t^2-t)(B_s^2-s)\right)=ts+2\min\{t,s\}-ts=2\min\{t,s\}\]

        \item
        As in \cref{eq:EBs2Bt2}
        \[E\left(B_t^2B_s^2\right)=ts+2\min\{t,s\}\]

        \item
        Same as \ref{pb:2.1} as shown by \cref{eq:cov def},
        \[\Cov(B_t^2,B_s^2)=2\min\{t,s\}\]
    \end{subproblem}

    \section{Problem 3}
    \begin{proof}
        Since $2X_i-1$ represents the movements of every step,
        we have that
        \[Y_n=\Delta x\cdot\left(\sum_{i=1}2X_i-1\right)
        =\Delta x(2S_n-n)\]
        Then by Laplace-De Moivre Theorem,
        \[\frac{S_n-\frac{n}{2}}{\sqrt{\frac{n}{4}}}\to N(0,1)\]
        i.e.,
        \[\frac{Y_n}{\sqrt{n}\Delta x}\to N(0,1)\]
        Since $D=(\Delta x)^2/\Delta t$ and $t=n\Delta t$,
        we have that
        \[\frac{Y_n}{\sqrt{n}\Delta x}=\frac{Y_n}{\sqrt{Dt}}\]
        And $n\to\infty$ is equivalent $t\to\infty$, therefore
        \[\frac{Y_t}{\sqrt{Dt}}\to N(0,1)\]
        It follows that
        \begin{align*}
        \lim_{n\to\infty,t=n\Delta t}P(a\leq Y_t\leq b)
        &=\frac{1}{\sqrt{2\pi}}
        \int_{a/\sqrt{Dt}}^{b/\sqrt{Dt}}\e^{-\frac{\xi^2}{2}}\diff\xi\\
        &=\frac{1}{\sqrt{2\pi Dt}}
        \int_a^b\e^{-\frac{x^2}{2Dt}}\diff x
        \end{align*}
    \end{proof}
\end{document}