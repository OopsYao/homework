    \problem
    \begin{subproblem}
        \item
        For an arbitrary $T>0$, denote $v(t,x)=u(T-t,\sqrt{2D}x)$,
        then we see that
        \[v_t(t,x)=-u_t(T-t,\sqrt{2D}x),v_{xx}=2Du_{xx}(T-t,\sqrt{2D}x)\]
        thus by the heat equation of $u(t,x)$,
        \[\left\{\begin{aligned}
            &v_t+\frac{v_{xx}}{2}=0,&x\in\mathbb R,t\in[0,T)\\
            &v(T,x)=\varphi(\sqrt{2D}x),&x\in\mathbb R
        \end{aligned}\right.\]

        Then we have from It\^o's lemma that
        \[\diff v(t,W_t)=\left(v_t+\frac{v_{xx}}{2}\right)\diff t
        +v_x\diff W_t=v_x\diff W_t\]
        which gives us
        \[v(T,W_T)-v(t,W_t)=\int_t^Tv_x\diff W_t\]
        Taking conditional expectation with $W_t=x$ on BHS gives us
        \[E[v(T,W_T)|W_t=x]-v(t,x)=0\]
        i.e.,
        \[v(t,x)=E[v(T,W_T)|W_t=x]=E[\varphi(\sqrt{2D}W_T)|W_t=x]\]
        Therefore\footnote{The expression of $u(t,x)$ only contains $T$
        formally, on which it does not depend indeed. For any $t$,
        one just needs to choose a $T$ greater than it.},
        \[u(t,x)=v(T-t,x/\sqrt{2D})
        =E[\varphi(\sqrt{2D}W_T)|W_{T-t}=x/\sqrt{2D}]\]
        which is the probabilistic representation. 

        \item
        \begin{proof}
            We have from the independent increments of Brownian motion
            that
            \[\begin{aligned}
                &E[\varphi(\sqrt{2D}W_T)|W_{T-t}=x/\sqrt{2D}]\\
                =&E[\varphi(x+\sqrt{2D}(W_T-W_{T_t})|W_{T-t}=x/\sqrt{2D}]\\
                =&E[\varphi(x+\sqrt{2D}(W_T-W_{T_t}))]
            \end{aligned}\]
            Then
            \[\begin{aligned}
                u(t,x)&=E[\varphi(x+\sqrt{2D}(W_T-W_{T-t}))]\\
                &=\int_{\mathbb R}\varphi(x+\sqrt{2D}z)\frac{\e^{-\frac{z^2}{2t}}}{\sqrt{2\pi t}}\diff z\\
                (y=x+\sqrt{2D}z)\quad&=\frac{1}{\sqrt{2\pi t}}
                \int_{\mathbb R}\varphi(y)\e^{-\frac{1}{2t}\left(\frac{y-x}{\sqrt{2D}}\right)^2}
                \frac{\diff y}{\sqrt{2D}}\\
                &=\frac{1}{\sqrt{4\pi Dt}}
                \int_{\mathbb R}\e^{-\frac{(x-y)^2}{4Dt}}\varphi(y)\diff y
            \end{aligned}\]
            as $W_T-W_{T-t}\sim\mathcal N(0,t)$.
        \end{proof}

        \item
        See \cref{fig:prob sol}.
        \begin{figure}[h]
            \centering
            \includegraphics[width=\textwidth]{prob-sol}
            \caption{Numerical Solution by Monte Carlo Simulation}
            \label{fig:prob sol}
        \end{figure}
    \end{subproblem}

    \problem
    Denote $X_t$ driven by $\diff X_t=\mu(t,W_t)\diff t+\sigma(t,W_t)\diff W_t$
    with initial value $X_0=x$
    is the corresponded It\^o process. Then we have from It\^o's lemma
    that
    \[\diff f(X_t)=f'(X_t)\diff X_t+f''(X_t)\frac{1}{2}(\diff X_t)^2
    =\left(\mu f'+\frac{1}{2}\sigma f''\right)\diff t+\sigma f'\diff W_t\]
    which gives us
    \[f(X_t)=f(X_0)+\int_0^t\left(\mu f'+\frac{1}{2}\sigma f''\right)\diff s
    +\int_0^t\sigma f'\diff W_s\]
    Taking expectation on BHS yields
    \[E[f(X_t)]=f(x)+\int_0^tE\left[\mu f'+\frac{1}{2}\sigma f''\right]\diff s\]
    Therefore,
    \[Af(x)=\lim_{t\to 0^+}\frac{E[f(X_t)]-f(x)}{t}
    =E\left[\mu f'+\frac{1}{2}\sigma f''\right]\]
    % TODO Something wrong, definition of infinitestimal generator?

    \problem
    % TODO

    \problem
    \begin{subproblem}[(\alph*)]
        \item
        It easy to find that the functions in Feymann-Kac formula
        satisfy $\mu=\sigma=f=1,\psi=v=0$, then we have that 
        \[u(t,x)=E\left[\left.\int_t^T\diff r\right|X_t=x\right]
        =T-t\]
        where $X_t$ is an It\^o process driven by
        \[\diff X_t=\mu\diff t+\sigma\diff W_t\]

        \item
        Denote that $\mathcal A$ is an infinitesimal of
        then It\^o process driven by
        \[\diff X_t=\diff t+\diff W_t\]
        then it easy to find that
        \[\mathcal Af(t,x)=f_t+f_x+\frac{1}{2}f_{xx}\]
        Hence for the solution $u(x,t)$ of the PDE, we have
        that $\mathcal Au=-1$.
        Therefore,
        % TODO Explain how this comes
        \[\begin{aligned}
            E\left[u(T,X_T)|X_t=x\right]
            -u(t,x)&=
            E\left[\left.\int_t^{T}
            \mathcal Af(s,X_{s})\diff s\right|X_t=x\right]\\
            &=t-T
        \end{aligned}\]
        which gives us
        \[u(t,x)=E[u(T,X_T)|X_t=x]+T-t=T-t\]
        And it is apparent that $u(t,x)=T-t$ is the solution to the
        PDE indeed.
    \end{subproblem}

    \problem
    Denote an It\^o diffusion $X_t$ driven by
    \begin{equation}
        \label{eq:a b Ito}
        \diff X_t=\alpha(t,X_t)\diff t+\beta(t,X_t)\diff W_t
    \end{equation}
    then the corresponded infinitesimal generator $\mathcal A$
    is given by
    \[\mathcal A=\frac{\partial}{\partial t}+\alpha(t,x)\frac{\partial}{\partial x}
    +\frac{\beta^2(t,x)}{2}\frac{\partial^2}{\partial x^2}\]
    Hence it is easy to see that we can choose
    \begin{equation}
        \label{eq:param}
        \alpha(t,x)=\mu x,\beta(t,x)=\sigma x
    \end{equation}
    to turn $\mathcal A$ the desired.

    Therefore,
    \[\begin{aligned}
        E[u(T,X_T)|X_t=x]-u(t,x)
        &=E\left[\left.\int_t^T\mathcal Au(s,X_s)\diff s\right|
        X_t=x\right]\\
        &=-\int_t^TE(X_s|X_t=x)\diff s
    \end{aligned}\]
    which gives us
    \[u(t,x)=\int_t^TE(X_s|X_t=x)\diff s\]
    as $u(T,X_T)=0$.
    And \cref{eq:a b Ito} together with \cref{eq:param} gives us
    \[E(X_s)-E(X_t)=\mu\int_t^sE(X_r)\diff r\]
    thus by solving the ODE we have $E(X_s|X_t=x)=x\e^{\mu(s-t)}$.
    Finally, we obatin $u(t,x)$ as
    \[u(t,x)=\int_t^Tx\e^{\mu(s-t)}\diff s
    =\frac{x}{\mu}\left(\e^{\mu(T-t)}-1\right)\]

    \problem
    We know that for the Brownian motion the infinitesimal generator
    $\mathcal A$ is $\Delta/2$, hence if we denote that $X_t=x+W_t$, then
    we have that
    \[E\left[\int_0^{\tau_x}\mathcal Au(x)\diff t\right]
    =E(u(X_{\tau_x}))-E(u(X_0))\]
    And we have from the PDE that $Au(x)=-1$,
    and since $X_{\tau_x}\in\partial\Omega$ thus $u(X_{\tau_x})=0$,
    then we obtain
    \[-E\left[\int_0^{\tau_x}\diff t\right]=-u(x)\]
    i.e.,
    \[u(x)=E(\tau_x)\]
    
    \problem
    \begin{proof}
        Denote $Y_t=u(X_t)\e^{\int_0^tc(X_s)\diff s}$,
        then $Y_0=u(x)$ and
        \[\diff Y_t=
        \begin{aligned}[t]
        u(X_t)\e^{-\int_0^tc(X_s)\diff s}(-c(X_t)\diff t)
        +\e^{\int_0^tc(X_s)\diff s}(\diff u(X_t))\\
        +\diff(u(X_t))\e^{-\int_0^tc(X_s)\diff s}(-c(X_t))\diff t
        \end{aligned}\]
        And we know that
        \[\diff(u(X_t))=u'(X_t)\diff X_t+\frac{1}{2}u''(X_t)(\diff X_t)^2
        =u'(X_t)\diff W_t+\frac{1}{2}u''(X_t)\diff t\]
        Therefore,
        \[\begin{aligned}
            \diff Y_t&=
            \left(\frac{1}{2}u''(X_t)-u(X_t)c(X_t)\right)\e^{-\int_0^tc(X_s)\diff s}\diff t
            +u'(X_t)\e^{-\int_0^tc(X_s)\diff s}\diff W_t\\
            &=\left(-f(X_t)\diff t+u'(X_t)\diff W_t\right)\e^{-\int_0^tc(X_s)\diff s}
        \end{aligned}\]
        where the last identity is due to the PDE.
        Integral over $[0,\tau_x]$ and taking expectation gives us
        \[E(Y_{\tau_x})-E(Y_0)
        =-E\left[\int_0^{\tau_x}f(X_t)\e^{-\int_0^{\tau_x}c(X_s)\diff s}\diff t\right]\]
        Note that $Y_0=u(x)$ and we have from boundary condition that $Y_{\tau_x}=0$,
        finally we obtain $u(x)$ as
        \[u(x)=E\left[\int_0^{\tau_x}f(X_t)\e^{-\int_0^{\tau_x}c(X_s)\diff s}\right]\]
    \end{proof}

    \appendix
    \section{Python Code}
    \lstinputlisting[language=Python]{montecarlo.py}