\documentclass{homework}

\title{Homework 7}

\begin{document}
    \maketitle

    \problem
    \begin{proof}
        Denote
        \[f(t,W_t):=\frac{1}{2}W_t^2-\frac{1}{2}t\]
        then it is equivalent to show that
        \[W_t\diff W_t=\diff f(t,W_t)\]
        as
        \[\int_0^T\diff f(s,W_s)=f(T,W_T)-f(0,W_0)=
        \frac{1}{2}W_T^2-\frac{1}{2}T\]
        And we have from It\^o's lemma that
        \[\diff f(t,W_t)=-\frac{\diff t}{2}+W_t\diff W_t
        +\frac{\diff t}{2}=W_t\diff W_t\]
        which is what we need exactly.
    \end{proof}

    \problem
    \begin{proof}
        Denote
        \[f(t,W_t):=S_t\]
        then it easy to see that
        \[\begin{aligned}
            \frac{\partial f}{\partial t}(t,W_t)
            &=\left(\mu-\frac{\sigma^2}{2}\right)f(t,W_t)
            =\left(\mu-\frac{\sigma^2}{2}\right)S_t\\
            \frac{\partial f}{\partial W_t}(t,W_t)
            &=\sigma f(t,W_t)=\sigma S_t\\
            \frac{\partial^2 f}{\partial^2 W_t}(t,W_t)
            &=\sigma^2 f(t,W_t)=\sigma^2 S_t
        \end{aligned}\]
        Then we have from It\^o's lemma that
        \[\begin{aligned}
            \diff S_t&=\diff f(t,W_t)\\
            &=\left(\mu-\frac{\sigma^2}{2}\right)S_t\diff t
            +\sigma S_t\diff W_t+\frac{\sigma^2 S_t}{2}\diff t\\
            &=\mu S_t\diff t+\sigma S_t\diff W_t
        \end{aligned}\]
    \end{proof}

    \problem
    \begin{subproblem}[(\alph*)]
        \item
        \begin{proof}
            % TODO Proof by definition
        \end{proof}

        \item
        \begin{proof}
            We have from It\^o's lemma that
            \[\begin{aligned}
                \diff\left(\frac{X_t}{Y_t}\right)&=
                \frac{\diff X_t}{Y_t}-\frac{X_t}{Y_t^2}\diff Y_t
                +\frac{1}{2}\left(
                    -\frac{2}{Y_t^2}\diff X_t\diff Y_t
                    +\frac{2X_t}{Y_t^3}(\diff Y_t)^2
                \right)\\
                &=\frac{Y_t\diff X_t-X_t\diff Y_t-\diff X_t\diff Y_t}%
                  {Y_t^2}+\frac{X_t}{Y_t^3}(\diff Y_t)^2
            \end{aligned}\]
        \end{proof}

        Therefore,
        \[\diff\left(\frac{X_t}{f(t)}\right)
        =\frac{f(t)\diff X_t-X_t\diff f(t)}{(f(t))^2}\]
        as $f(t)$ is determinstic which implies that
        \[\diff X_t\diff f(t)=(\diff f(t))^2=0\]
    \end{subproblem}

    \problem
    \begin{subproblem}[(\alph*)]
        \item
        Since $X_t^{(1)},X_t^{(2)}$ are two It\^o diffusions,
        then we have that, for $i=1,2$,
        \[\left(\diff W_t^{(i)}\right)^2=\mu_i^2(\diff t)^2
        +\sigma_i^2\left(\diff W_t^{(i)}\right)^2
        +2\mu_i\sigma_i\diff t\diff W_t^{(i)}
        =\sigma_i^2\diff t\]
        as $W_t^{(1)},W_t^{(2)}$ are independent, where
        $\mu_i,\sigma_i$ is short for
        \[\mu_i\left(t,W_t^{(i)}\right),
        \sigma_i\left(t,W_t^{(i)}\right)\]
        accordingly.

        And we have from multi-dimensional Taylor's expansion that
        \[\begin{aligned}
            \diff f\left(t,X_t^{(1)},X_t^{(2)}\right)=
            \begin{aligned}[t]
            &\frac{\partial f}{\partial t}\diff t
            +\frac{\partial f}{\partial x_1}\diff X_t^{(1)}
            +\frac{\partial f}{\partial x_2}\diff X_t^{(2)}\\
            &+\frac{1}{2}\left(
                \frac{\partial^2 f}{\partial x_1^2}\left(\diff X_t^{(1)}\right)^2
                +\frac{\partial^2 f}{\partial x_2^2}\left(\diff X_t^{(2)}\right)^2
            \right)
            \end{aligned}
        \end{aligned}\]
        where $\frac{\partial f}{\partial x_1}$ is short for
        \[\left.\frac{f(t,x_1,x_2)}{\partial x_1}
        \right|_{(t,x_1,x_2)=\left(t,X_t^{(1)},X_t^{(2)}\right)}\]
        and so on.
        Note that in the equation above we throw away terms that contain,
        \[\left(\diff X_t^{(1)}\right)^p\left(\diff X_t^{(2)}\right)^q,
        \left(\diff X_t^{(i)}\right)^k(k\geq 3)\]
        as they all equal 0 due to the independence and high order
        terms being zero, i.e.,
        \[\left(\diff X_t^{(i)}\right)^k
        =\left(\diff X_t^{(i)}\right)^{k-2}\cdot\sigma_i^2\diff t=0,
        k\geq 3\]
        hence 2-dimensional It\^o's lemma can be obtained as
        \begin{equation}
            \label{eq:2-d Ito lemma}
            \begin{aligned}
            \diff f\left(t,X_t^{(1)},X_t^{(2)}\right)&=
            \begin{aligned}[t]
            &\frac{\partial f}{\partial t}\diff t
            +\frac{\partial f}{\partial x_1}\diff X_t^{(1)}
            +\frac{\partial f}{\partial x_2}\diff X_t^{(2)}\\
            &+\frac{1}{2}\left(
                \frac{\partial^2 f}{\partial x_1^2}\sigma_1^2\diff t
                +\frac{\partial^2 f}{\partial x_2^2}\sigma_2^2\diff t
            \right)
            \end{aligned}\\
            &=\begin{aligned}[t]
            &\left(
                \frac{\partial f}{\partial t}+     
                \frac{1}{2}\left(
                \frac{\partial^2 f}{\partial x_1^2}\sigma_1^2
                +\frac{\partial^2 f}{\partial x_n^2}\sigma_n^2
                \right)
            \right)\diff t\\
            &+\frac{\partial f}{\partial x_1}\diff X_t^{(1)}
            +\frac{\partial f}{\partial x_2}\diff X_t^{(2)}
            \end{aligned}
            \end{aligned}
        \end{equation}

        \item
        The generalization is rather straight-forward since
        the critical condition independence still holds and,
        \[\diff W_t^{i}\diff W_t^{(j)}=
        \begin{cases}
            \sigma_j^2\diff t,&i=j\\
            0,&i\neq j
        \end{cases}\]
        Therefore \cref{eq:2-d Ito lemma} can be simply generalized
        to $n$-dimension,
        \begin{equation*}
            \begin{aligned}
            \diff f\left(t,X_t^{(1)},\ldots,X_t^{(n)}\right)&=
            \begin{aligned}[t]
            &\frac{\partial f}{\partial t}\diff t
            +\frac{\partial f}{\partial x_1}\diff X_t^{(1)}
            +\cdots
            +\frac{\partial f}{\partial x_n}\diff X_t^{(n)}\\
            &+\frac{1}{2}\left(
                \frac{\partial^2 f}{\partial x_1^2}\sigma_1^2\diff t
                +\ldots
                +\frac{\partial^2 f}{\partial x_n^2}\sigma_n^2\diff t
            \right)
            \end{aligned}\\
            % TODO Better notation
            &=\left(\frac{\partial f}{\partial t}
            +\frac{1}{2}\right)\diff t
            +\nabla f\cdot\diff\boldsymbol X_t
            \end{aligned}
        \end{equation*}
        where gradient operator $\nabla$ is defined as\sidenote{
            Here the notation $\frac{\partial f}{\partial x_1}$ is
            similar to the 2-dimension case, as well as other notations
            like $\mu_1,\sigma_1$.
        }
        \[\nabla f:=\left(\frac{\partial f}{\partial x_1},\ldots,\frac{\partial f}{\partial x_n}\right)\]
        and
        \[\begin{aligned}
            \diff\boldsymbol X_t&:=
            \left(\diff W_t^{(1)},\ldots,\diff W_t^{(n)}\right)\\
            \boldsymbol\sigma&:=
            \left(\sigma_1,\ldots,\sigma_n\right)
        \end{aligned}\]
        
        \item
        % TODO Generalization with coefficients
    \end{subproblem}

    \problem
    By It\^o's lemma, we have that
    \[\diff (tW_t^2)=W_t^2\diff t+2tW_t\diff W_t+\frac{2t}{2}\diff t
    =(W_t^2+t)\diff t+2tW_t\diff W_t\]
\end{document}