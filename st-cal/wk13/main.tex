\documentclass[cn]{homework}

\title{第十三周作业}
\newcommand{\E}{\mathbb E}
\newcommand{\prob}{\mathbb P}

\begin{document}
    \maketitle

    \problem[习题4.15]
    \begin{subproblem}
        \item
        \begin{proof}
            由于
            \[\diff B_i(t)=
            \sum_{j=1}^d\frac{\sigma_{ij}(t)}{\sigma_i(t)}\diff W_j(t)\]
            则$B_i(t)$为关于此$d$维布朗运动的连续鞅,同时
            \[\begin{aligned}
                (\diff B_i(t))^2
                &=\sum_{j,k}\frac{\sigma_{ij}(t)\sigma_{ik}(t)}{\sigma_i^2(t)}
                \diff W_j(t)\diff W_k(t)\\
                &=\sum_{j=1}^d\frac{\sigma_{ij}^2(t)}{\sigma_i^2(t)}(\diff W_j(t))^2\\
                &=\diff t
            \end{aligned}\]
            故由列维引理知$B_i(t)$为布朗运动。
        \end{proof}

        \item
        \begin{proof}
            直接计算可得
            \[\begin{aligned}
                \diff B_i(t)\diff B_k(t)
                &=\left(\sum_{j=1}^d\frac{\sigma_{ij}(t)}{\sigma_i(t)}\diff W_j(t)\right)
                \left(\sum_{l=1}^d\frac{\sigma_{kl}(t)}{\sigma_k(t)}\diff W_l(t)\right)\\
                &=\sum_{j,l}\frac{\sigma_{ij}(t)\sigma_{kl}(t)}{\sigma_i(t)\sigma_k(t)}
                \diff W_j(t)\diff W_i(t)\\
                &=\sum_{j=1}^d\frac{\sigma_{ij}(t)\sigma_{kj}(t)}{\sigma_i(t)\sigma_k(t)}
                (\diff W_j(t))^2\\
                &=\rho_{ik}(t)\diff t
            \end{aligned}\]
        \end{proof}
    \end{subproblem}

    \problem[习题4.19]
    \begin{subproblem}
        \item
        \newcommand{\sgn}{\mathrm{sign}}
        \begin{proof}
            由$B(t)$为伊藤积分知其为连续鞅且$B(0)=0$,而
            \[\diff B(t)=\sgn(W(t))\diff W(t)\]
            于是
            \[(\diff B(t))^2=(\diff W(t))^2=\diff t\]
            从而由列维引理知$B(t)$为布朗运动。
        \end{proof}

        \item
        \begin{proof}
            由于
            \[\begin{aligned}
                \diff[B(t)W(t)]&=B(t)\diff W(t)+W(t)\diff B(t)+\diff B(t)\diff W(t)\\
                &=[B(t)+W(t)\sgn(W(t))]\diff W(t)+\sgn(W(t))\diff t
            \end{aligned}\]
            从而两端积分并取期望,由伊藤积分的零期望性得,
            \[\begin{aligned}
                \E[B(t)W(t)]-E[B(0)W(0)]&=\E\int_0^t\sgn(W(s))\diff s\\
                &=\int_0^t\E[\sgn(W(s))]\diff s\\
                &=\int_0^t\left(\int_{W(s)\geq 0}\diff\prob
                -\int_{W(s)<0}\diff\prob\right)\diff s\\
                &=\int_0^t(\prob(W(s)\geq 0)-\prob(W(s)<0))\diff s\\
                &=0
            \end{aligned}\]
            从而
            \[\E[B(t)W(t)]=\E[B(0)W(0)]=0\]
        \end{proof}

        \item
        由乘积法则
        \[\diff W^2(t)=2W(t)\diff W(t)+(\diff W(t))^2=2W(t)\diff W(t)+\diff t\]

        \item
        由于
        \[\begin{aligned}
            \diff[B(t)W^2(t)]&=W^2(t)\diff B(t)+B(t)\diff W^2(t)+\diff B(t)\diff W^2(t)\\
            &=\begin{aligned}[t]
                &[2W(t)\sgn(W(t))+B(t)]\diff t\\
                &+[W^2(t)\sgn(W(T))+2B(t)W(t)]\diff W(t)
            \end{aligned}\\
        \end{aligned}\]
        两端积分取期望得
        \[\begin{aligned}
            &\E[B(t)W^2(t)]-\E[B(0)W^2(0)]\\
            =&\E\int_0^t[2W(s)\sgn(W(s))+B(s)]\diff s\\
            =&\int_0^t2\E[W(s)\sgn(W(s))]\diff t+\int_0^t\E[B(s)]\diff s
        \end{aligned}\]
        注意到
        \[W(s)\sgn(W(s))=|W(s)|\]
        而$B(s)$为伊藤积分,故$\E[B(s)]=0$,从而
        \[\E[B(t)W^2(t)]=2\int_0^t\E[|W(s)|]\diff s>0=\E[B(t)]\cdot\E[W^2(t)]\]
        尽管$B(t)$与$W(t)$不相关,但它们并不服从二元联合正态分布,因此不足以蕴含独立。
    \end{subproblem}
\end{document}