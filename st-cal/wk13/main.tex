\documentclass[cn]{homework}

\title{第十三周作业}

\begin{document}
    \maketitle

    \problem[习题4.15]
    \begin{subproblem}
        \item
        \begin{proof}
            由于
            \[\diff B_i(t)=
            \sum_{j=1}^d\frac{\sigma_{ij}(t)}{\sigma_i(t)}\diff W_j(t)\]
            则$B_i(t)$为关于此$d$维布朗运动的连续鞅,同时
            \[\begin{aligned}
                (\diff B_i(t))^2
                &=\sum_{j,k}\frac{\sigma_{ij}(t)\sigma_{ik}(t)}{\sigma_i^2(t)}
                \diff W_j(t)\diff W_k(t)\\
                &=\sum_{j=1}^d\frac{\sigma_{ij}^2(t)}{\sigma_i^2(t)}(\diff W_j(t))^2\\
                &=\diff t
            \end{aligned}\]
            故由列维引理知$B_i(t)$为布朗运动。
        \end{proof}

        \item
        \begin{proof}
            直接计算可得
            \[\begin{aligned}
                \diff B_i(t)\diff B_k(t)
                &=\left(\sum_{j=1}^d\frac{\sigma_{ij}(t)}{\sigma_i(t)}\diff W_j(t)\right)
                \left(\sum_{l=1}^d\frac{\sigma_{kl}(t)}{\sigma_k(t)}\diff W_l(t)\right)\\
                &=\sum_{j,l}\frac{\sigma_{ij}(t)\sigma_{kl}(t)}{\sigma_i(t)\sigma_k(t)}
                \diff W_j(t)\diff W_i(t)\\
                &=\sum_{j=1}^d\frac{\sigma_{ij}(t)\sigma_{kj}(t)}{\sigma_i(t)\sigma_k(t)}
                (\diff W_j(t))^2\\
                &=\rho_{ik}(t)\diff t
            \end{aligned}\]
        \end{proof}
    \end{subproblem}
\end{document}