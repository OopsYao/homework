\documentclass[cn]{homework}

\title{第十一周作业}

\newcommand{\E}{\mathbb{E}}

\begin{document}
    \maketitle
    
    \problem[习题4.5]
    \begin{subproblem}
        \item
        由伊藤公式
        \[\diff\log S(t)=\frac{\diff S(t)}{S(t)}-\frac{(\diff S(t))^2}{2S^2(t)}\]
        代入化简即得
        \begin{equation}
            \label{eq:d ln}
            \diff\log S(t)=\left(\alpha(t)-\frac{\sigma^2(t)}{2}\right)\diff t
            +\sigma(t)\diff W(t)
        \end{equation}

        \item
        \cref{eq:d ln}两端积分即得
        \[\log S(t)-\log S(0)
        =\int_0^t\left(\alpha(s)-\frac{\sigma^2(s)}{2}\right)\diff s
        +\int_0^t\sigma(s)\diff W(s)\]
        即
        \[S(t)=S(0)\exp\left\{
        \int_0^t\left(\alpha(s)-\frac{\sigma^2(s)}{2}\right)\diff s
        +\int_0^t\sigma(s)\diff W(s)
        \right\}\]
    \end{subproblem}

    \problem[习题4.6]
    由于
    \[S^p(t)=S^p(0)\e^{p\alpha W(t)+p\left(\alpha-\frac{\sigma^2}{2}\right)t}\]
    从而由伊藤公式
    \[\begin{aligned}
        \diff(S^p(t))&=p\left(\alpha-\frac{\sigma^2}{2}\right)S^p(t)\diff t
        +p\sigma S^p(t)\diff W(t)+\frac{1}{2}p^2\sigma^2S^p(t)\diff t\\
        &=pS^p(t)\left(
        \left(\alpha+\frac{\sigma^2(p-1)}{2}\right)\diff t+\sigma\diff W(t)
        \right)
    \end{aligned}\]

    \problem[习题4.7]
    \begin{subproblem}
        \item
        由伊藤公式,
        \[\diff W^4(t)=4W^3(t)\diff W(t)+6W^2(t)\diff t\]
        两端在$[0,T]$上积分得
        \begin{equation}
            \label{eq:W^4}
            W^4(T)=4\int_0^TW^3(t)\diff W(t)+6\int_0^TW^2(t)\diff t
        \end{equation}

        \item
        由伊藤积分的零期望特性,\cref{eq:W^4}两端取期望得
        \[\E W^4(t)=6\E\int_0^TW^2(t)\diff t=6\int_0^T\E W^2(t)\diff t
        =6\int_0^Tt\diff t=3T^2\]

        \item
        由伊藤公式
        \[\diff W^6(t)=6W^5(t)\diff W(t)+15W^4(t)\diff t\]
        两端积分取期望得
        \[\begin{aligned}
            \E W^6(T)&=15\E\int_0^TW^4(t)\diff t\\
            &=15\int_0^T\E W^4(t)\diff t\\
            &=15\int_0^T3t^2\diff t\\
            &=15T^3
        \end{aligned}\]

    \end{subproblem}

    \problem[习题4.8]
    \begin{subproblem}
        \item
        由韦萨切克方程知$R(t)$为伊藤过程,因此由伊藤引理
        \[\diff (\e^{\beta t}R(t))=\beta\e^{\beta t}R(t)\diff t
        +\e^{\beta t}R(t)\diff R(t)\]
        这里注意到$\partial^2(\e^{\beta tR(t)})/\partial R(t)^2=0$。
        从而代入$\diff R(t)$即得
        \begin{equation}
            \label{eq: deR}
            \diff (\e^{\beta t}R(t))=\e^{\beta t}(\alpha\diff t+\sigma\diff W(t))
        \end{equation}

        \item
        \cref{eq: deR}积分即得
        \[\begin{aligned}
            \e^{\beta t}R(t)-R(0)&=\alpha\int_0^t\e^{\beta s}\diff s
            +\sigma\int_0^t\e^{\beta s}\diff W(s)\\
            &=\frac{\alpha}{\beta}(\e^{\beta t}-1)+\sigma\int_0^t\e^{\beta s}\diff W(s)
        \end{aligned}\]
        即
        \[R(t)=\e^{-\beta t}R(t)+\frac{\alpha}{\beta}(1-\e^{-\beta t})
        +\sigma\e^{-\beta t}\int_0^t\e^{\beta s}\diff W(s)\]
    \end{subproblem}
\end{document}