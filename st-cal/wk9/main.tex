\documentclass[cn]{homework}

\title{第九周作业}

\newcommand{\E}{\mathbb E}

\begin{document}
    \maketitle

    \problem[习题3.6]
    \begin{subproblem}
        \item
        \begin{proof}
            由于
            \[x+X(t)-X(s)\sim\mathcal N(x+\mu(t-s),t-s)\]
            则显然$g(x)$可以改写为
            \[g(x)=\E(x+X(t)-X(s))\]
            考虑到$X(t)-X(s)$与$X(s)$独立,则由独立性引理
            \[\E[f(X(t))|\mathcal F(s)]=\E[f(X(t)-X(s)+X(s))|\mathcal F(s)]
            =g(X(s))\]
            即证。
        \end{proof}

        \item
        \begin{proof}
            
        \end{proof}
    \end{subproblem}
\end{document}