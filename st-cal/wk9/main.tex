\documentclass[cn]{homework}

\title{第九周作业}

\newcommand{\E}{\mathbb E}

\begin{document}
    \maketitle

    \problem[习题3.6]
    \begin{subproblem}
        \item
        \begin{proof}
            由于
            \[x+X(t)-X(s)\sim\mathcal N(x+\mu(t-s),t-s)\]
            则显然$g(x)$可以改写为
            \[g(x)=\E[x+X(t)-X(s)]\]
            考虑到$X(t)-X(s)$与$\mathcal F(s)$独立,$X(s)$为
            $\mathcal F(s)$~-可测,则由独立性引理
            \[\E[f(X(t))|\mathcal F(s)]=\E[f(X(t)-X(s)+X(s))|\mathcal F(s)]
            =g(X(s))\]
            即证。
        \end{proof}

        \item
        \begin{proof}
            \newcommand{\npdf}[2]{\frac{\e^{-\frac{{#1}^2}{2 #2}}}{\sqrt{2\pi #2}}}
            由于
            \[\sigma(W(t)-W(s))+\nu\tau\sim\mathcal N(\nu\tau,\sigma^2\tau)\]
            则对于$x>0$,
            \[\E[f(x\e^{\sigma(W(t)-W(s))+\nu\tau})]
            =\int_{-\infty}^\infty
            f(x\e^\xi)
            \npdf{(\xi-\nu\tau)}{\sigma^2\tau}
            \diff\xi\]
            考虑变换$x\e^\xi=y$,则上述积分变为
            \[\begin{aligned}
            \int_{-\infty}^\infty
            f(x\e^\xi)
            \npdf{(\xi-\nu\tau)}{\sigma^2\tau}
            \diff\xi
            &=
            \int_{0}^\infty
            f(y)\npdf{\left(\log\frac{y}{x}-\nu\tau\right)}{\sigma^2\tau}
            \frac{\diff y}y\\
            &=\int_0^\infty f(y)p(\tau,x,y)\diff y
            \end{aligned}\]
            于是
            \[g(x)=\E[f(x\e^{\sigma(W(t)-W(s))+\nu\tau})]\]
            而由于$\e^{\sigma(W(t)-W(s))+\nu\tau}$与$\mathcal F(s)$独立而$S(s)$为
            $\mathcal F(s)$~-可测,则由独立性引理,
            \[\E[f(S(t))|\mathcal F(s)]=\E[f(S(s)\e^{\sigma(W(t)-W(s))+\nu\tau})|\mathcal F(s)]
            =g(S(s))\]
            从而$S$具有Markov性,转移密度为$p(\tau,x,y)$。
        \end{proof}
    \end{subproblem}

    \problem[习题3.7]
    \begin{subproblem}
        \item
        \begin{proof}
            由于
            \[Z(t)=Z(s)\e^{\sigma(W(t)-W(s))-\frac{\sigma^2}{2}(t-s)}\]
            而$Z(s)$为$\mathcal F(s)$~-可测,$W(t)-W(s)$与$\mathcal F(s)$独立,
            故
            \[\E[Z(t)|\mathcal F(s)]
            =Z(s)\cdot\E\e^{\sigma(W(t)-W(s))-\frac{\sigma^2}{2}(t-s)}\]
            而$W(t)-W(s)\sim\mathcal N(0,t-s)$,有
            \[\E\e^{\sigma(W(t)-W(s))}
            =\e^{\frac{\sigma^2(t-s)}{2}}\]
            因此
            \[\E[Z(t)|\mathcal F(s)]=Z(s)\]
            即$Z(t)$为鞅。
        \end{proof}

        \item
        \begin{proof}
            由于$Z(t)$为鞅,故其停止过程$Z(t\land\tau_m)$亦为鞅,
            因此
            \[\E[Z(t\land\tau_m)]=Z(0\land\tau_m)=Z(0)=1\]
        \end{proof}

        \item
        \begin{proof}
            由于当$\tau_m<\infty$时,则有逐点收敛
            \[\lim_{t\to\infty}Z(t\land\tau_m)
            =\e^{\sigma m-\left(\sigma\mu+\frac{\sigma^2}{2}\right)\tau_m}\]
            而当$\tau_m=\infty$时,则由$\sigma> 0,\mu\geq 0$
            \[Z(t\land\tau_m)=Z(t)\leq
            \e^{\sigma m-\left(\sigma\mu+\frac{\sigma^2}{2}\right)t}
            \to 0\]
            因此对于任意路径都有
            \[\lim_{t\to\infty}Z(t\land\tau_m)
            =\e^{\sigma m-\left(\sigma\mu+\frac{\sigma^2}{2}\right)\tau_m}
            \mathbb I_{\{\tau_m<\infty\}}\]
            于是
            \[\E\left[
            \e^{\sigma m-\left(\sigma\mu+\frac{\sigma^2}{2}\right)\tau_m}
            \mathbb I_{\{\tau_m<\infty\}}\right]
            =\E\left[\lim_{t\to\infty}Z(t\land\tau_m)\right]\]
            而对于任意路径
            \[|Z(t\land\tau_m)|\leq\e^{\sigma m}\]
            因此由控制收敛定理
            \[\E\left[\lim_{t\to\infty}Z(t\land\tau_m)\right]
            =\lim_{t\to\infty}\E[Z(t\land\tau_m)]=1\]
            即
            \[\E\left[
                \exp\left\{
                    \sigma m-\left(\sigma\mu+\frac{\sigma^2}{2}\right)\tau_m
                    \right\}
                \mathbb I_{\{\tau_m<\infty\}}
            \right]=1\]
            移项得
            \begin{equation}
                \label{eq:laplace}
                \E\left[\e^{-\left(\sigma\mu+\frac{\sigma^2}{2}\right)\tau_m}
                \mathbb I_{\{\tau_m<\infty\}}\right]
                =\e^{-\sigma m}
            \end{equation}
            而对于任意给定路径,
            \[\e^{-\left(\sigma\mu+\frac{\sigma^2}{2}\right)\tau_m}
            \mathbb I_{\{\tau_m<\infty\}}\]
            关于$\sigma$单调,因此由单调收敛定理,
            \[\begin{aligned}
                \E\left[
                \lim_{\sigma\to0^+}
                \e^{-\left(\sigma\mu+\frac{\sigma^2}{2}\right)\tau_m}
                \mathbb I_{\{\tau_m<\infty\}}
            \right]&=
            \lim_{\sigma\to0^+}
            \E\left[\e^{-\left(\sigma\mu+\frac{\sigma^2}{2}\right)\tau_m}
            \mathbb I_{\{\tau_m<\infty\}}\right]\\
            &=\lim_{\sigma\to0^+}\e^{-\sigma m}\\
            &=1\\
            \end{aligned}\]
            而
            \[\lim_{\sigma\to0^+}
            \e^{-\left(\sigma\mu+\frac{\sigma^2}{2}\right)\tau_m}
            \mathbb I_{\{\tau_m<\infty\}}
            =\mathbb I_{\{\tau_m<\infty\}}\]
            因此
            \[\mathbb P\{\tau_m<\infty\}
            =\E\left[\mathbb I_{\{\tau_m<\infty\}}\right]=1\]
            从而\cref{eq:laplace}变为
            \[\E\left[\e^{-\left(\sigma\mu+\frac{\sigma^2}{2}\right)\tau_m}\right]
            =\e^{-\sigma m}\]
            令$\alpha=\sigma\mu+\sigma^2/2>0$,则由$\sigma>0$可解得
            \[\alpha=-\mu+\sqrt{\mu^2+2\alpha}\]
            因此拉普拉斯变换为
            \[\E\left[\e^{-\alpha\tau_m}\right]
            =\e^{m\mu-m\sqrt{\mu^2+2\alpha}}\]
            
        \end{proof}
    \end{subproblem}
\end{document}