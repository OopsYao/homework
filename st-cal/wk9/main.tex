\documentclass[cn]{homework}

\title{第九周作业}

\newcommand{\E}{\mathbb E}

\begin{document}
    \maketitle

    \problem[习题3.6]
    \begin{subproblem}
        \item
        \begin{proof}
            由于
            \[x+X(t)-X(s)\sim\mathcal N(x+\mu(t-s),t-s)\]
            则显然$g(x)$可以改写为
            \[g(x)=\E[x+X(t)-X(s)]\]
            考虑到$X(t)-X(s)$与$\mathcal F(s)$独立,$X(s)$为
            $\mathcal F(s)$~-可测,则由独立性引理
            \[\E[f(X(t))|\mathcal F(s)]=\E[f(X(t)-X(s)+X(s))|\mathcal F(s)]
            =g(X(s))\]
            即证。
        \end{proof}

        \item
        \begin{proof}
            \newcommand{\npdf}[2]{\frac{\e^{-\frac{{#1}^2}{2 #2}}}{\sqrt{2\pi #2}}}
            由于
            \[\sigma(W(t)-W(s))+\nu\tau\sim\mathcal N(\nu\tau,\sigma^2\tau)\]
            则对于$x>0$,
            \[\E[f(x\e^{\sigma(W(t)-W(s))+\nu\tau})]
            =\int_{-\infty}^\infty
            f(x\e^\xi)
            \npdf{(\xi-\nu\tau)}{\sigma^2\tau}
            \diff\xi\]
            考虑变换$x\e^\xi=y$,则上述积分变为
            \[\begin{aligned}
            \int_{-\infty}^\infty
            f(x\e^\xi)
            \npdf{(\xi-\nu\tau)}{\sigma^2\tau}
            \diff\xi
            &=
            \int_{0}^\infty
            f(y)\npdf{\left(\log\frac{y}{x}-\nu\tau\right)}{\sigma^2\tau}
            \frac{\diff y}y\\
            &=\int_0^\infty f(y)p(\tau,x,y)\diff y
            \end{aligned}\]
            于是
            \[g(x)=\E[f(x\e^{\sigma(W(t)-W(s))+\nu\tau})]\]
            而由于$\e^{\sigma(W(t)-W(s))+\nu\tau}$与$\mathcal F(s)$独立而$S(s)$为
            $\mathcal F(s)$~-可测,则由独立性引理,
            \[\E[f(S(t))|\mathcal F(s)]=\E[f(S(s)\e^{\sigma(W(t)-W(s))+\nu\tau})|\mathcal F(s)]
            =g(S(s))\]
            从而$S$具有Markov性,转移密度为$p(\tau,x,y)$。
        \end{proof}
    \end{subproblem}
\end{document}