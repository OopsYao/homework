\documentclass[cn]{homework}

\title{第五周作业}

\DeclareMathOperator{\E}{E}
\newcommand{\Econd}[2]{\mathrm E\left[\left.#1\right|#2\right]}

\begin{document}
    \maketitle

    \problem
    \begin{proof}
        由条件期望定义有无偏性
        \[\E(X\cdot \E[Y|G])=\E(\E[X\cdot \E[Y|G]|G])\]
        而由提取已知量的性质,注意到$\E(Y|G)$是$G$可测的,
        \[\E[X\cdot \E[Y|G]|G]=\E[Y|G]\cdot \E[X|G]\]
        因此
        \[\E(X\cdot \E[Y|G])=\E(\E[Y|G]\cdot \E[X|G])\]
        同理由对称性可得
        \[\E(Y\cdot \E[X|G])=\E(\E[X|G]\cdot \E[Y|G])\]
        故
        \[\E(X\cdot \E[Y|G])=\E(Y\cdot \E[X|G])\]
    \end{proof}

    \problem
    \begin{subproblem}
        \item
        % TODO 2.(1)

        \item
        \begin{proof}
            由于
            \[\begin{aligned}
                \E(\E[X|Y])&=Y\\
            \end{aligned}\]
            因此对于任意$A\in\sigma(Y)$,
            \[\int_AX\diff P=\int_AY\diff P\]
            同样地由于$\E[Y|X]=X$,上式对于任意
            $A\in\sigma(X)$亦成立,

            考虑事件$B=\{\omega\in\Omega;X(\omega)<Y(\omega)\}$,
            和事件
            \[B'=\bigcup_{q\in\mathbb Q}\{\omega\in\Omega;X(\omega)<q<Y(\omega)\}\]
            对于任意$\omega\in B'$,显然$\omega\in B$,反过来,对于任意$\omega\in B$,
            由有理数在实数中的稠密性,总有$q\in\mathbb Q$,使得
            \[X(\omega)<q<Y(\omega)\]
            因此$\omega\in B'$,即$B=B'$。
            而注意到$\{\omega\in\Omega;X(\omega)<q\}\in\sigma(X)$
            
            % TODO Something goes wrong ...
        \end{proof}
    \end{subproblem}

    \problem
    \begin{subproblem}
        \item
        \begin{proof}
            先证明必要性,对于任意$G$-可测随机变量$X$,和其中的事件$A\in\sigma(X)$,
            存在Borel集$S$,满足
            \[A=X^{-1}(S)\]
            而$X$是$G$-可测的,于是
            \[A\in G\]
            因此$\sigma(X)\subset G$,同理可得$\sigma(Y)\subset H$,
            故$G,H$独立蕴含$X,Y$独立。

            对于必要性,考虑任意事件$A\in G$与$B\in H$,
            我们下面证明总存在$G$-可测随机变量$X$与$H$-可测随机变量$Y$,
            使得$A\in\sigma(A),B\in\sigma(Y)$。
            不妨设
            $X$为$A$的特征函数,$Y$为$B$的特征函数,显然它们分别是$G,H$-可测的,
            \[X=\begin{cases}
                1,&\omega\in A\\
                0,&\omega\not\in A
            \end{cases}\]
            于是存在
            \[S=\{1\}=[0,1]\cap[1,2]\in\mathcal B(\mathbb R)\]
            使得
            \[A=X^{-1}(S)\]
            故$A\in\sigma(X)$。同理有$B\in\sigma(Y)$,
            而由$X,Y$独立可知
            \[P(A\cap B)=P(A)\cdot P(B)\]
            因此由$A,B$的任意性知$G,H$独立。
        \end{proof}

        \item
        \begin{proof}
            由独立性引理知必要性是显然的,对于充分性,
            考虑任意$G$-可测随机变量$Y$,
            提取已知量,
            \[\begin{aligned}
                \E[\e^{uX}\cdot\e^{vY}|G]&=\e^{vY}\cdot\E[\e^{uX}|G]\\
                &=\e^{vY}\cdot\E(\e^{uX})
            \end{aligned}\]
            因此由重期望
            \[\begin{aligned}
                \E(\e^{uX}\cdot\e^{vY})&=\E(\E[\e^{uX}\cdot\e^{vY}|G])\\
                &=\E(\e^{vY}\cdot\E(\e^{uX}))\\
                &=\E(\e^{vY})\cdot\E(\e^{uX})
            \end{aligned}\]
            故$X$与$Y$独立,
            于是对于任意$A\in\sigma(X)$与$B\in G$,皆可考虑$Y$为
            $G$-可测随机变量,$B$的特征函数,于是由上题可知,
            \[B\in\sigma(Y)\]
            再由$X$与$Y$的独立性知,$A$与$B$独立,于是$X$与$G$独立。
        \end{proof}
    \end{subproblem}
\end{document}