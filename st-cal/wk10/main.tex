\documentclass[cn]{homework}

\title{第十周作业}

\newcommand{\E}{\mathbb E}
\begin{document}
    \maketitle

    \problem[习题4.1]
    \begin{proof}
        对于$s\leq t$,不妨设$t_l\leq s<t_{l+1},t_m\leq t<t_{m+1}$,
        且$t_l\leq t_m$,则根据定义
        \begin{multline*}
            I(t)=I(s)+\sum_{k=l+1}^{m-1}\Delta(t_k)[M(t_{k+1})-M(t_k)]\\
            +\Delta(t_l)[M(s)-M(t_l)]+\Delta(t_m)[M(t)-M(t_m)]
        \end{multline*}
        由于$\Delta(t)$适应于$\mathcal F(t)$,$M(t)$为鞅(从而也适应于
        $\mathcal F(t)$),从而有
        \[\begin{aligned}
            &\E[\Delta(t_k)(M(t_{k+1})-M(t_k))\mid\mathcal F(s)]\\
            =&\E\{\E[\Delta(t_k)M(t_{k+1})-M(t_k))\mid\mathcal F(t_k)]\mid \mathcal F(s)\}\\
            =&\E\{\Delta(t_k)\cdot \E[M(t_{k+1})-M(t_k)\mid\mathcal F(t_k)]\mid\mathcal F(s)\}\\
            =&0\\
            &\E[\Delta(t_l)(M(s)-M(t_l))\mid\mathcal F(s)]\\
            =&\Delta(t_l)\cdot\E[M(s)-M(t_l)\mid\mathcal F(s)]\\
            =&0\\
            &\E[\Delta (t_m)(M(t)-M(t_m))\mid\mathcal F(s)]\\
            =&\E\{\E[\Delta(t_m)(M(t)-M(t_m)\mid\mathcal F(\max\{s,t_m\})]\mid F(s)\}\\
            =&\E\{\Delta(t_m)\cdot\E[M(t)-M(t_m)\mid\mathcal F(\max\{s,t_m\})]\mid F(s)\}\\
            =&0
        \end{aligned}\]
        注意到在上面的推导中我们利用了
        \[\E\{\E[X\mid \mathcal F]\mid\mathcal G\}=\E[X\mid\mathcal G],
        \mathcal G\subset \mathcal F\]
        于是
        \[\begin{aligned}
            \E[I(t)\mid\mathcal F(s)]=I(s)
        \end{aligned}\]
        即$I(s)$为鞅。
    \end{proof}

    \problem[习题4.2]
    \begin{subproblem}
        \item
        \label{spb:independence}
        \begin{proof}
            对于分点$t_k<t_l$,有
            \begin{equation}
                \label{eq:delta I}
                I(t_l)-I(t_k)=\sum_{j=l}^{k-1}\Delta(t_j)[W(t_{j+1})-W(t_j)]
            \end{equation}
            而由于$\Delta(t_j)$非随机,$W(t_{j+1})-W(t_j),j\geq l$独立于$\mathcal F(t_k)$,
            故$I(t_l)-I(t_k)$独立于$\mathcal F(t_k)$,从而$I(t)-I(s)$独立于$\mathcal F(s)$。
        \end{proof}

        \item
        \label{spb:normal}
        \begin{proof}
            也只需证明对于任意分点$t_k<t_l$的情况。
            由于$W(t_{j+1})-W(t_j)$相互独立,故
            由\cref{eq:delta I}知,
            \[I(t_l)-I(t_k)\sim\mathcal N\left(0,\sum_{j=l}^{k-1}(\Delta(t_j))^2(t_{j+1}-t_j)\right)\]
            而注意到
            \[\int_{t_l}^{t_k}\Delta^2(u)\diff u=\sum_{j=l}^{k-1}(\Delta(t_j))^2(t_{j+1}-t_j)\]
            故有题述结论。
        \end{proof}

        \item
        \begin{proof}
            由\ref{spb:independence}的独立性和\ref{spb:normal}知,
            \[\E[I(t)-I(s)\mid\mathcal F(s)]=\E[I(t)-I(s)]=0\]
            而$I(s)$是$\mathcal F(s)$可测的,因此$I(t)$为鞅。
        \end{proof}
        
        \item
        \begin{proof}
            由于$I(t)-I(s)$独立于$\mathcal F(s)$,$I(s)$为$\mathcal F(s)$可测,
            \[\begin{aligned}
                &\E[I^2(t)\mid\mathcal F(s)]\\
                =&\begin{aligned}[t]
                &\E[(I(t)-I(s))^2\mid\mathcal F(s)]\\
                &+2\E[I(s)(I(t)-I(s))\mid\mathcal F(s)]
                +\E[I^2(s)\mid\mathcal F(s)]
                \end{aligned}\\
                =&\E[(I(t)-I(s))^2]+2\E[I(s)]\cdot\E[I(t)-I(s)]
                +I^2(s)\\
                =&\int_s^t\Delta^2(u)\diff u+I^2(s)
            \end{aligned}\]
            从而
            \[\E\left[\left.I^2(t)-\int_0^t\Delta^2(u)\diff u\right|\mathcal F(s)\right]
            =I^2(s)-\int_0^s\Delta^2(u)\diff u\]
            即$I^2(t)-\int_0^t\Delta^2(u)\diff u$为鞅。
        \end{proof}
    \end{subproblem}
\end{document}