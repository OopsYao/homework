\documentclass[cn]{homework}

\title{作业2}

\DeclareMathOperator{\prob}{\mathbb P}

\begin{document}
    \maketitle

    \section{题1}
    \begin{subproblem}
        \item
        \begin{proof}
            由于
            \begin{align*}
                \prob(A)+\prob(A^c)&=
                \sum_{\omega\in A}\prob(\omega)+\sum_{\omega\in A^c}\prob(\omega)\\
                &=\sum_{\omega\in\Omega}P(\omega)\\
                &=1
            \end{align*}
            即证。
        \end{proof}

        \item
        \begin{proof}
            先证明$N=2$的情况,将$A_1\cup A_2$分解为不交并
            \[A_1\cup A_2=((A_1\cup A_2)\backslash A_2)\cup A_2\]
            则
            \begin{align*}
                \prob(A_1\cup A_2)&=\prob((A_1\cup A_2)\backslash A_2)
                +\prob(A_2)
            \end{align*}
            又由于
            \[(A_1\cup A_2)\backslash A_2\subset A_1\]
            则
            \[\prob((A_1\cup A_2)\backslash A_2)\leq\prob(A_1)\]
            因此
            \[\prob(A_1\cup A_2)\leq\prob(A_1)+\prob(A_2)\]
            于是$N=2$的情况得证。

            对于$N>2$的情况,由于
            \[\bigcup_{i=1}^NA_i=\left(\bigcup_{i=1}^{N-1}A_i\right)\cup A_N\]
            以及
            \[\sum_{i=1}^N\prob(A_i)=\prob(A_N)+\sum_{i=1}^{N-1}\prob(A_i)\]
            则由数学归纳法自然证得。
        \end{proof}
    \end{subproblem}
\end{document}