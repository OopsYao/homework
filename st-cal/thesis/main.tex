\documentclass[12pt]{ctexart}

\usepackage{amsmath}
\usepackage{amsfonts}
\usepackage{amsthm}
\usepackage{hyperref}

\def\diff{\mathrm{d}}
\def\E{\mathbb{E}}
\def\prob{\mathbb{P}}

\newtheorem*{theorem}{定理}

\title{欧式看涨期权的风险中性定价方法}
\date{}

\begin{document}
    \maketitle

    \section{背景介绍}
    期权是一种重要的金融产品,但由于其价值依赖于原生资产如股票等,
    而这些原生资产的价格往往是随机的,定价成为一个比较重要的问题。
    但我们可以利用$\Delta$-对冲或者说是复制资产的方法,为期权定价。

    原生资产的价格经常被假定为适应于布朗运动的过程$S(t)$,为几何布朗运动,
    \begin{equation}
        \label{eq:stock price}
        \diff S(t)=\alpha(t)S(t)\diff t+\sigma(t)S(t)\diff W(t)
    \end{equation}
    关于资产的贴现率$D(t)$,我们假定其满足微分方程
    \begin{equation}
        \label{eq:discount}
        \diff D(t)=-R(t)D(t)\diff t
    \end{equation}
    
    我们的目标在于知道终期价值($T$时刻)的情况下,推导出欧式看涨期权
    在任意时刻的价值$V(t)$
    (知道$t$时刻及其以前的信息,比如当时的股价$S(t)$)。

    \section{风险中性定价方法}
    在针对欧式看涨期权定价之前,我们可以探索一般的衍生证券的定价方法。
    其基本思想依然是复制资产,即$\Delta$-对冲。

    \subsection[Delta-对冲方法]{$\Delta$-对冲方法}
    假设我们有货币与股票两个账户,
    在$t$时刻,若我们持有的股票头寸为$\Delta t$,
    于是货币账户头寸为$X(t)-\Delta(t)S(t)$,
    则我们总资产的增值由股票的增值与货币利息组成,即
    \begin{equation}
        \label{eq:asset value}        
        \diff X(t)=\Delta(t)\diff S(t)+R(t)(X(t)-\Delta(t)S(t))\diff t
    \end{equation}
    特别地,当$V(T)=X(T)$
    倘若我们能证明,我们这样构造的资产与衍生证券的回报是一样的,
    那么根据无套利原理
    % TODO Relationship between X(t) and V(t), why do we construct X(t)?

    \subsection{贴现资产}
    下面我们将会展示,资产的贴现过程$D(t)X(t)$经过合适的测度变换会成为该测度下的鞅。
    注意到$D(t)$与$X(t)$都是伊藤过程,故由式(\ref{eq:discount})知
    \[\begin{aligned}
        \diff(D(t)X(t))&=D(t)\diff X(t)+X(t)\diff D(t)\\
        &=D(t)(\diff X(t)-R(t)X(t)\diff t)
    \end{aligned}\]
    进一步我们根据式(\ref{eq:asset value})拆解$\diff X(t)$可得
    \[\diff(D(t)X(t))=\Delta(t)D(t)(\diff S(t)-R(t)S(t)\diff t)\]
    最后根据式(\ref{eq:stock price})拆解$\diff S(t)$可得
    \begin{equation}
        \label{eq:asset discount}
        \begin{aligned}
        \diff (D(t)X(t))&=\Delta(t)D(t)S(t)[(\alpha(t)-R(t))\diff t+\sigma(t)\diff W(t)]\\
        &=\sigma(t)\Delta(t)D(t)S(t)(\Theta(t)\diff t+\diff W(t))
        \end{aligned}
    \end{equation}
    这里
    \[\Theta(t)=\frac{\alpha(t)-R(t)}{\sigma(t)}\]
    被称为风险的市场价格。
    于是我们得到了资产价值贴现过程的微分形式,但其中含有$\diff t$项,
    难以看出$D(t)X(t)$的显式格式。现在我们希望$\Theta(t)\diff t+\diff W(t)$
    能在一个新的测度下成为一个布朗运动的微分,这样在新测度下贴现过程$D(t)X(t)$
    成为了一个鞅,具有许多良好的性质。

    \subsection{哥萨诺夫定理}
    布朗运动加上漂移项之后不再是布朗运动,
    但哥萨诺夫定理说明在一定的条件下,在根据漂移项所对应的一个新测度下,
    该过程是一个布朗运动,达到了去掉漂移项的效果。

    \begin{theorem}[哥萨诺夫]
        假如随机过程$W(t),0\leq t\leq T$是概率空间$(\Omega,\mathcal F,\prob)$上的
        布朗运动,$\mathcal F(t)$是该布朗运动所生成的域流。设$\Theta(t),0\leq t\leq T$
        是适应于$\mathcal F(t)$的随机过程,定义如下的随机过程$\tilde W(t),Z(t)$,
        \[\begin{aligned}
            \diff\tilde W(t)&=\Theta(t)\diff t+\diff W(t)\\
            \diff Z(t)&=-\Theta(t)Z(t)\diff W(t)
        \end{aligned}\]
        这里$\tilde W(t),Z(t)$满足初值条件$W(0)=0,Z(0)=1$,
        且满足
        \[\E[\int_0^T\Theta^2(u)Z^2(u)\diff u]<\infty\]
        
        则在$Z=Z(T)$所诱导的概率测度$\tilde\prob$下(不难验证$\E Z=1$),
        $\tilde W(t)$是布朗运动。
    \end{theorem}

    于是我们令
    \begin{equation}
        \label{eq:transform}
        \diff\tilde W(t)=\Theta(t)\diff t+\diff W(t)
    \end{equation}
    从而式(\ref{eq:asset discount})变为
    \[\diff(D(t)X(t))=\sigma(t)\Delta(t)D(t)S(t)\diff\tilde W(t)\]
    则根据哥萨诺夫定理,$\tilde W(t)$是$\tilde\prob$下的布朗运动,
    从而$D(t)X(t)$是$\tilde\prob$下的鞅,事实上$\tilde\prob$也就是所谓的
    风险中性测度,
    于是
    \[D(t)X(t)=\tilde\E[D(T)X(T)|\mathcal F(t)]\]
    这里$\tilde\E$是测度$\prob$下的期望。
    结合无套利原理$X(t)=V(t)$我们知道
    \[D(t)V(t)=\tilde\E[D(T)V(T)|\mathcal F(t)]\]
    这就是衍生证券的风险中性定价公式。

    \section{欧式看涨期权}
    欧式期权是一种买入方必须在期权到期日当天才能行使的期权。
    对于欧式看涨期权来说,其到期日的支付为
    \[V(T)=(S(T)-K)^+\]
    这里$K$是期权的约定价格。
    为了求出定价公式,我们需要将$V(T)$表示成关于$\tilde W(t),0\leq t\leq T$的表达式,
    通过代换式(\ref{eq:transform}),原生资产(股价)可以从式(\ref{eq:stock price})
    被表示为
    \[\diff S(t)=R(t)\diff t+\sigma(t)\diff\tilde W(t)\]
    特别地,为简单起见我们令$\sigma(t)=\sigma,R(t)=r$为常数,于是上式可以
    得到显式解,从而在终止时刻
    \[S(T)=S(t)\exp\left\{\sigma(\tilde W(T)-\tilde W(t))+(r-\sigma^2/2)(T-t))\right\}\]
    由于布朗运动的独立增量,$\tilde W(T)-\tilde W(t)$独立于$\mathcal F(t)$,
    而$S(t)$为$\mathcal F(t)$可测,$D(T)$是退化的(非随机)故由独立性引理,
    $V(t)$可表示为
    \[D(t)V(t)=D(T)\mu(t,S(t))\]
    这里
    \begin{equation}
        \label{eq:mu}
        \begin{aligned}
            \mu(t,s)&=\tilde\E[(s\exp(\sigma Y+(r-\sigma^2/2)\tau)-K)^+]\\
            \tau&=T-t\\
            Y&=\tilde W(T)-\tilde W(t)\sim\mathcal N(0,\tau)
        \end{aligned}
    \end{equation}
    求解式(\ref{eq:discount})可以得到
    \[\frac{D(T)}{D(t)}=\exp(-r\tau)\]
    从而最终我们得到
    \[V(t)=\exp(-r\tau)\mu(t,S(t))\]
    这里$\mu(t,s)$由式(\ref{eq:mu})决定。

    通过进一步的计算可以得到式(\ref{eq:mu})的具体表达式,
    \[\mu(t,s)=s\exp(r\tau)N(d_+(\tau,s))-KN(d_-(\tau,s))\]
    其中$N(s)$为标准正态分布的累积分布函数,$d_\pm(\tau,s)$定义为
    \[d_\pm (\tau,s)=\frac{1}{\sigma\sqrt\tau}
    \left(\ln\frac{s}{K}+(r\pm\sigma^2/2)\tau\right)\]
    从而有BSM公式
    \[\nu(t,s)=\exp(-r\tau)\mu(t,s)=xN(d_+(\tau,s))-K\exp(-r\tau)N(d_-(\tau,s))\]
\end{document}