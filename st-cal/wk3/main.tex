\documentclass[cn]{homework}

\title{第三周作业}

\DeclareMathOperator{\prob}{\mathbb P}

\begin{document}
    \maketitle

    \section{习题2.1}
    \begin{proof}
        考虑到$\Omega$非空,则存在$\omega_0\in\Omega$,
        设$c=X(\omega_0)$。考虑$\mathbb R$中的Borel集
        \[\{c\}=[c-1,c]\cap[c,c+1]\]
        则由于$X$为$\mathcal F_0$可测的,因此
        \[X^{-1}(\{c\})\in\mathcal F_0\]
        注意到有$\omega_0\in X^{-1}(\{c\})$,因此
        \[X^{-1}=\Omega\]
        即对于任意$\omega\in\Omega$,
        \[X(\omega)=c\]
        故$X$非随机。
    \end{proof}

    \section{习题2.2}
    \begin{subproblem}
        \item
        \[\sigma(X)=\left\{\varnothing,\Omega_2,\{HT,TH\},\{HH,TT\}\right\}\]

        \item
        \[\sigma(S_1)=\left\{\varnothing,\Omega_2,\{HH,HT\},\{TH,TT\}\right\}\]

        \item
        \renewcommand{\prob}{\tilde{\mathbb P}}
        \begin{proof}
            只需证明对于任意$A\in\sigma(X),B\in\sigma(S_1)$,
            有
            \[\prob(A\cap B)=\prob(A)\cdot \prob(B)\]
            即可。对于$A,B$其一为$\varnothing,\Omega_2$时显然成立。
            故只需验证其他情况,
            \[\begin{aligned}
                \prob(\{HT,TH\}\cap\{HH,HT\})&=\prob(\{HT\})\\
                &=\frac{1}{4}\\
                &=\prob(\{HT,TH\})\cdot\prob(\{HH,HT\})\\
                \prob(\{HT,TH\}\cap\{TH,TT\})
                &=\prob(\{TH\})\\
                &=\frac{1}{4}\\
                &=\prob(\{HT,TH\})\cdot\prob(\{TH,TT\})\\
                \prob(\{HH,TT\}\cap\{HH,HT\})
                &=\prob(\{HH\})\\
                &=\frac{1}{4}\\
                &=\prob(\{HH,TT\})\cdot\prob(\{HH,HT\})\\
                \prob(\{HH,TT\}\cap\{TH,TT\})
                &=\prob(\{TT\})\\
                &=\frac{1}{4}\\
                &=\prob(\{HH,TT\})\cdot\prob(\{TH,TT\})
            \end{aligned}\]
            故$\sigma(X),\sigma(S_1)$是独立的。
        \end{proof}        

        \item
        \renewcommand{\prob}{\mathbb P}
        \begin{proof}
            由于
            \[\prob(\{HT,TH\}\cap\{HH,HT\})=\prob(\{HT\})=\frac{2}{9}\]
            而
            \[\prob(\{HT,TH\})\cdot\prob(\{HH,HT\})
            =\frac{4}{9}\cdot\frac{6}{9}
            \neq\frac{2}{9}\]
            因此$\sigma(X)$与$\sigma(S_1)$不是独立的。
        \end{proof}

        \item
        由于$X$与$S_1$之间不独立,因此$X$的信息将有助于我们确定$S_1$。
    \end{subproblem} 
\end{document}