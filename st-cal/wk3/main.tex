\documentclass[cn]{homework}

\title{第三周作业}

\DeclareMathOperator{\prob}{\mathbb P}

\begin{document}
    \maketitle

    \problem
    \begin{subproblem}
        \item
        \begin{proof}
            由于
            \begin{align*}
                \prob(A)+\prob(A^c)&=
                \sum_{\omega\in A}\prob(\omega)+\sum_{\omega\in A^c}\prob(\omega)\\
                &=\sum_{\omega\in\Omega}P(\omega)\\
                &=1
            \end{align*}
            即证。
        \end{proof}

        \item
        \begin{proof}
            先证明$N=2$的情况,将$A_1\cup A_2$分解为不交并
            \[A_1\cup A_2=((A_1\cup A_2)\backslash A_2)\cup A_2\]
            则
            \begin{align*}
                \prob(A_1\cup A_2)&=\prob((A_1\cup A_2)\backslash A_2)
                +\prob(A_2)
            \end{align*}
            又由于
            \[(A_1\cup A_2)\backslash A_2\subset A_1\]
            则
            \[\prob((A_1\cup A_2)\backslash A_2)\leq\prob(A_1)\]
            因此
            \[\prob(A_1\cup A_2)\leq\prob(A_1)+\prob(A_2)\]
            于是$N=2$的情况得证。

            对于$N>2$的情况,由于
            \[\bigcup_{i=1}^NA_i=\left(\bigcup_{i=1}^{N-1}A_i\right)\cup A_N\]
            以及
            \[\sum_{i=1}^N\prob(A_i)=\prob(A_N)+\sum_{i=1}^{N-1}\prob(A_i)\]
            则由数学归纳法自然证得。
        \end{proof}
    \end{subproblem}

    \problem
    \begin{subproblem}
        \item
        $S_3$分布如下
        \[\begin{aligned}
            &\prob(S_3=32)
            =\prob\left(S_3=\frac{1}{2}\right)
            =\frac{1}{8}\\
            &\prob(S_3=8)
            =\prob(S_3=2)
            =\frac{3}{8}\\
        \end{aligned}\]

        \item
        \newcommand{\E}{\tilde{\mathbb E}}
        期望如下,
        \[\begin{aligned}
            \E S_1&=\frac{8+2}{2}=5\\
            \E S_2&=\frac{16+4\cdot 2+1}{2^2}=\frac{25}{4}\\
            \E S_3&=\frac{32+8\cdot 3+2\cdot 3+\frac{1}{2}}{2^3}
            =\frac{125}{16}
        \end{aligned}\]

        \item
        \newcommand{\preal}{\mathbb P}
        $S_3$分布如下
        \[\begin{aligned}
            \preal(S_3=32)&=p^3=\frac{8}{27}\\
            \preal(S_3=8)&=3p^2q=\frac{4}{9}\\
            \preal(S_3=2)&=3pq^2=\frac{2}{9}\\
            \preal\left(S_3=\frac{1}{2}\right)&=q^3=\frac{1}{27}\\
        \end{aligned}\]
        实际期望
        \renewcommand{\E}{\mathbb E}
        \[\begin{aligned}
            \E S_1&=8p+2q=6\\
            \E S_2&=16p^2+4\cdot 2pq+q^2=9\\
            \E S_3&=32p^3+8\cdot 3p^2q+2\cdot 3pq^2+\frac{1}{2}q^3
            =\frac{27}{2}
        \end{aligned}\]
    \end{subproblem}
\end{document}