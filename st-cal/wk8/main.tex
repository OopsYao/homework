\documentclass[cn]{homework}

\title{第八周作业}

\newcommand{\E}{\mathrm E}

\begin{document}
    \maketitle
    
    \section{习题3.2}
    \newcommand{\filt}{\mathcal F(s)}
    \begin{proof}
        考虑$0\leq s\leq t$,
        由于
        \[W^2(t)=(W(t)-W(s))^2+2W(t)W(s)-W^2(s)\]
        于是
        \[\E(W^2(t)|\filt)=
        \E(W(t)-W(s))^2+2W(s)\cdot\E(W(t)|\filt)-W^2(s)\]
        考虑到布朗运动为鞅,即
        \[\E(W(t)|\filt)=W(s)\]
        于是
        \[\E(W^2(t)|\filt)=t-s+2W^2(s)-W^2(s)=W^2(s)+t-s\]
        因此
        \[\E(W^2(t)-t|\filt)=W^2(s)-s\]
        即$W^2(t)-t$为鞅。
    \end{proof}

    \section{习题3.4}
    \begin{subproblem}
        \newcommand{\gap}{|W(t_{j+1})-W(t_j)|}
        \newcommand{\sumhead}{\sum_{j=1}^{n-1}}
        \newcommand{\limhead}{\lim_{\tau\to0^+}}
        \item
        \begin{proof}
            反证法。记$\tau=\max_j|t_{j+1}-t_j|$,
            倘若存在非零测的路径集合,其中路径
            一阶变差有限,即
            \[\limhead\sumhead\gap<\infty\]
            则由于
            \[\sumhead\gap^2\leq\tau\sumhead\gap\]
            因此二阶变差
            \[\limhead\sumhead\gap^2=0\]
            这与布朗运动几乎所有路径二阶变差为$T$相矛盾,
            因此几乎所有路径一阶变差趋于$\infty$。
        \end{proof}

        \item
        \begin{proof}
            记号同上,同样地有
            \[\sumhead\gap^3\leq\tau\sumhead\gap^2\]
            而
            \[\limhead\sumhead\gap^2=T\quad\text{a.c.}\]
            于是两端取$\tau\to0^+$即得几乎所有路径三阶变差为0。
        \end{proof}
    \end{subproblem}
\end{document}