\documentclass[cn]{homework}

\title{第八周作业}

\newcommand{\E}{\mathbb E}

\begin{document}
    \maketitle
    
    \section{习题3.2}
    \newcommand{\filt}{\mathcal F(s)}
    \begin{proof}
        考虑$0\leq s\leq t$,
        由于
        \[W^2(t)=(W(t)-W(s))^2+2W(t)W(s)-W^2(s)\]
        于是
        \[\E(W^2(t)|\filt)=
        \E(W(t)-W(s))^2+2W(s)\cdot\E(W(t)|\filt)-W^2(s)\]
        考虑到布朗运动为鞅,即
        \[\E(W(t)|\filt)=W(s)\]
        于是
        \[\E(W^2(t)|\filt)=t-s+2W^2(s)-W^2(s)=W^2(s)+t-s\]
        因此
        \[\E(W^2(t)-t|\filt)=W^2(s)-s\]
        即$W^2(t)-t$为鞅。
    \end{proof}

    \section{习题3.3}
    由于
    \[\varphi^{(3)}(u)=\e^{\frac{\sigma^2u^2}{2}}(3\sigma^4u+\sigma^4u^2)\]
    故
    \[\varphi^{(4)}(u)=\sigma^2u\e^{\frac{\sigma^2u^2}{2}}(3\sigma^4u+\sigma^4u^2)
    +\e^{\frac{\sigma^2u^2}{2}}(3\sigma^4+2\sigma^4u)\]
    因此
    \[\E[(X-\mu)^4]=\varphi^{(4)}(0)=3\sigma^4\]

    \section{习题3.4}
    \begin{subproblem}
        \newcommand{\gap}{|W(t_{j+1})-W(t_j)|}
        \newcommand{\sumhead}{\sum_{j=1}^{n-1}}
        \newcommand{\limhead}{\lim_{\tau\to0^+}}
        \item
        \begin{proof}
            反证法。记$\tau=\max_j|t_{j+1}-t_j|$,
            倘若存在非零测的路径集合,其中路径
            一阶变差有限,即
            \[\limhead\sumhead\gap<\infty\]
            则由于
            \[\sumhead\gap^2\leq\tau\sumhead\gap\]
            因此二阶变差
            \[\limhead\sumhead\gap^2=0\]
            这与布朗运动几乎所有路径二阶变差为$T$相矛盾,
            因此几乎所有路径一阶变差趋于$\infty$。
        \end{proof}

        \item
        \begin{proof}
            记号同上,同样地有
            \[\sumhead\gap^3\leq\tau\sumhead\gap^2\]
            而
            \[\limhead\sumhead\gap^2=T\quad\text{a.c.}\]
            于是两端取$\tau\to0^+$即得几乎所有路径三阶变差为0。
        \end{proof}
    \end{subproblem}

    \section{习题3.5}
    \begin{proof}
        为简便起见,我们记
        \[\begin{aligned}
            d_+=d_+(T,S(0))\\
            d_-=d_-(T,S(0))\\
        \end{aligned}\]
        则有
        \[d_-+\sigma\sqrt T=d_+\]
        由于事件$S(T)>K$等价于
        \[\frac{W(T)}{\sqrt{T}}>
        \frac{1}{\sigma\sqrt{T}}
        \left(\log\frac{K}{S(0)}-\left(r-\frac{\sigma^2}2\right)T\right)=-d_-\]
        注意到$W(T)/\sqrt{T}\sim\mathcal N(0,1)$,
        于是
        \[\begin{aligned}
            \E(\e^{rT}(S(T)-K)^+)
            &=\int_{S(T)>K}\e^{-rT}(S(T)-K)\diff P\\
            &=\int_{-d_-}^\infty
              (S(0)\e^{-\frac{\sigma^2}{2}T+\sigma\sqrt{T}x}
              -K\e^{-rT})\frac{\e^{-\frac{x^2}{2}}}{\sqrt{2\pi}}\diff x\\
            &=\int_{-d_-}^\infty
              (S(0)\e^{-\frac{\sigma^2}{2}T+\sigma\sqrt{T}x}
              -K\e^{-rT})\frac{\e^{-\frac{x^2}{2}}}{\sqrt{2\pi}}\diff x\\
            &=\begin{aligned}[t]
                \int_{-d_-}^\infty S(0)\e^{-\frac{\sigma^2}{2}T+\sigma\sqrt{T}x}
                \cdot\frac{\e^{-\frac{x^2}{2}}}{\sqrt{2\pi}}\diff x\\
                -K\e^{-rT}\int_{-d_-}^\infty
                \frac{\e^{-\frac{x^2}{2}}}{\sqrt{2\pi}}\diff x
            \end{aligned}
        \end{aligned}\]
        而前一项
        \[\begin{aligned}
            \int_{-d_-}^\infty S(0)
            \e^{-\frac{\sigma^2}{2}T+\sigma\sqrt{T}x}
            \cdot\frac{\e^{-\frac{x^2}{2}}}{\sqrt{2\pi}}\diff x
            &=S(0)\int_{-d_-}^\infty
             \frac{\e^{-\frac{(x-\sigma\sqrt{T})^2}{2}}}{\sqrt{2\pi}}\diff x\\
            &=S(0)\int_{-d_--\sigma\sqrt{T}}^\infty\frac{\e^{-\frac{\xi^2}{2}}}{\sqrt{2\pi}}
             \diff\xi\\
            &=S(0)\int_{-d_+}^\infty\frac{\e^{-\frac{\xi^2}{2}}}{\sqrt{2\pi}}
             \diff\xi\\
            &=S(0)N(d_+)
        \end{aligned}\]
        于是
        \[\E(\e^{-rT}(S(T)-K)^+)=S(0)N(d_+)-K\e^{-rT}N(d_-)\]
    \end{proof}
\end{document}